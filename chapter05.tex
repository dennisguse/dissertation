\chapter{First work on Multi-episodic QoE}\label{chap:05} %state-of-the-art?
\section*{Abstract}
%NOTE: This chapter might be too small, consider merging with previous chapter.

Here I present the state-of-the-art on multi-episodic QoE (Moeller2011, Duncanson1969, Guse2013?).
Re-state the research question in more detail here again.
Major point here is the research method (task-driven, defined usage behavior, limited freedom in usage behavior).
%Hint here practical limitations (setup, for test subjects).
Duncanson actually did not multi-episodic QoE but rather \textit{expected average QoE}.
Moeller2011 developed and a applied a research methodology for multi-episodic QoE, but results were limited: low performance episode did not statistically affect multi-episodic QoE.
Guse2013 (following Moeller) found a relative slow adaptation of multi-episodic QoE.
%The two data sets cannot be used alone for modeling multi-episodic QoE.

%Duncanson: looked actually not into multi-episodic QoE, but rather what influences the expected average system performance?
%Results: low performance is bad.

Moeller (Skype): First conducted multi-episodic QoE study.
Major Result: methodology can be applied, episodic quality judgments show that performance could be provided with the used system.
Minor Results2: 
\begin{itemize}
\item increase of episodic QoE judgments over time? (significant)
\item conditions did not yield significant different multi-episodic QoE, but at least consistent
%TODO Critize study as the performance cannot be replicated, no recordings
%Limitation: no "training" and no supervised usage(?) 
%NO imposter detection!, very expansive
\end{itemize}

\section{Assessment Methods for Multi-episodic QoE}
%TODO: Discuss here potential methods for multi-episodic QoE and highlight that either a really large amount of people is necessary OR "freedom" of test subjects needs to be limited.


\section{Moeller 2011}

\section{Guse 2013}