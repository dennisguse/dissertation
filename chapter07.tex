\chapter{Experiments: One Session}\label{chap:lab}
\begin{chapter-abstract}
First question: can multi-episodic QoE be studied in "short" laboratory study?
I with my laboratory studies on with one speech telephony service focusing on effects of position(s) of degraded usage episodes (the finished Journal paper) and the already finished extensions.

Then the parallel-use studies are presented (impact of a 2nd service used in parallel).
This chapter closes with the description of cross-service QoE (using results of studies with two services) (However, results are limited).
\end{chapter-abstract}

For the evaluation of the six hypotheses with regard to multi-episodic perceived quality for sequential use in one session three experiments were conducted.
These experiments follow the same procedure, but differ in usage situation and, respectively, service type.
In the following first the procedure incl. measurements are presented, followed by an overview on the applied conditions to investigate the hypotheses.

\section{Procedure}
In pone 


\section{Conditions}
Before the three experiments are presented in detail
The presented conditions can be assessed in 
%Measurement after 6th usage episode OR DAY

\begin{table}[h]
 \centering
 \begin{tabulary}{\textwidth}{C|C||C|C|C||C||}
 Condition & \multicolumn{5}{c|}{Episodic Performance}        \\
           & 1-3	& 4           & 5           & 6           & 7-9 \\
 \midrule
 1         & HP 	& \textbf{LP} & HP          & HP          & - \\
 \hline
 2a        & HP 	& HP          & \textbf{LP} & HP          & - \\
 \hline
 2b        & HP 	& HP          & \textbf{LP}, long & HP    & - \\
 \hline
 3         & HP 	& HP          & HP          & \textbf{LP} & - \\
 \hline
 4         & HP 	& \textbf{LP} & \textbf{LP} & HP          & - \\
 \hline
 5a        & HP 	& HP          & \textbf{LP} & \textbf{LP} & - \\
 \hline
 5b        & HP 	& HP          & \textbf{LP} & \textbf{LP} & HP \\
 \hline
 6         & HP 	& \textbf{LP} & \textbf{LP} & \textbf{LP} & - \\
 \hline
 7         & HP 	& HP          & \textbf{LP} & \emph{MP}   & HP \\
 \end{tabulary}
 \caption{Overview of all conditions with the episodic performance of all usages episodes and showing which conditions are compared for each of the three hypotheses.
 Non-HP episodes are in bold (\ac{LP}) and italic (\ac{MP}).}
 \label{tab:chap06:hypothesesComparison}
\end{table}

%Describe each used service and usage situation (mobile vs. PC vs. [optional] living room)
%Describe tasks per service and requirements

\section{Experiments}
%Put table with studies here!?                                                     

\begin{table}[h]
	\begin{tabular}{|c|c|c|c|}
	Identifier	& Service type 			& Task 									& Hypotheses \\
	\hline
	S1			& Telephony				& Two-party Conversation with \ac{SCT}	& H1, H2, H3, H5 \\
	S2a			& Telephony				& Third-party Listening					& H1, H2 \\ 
	S2b			& Telephony and \ac{VoD}& Third-party Listening	and Entertainment & H1, H2 \\ 
	S3			& \ac{AoD}				& -										& H1, H2,
	\end{tabular}
\end{table}

\subsection{Design}

\subsection{Details}

\subsection{Participants}
S1 - S3


\section{One Service}
%Follow Moeller: sequential use -> seperate usage episodes
%Audio-only (avoid multi-modality)
%Controlled environment, same system for all

%Assume well working-system in the beginning for adaptation.
%Present peformance modes AND non-failure!

\subsection{Experimental Design}
%Feedback: scale + questions
%System description Appendix?
%Digitial feedback, paper allows to go back in time...
%Task: Conversation vs. Listening (Note taking) vs. Listening (retrospective content)
%Type of services
%Shared characteristics HP vs. LP (Table!) with MOS_LQO!

\begin{table}
 \centering
 \begin{tabulary}{\columnwidth}{C|C|C|C}
   Performance & Signal bandwidth & Codec & $MOS_{LQO}$ \\
   \midrule
   HP & 50 to \unit[7000]{Hz}  & G.722, Mode 1 & 4.0 \\ %MOS1-5: 3.9
   \hline
   MP & 300 to \unit[3400]{Hz} & G.711         & 3.3 \\ %MOS1-5: 3.3
   \hline
   LP & 300 to \unit[3400]{Hz} & LPC-10        & 1.9 \\ %MOS1-5: 2.0
   \end{tabulary}
   \caption{Details of performance levels (\ac{HP}, \ac{MP} and \ac{LP}) with \ac{POLQA} prediction (Mode: Super-wideband). The prediction was transformed on the continuous 7-pt scale shown in \autoref{img:chap05:quality-scale} by applying the transformation described by  \cite{koster_comparison_2015}.}
   \label{tab:performance}
\end{table}

\subsection{Experiments}


\subsection{Results}

\subsubsection*{Aspect: Task influence}
%Task (Conversation vs. Listening): their might be an effect due to mental capacity
%No Conversation related degradation...


\section{Two Services}
\subsection{Sequential use}
\begin{itemize}
\item Second \textit{unimpaired} service: reduced negative effect of LP episodes
\item Second \textit{impaired} service: same effect
\end{itemize}

%\subsection{Excurse: Parallel use} %DO NOT INCLUDE!
%Present here that using two services in parallel only has a slight impact on Web-QoE.
%First present degraded web-only and then web+TV.