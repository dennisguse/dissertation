\chapter{Multi-episodic QoE over multiple days}\label{chap:07}
\begin{chapter-abstract}
Here I present all studies that I conducted on multi-episodic QoE over several days.
This will include studies with one service only, but also with two services (multi-service) part.

\textit{Key question:} do field trials (longer timespan) yield similar effects as found in lab trials (chap 6).
What are the differences? (If there are any)
What services are technically feasible to deploy (or socially manageable)?
How to conduct a study over several days (lab vs. field)?

The main study will be the currently \textit{successfully} running Audio-on-demand study with BYOD.
This study is complemented by speech telephony (SIPGATE study) and QoMEX2014 study.
This chapter will also include an overview of limitations and practical knowledge for \textit{successful} field trials.
\end{chapter-abstract}

\section{Studies}
\begin{itemize}
%\item Jitsi + Silverlight (DAGA); Actually I would like to use this in Chap 4.
\item CSipSimple + Silverlight
\item AoD + VoD (QoMEX)
\item SIPGate Telephony
\item Field AoD: to be evaluated
\item (Sabrina, Gaming): over multiple days; is there an effect on multi-episodic? Do subjects get more (delay) sensitive?
\cite{guse_macro-temporal_2013} %Pre-Stud
\subsection{Guse 2013}
%Guse2013 (following Moeller) found a relative slow adaptation of multi-episodic QoE.
%The two data sets cannot be used alone for modeling multi-episodic QoE.
\end{itemize}

\section{Multi-episodic QoE with one service}

%TODO are there cross-service effects?

Finish this section with a comparison to laboratory studies?

%\section{Excurse: on retaining information}
%Can subjects recall

\section{Cross-service QoE}
%BLABLABLA

\section{On practical aspects of Field Studies}
\begin{itemize}
\item Production Ready Systems
\item Temporal Constraint
\item "Real" environment (subject's own context)
\item Cheater detection for consumption only
\item Drop out rate
\end{itemize}