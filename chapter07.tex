\chapter{Experiments: One Session}\label{chap:lab}
\begin{chapter-abstract}
First question: can multi-episodic QoE be studied in "short" laboratory study?
I with my laboratory studies on with one speech telephony service focusing on effects of position(s) of degraded usage episodes (the finished Journal paper) and the already finished extensions.
\end{chapter-abstract}

For the evaluation of the six hypotheses with regard to multi-episodic perceived quality for sequential use in one session four experiments were conducted.
These experiments follow the same procedure, but differ in usage situation and, respectively, service type as shown in \autoref{tab:lab:experiments}.

%Overview on Experiments
\begin{table}[h]
	\begin{tabulary}{\textwidth}{C|C|C|C}
	Experiment	& Service Type 				& Task								& Episodes \\
	\hline
	E1			& Telephony					& Two-party conversation (\ac{SCT})	& 6-9 \\
	\hline
	E2a			& Telephony					& 3rd-party listening (\ac{SCT})	& 6 \\
	\hline
	E2b			& Telephony and \ac{VoD}	& E2a with Movie					& 12 \\
	\hline
	E3			& \ac{AoD}					& Audiobook							& 6 \\
	\end{tabulary}
	\caption{Overview on conducted experiments for multi-episodic perceived quality in one session.}
	\label{tab:lab:experiments}
\end{table}

In the following, first the experimental design that is shared between the four conducted experiments is presented, followed by an overview on the applied conditions to investigate the hypotheses.

\section{Design}
For the investigation of multi-episodic perceived quality in one session service types and usage situations were selected that are based upon audio, \ie, mainly speech.
Using a unimodal rather than a multi-modal service avoids that the integration process of individual modalities into an \emph{overall perceived quality} must be considered and thus eliminates one potential influencing factor.
Varying the service type as well as usage situation, it is not yet known, how if those affect multi-episodic perceived quality.

All experiments consisted of at least 6 usage episodes.
Each usage episode should lead to a minimal usage duration of \unit[2]{min}.
This is, in fact, only an issue for experiment E1 due to the fact that in this experiment, the user behavior affects the duration of an episode, which is not the case in the other three experiments.

For all experiments the first three episodes were always presented with the highest service performance (\ie, \acf{HP}).
This enables participants could experience the service in a well-working setting \citep[\cf,][]{moller_single-call_2011}.
Non-\ac{HP} were only introduced per service for episode 4, 5, and 6.

In all experiments a multi-episodic judgment on the 7-point \ac{CCR} is taken after finishing every third episode.

\subsection{Conditions}
Overall 10 conditions were created that allow to investigate the six hypotheses in detail (\cf, \autoref{chap:towards}).
All conditions are shown in \autoref{tab:lab:hypothesesComparison}.
Differences between conditions can be evaluated by comparing multi-episodic judgments that were taken after the same usage episode, \ie participants experienced the same number of usage episodes so far.

\begin{table}[h]
 \centering
 \begin{tabulary}{\textwidth}{C|C||C|C|C||C||}
 Condition & \multicolumn{5}{c|}{Episodic Performance}        \\
           & 1-3	& 4           & 5           & 6           & 7-9 \\
 \midrule
 1         & HP 	& \textbf{LP} & HP          & HP          & - \\
 \hline
 2a        & HP 	& HP          & \textbf{LP} & HP          & - \\
 \hline
 2b        & HP 	& HP          & \textbf{LP}, long & HP    & - \\
 \hline
 3         & HP 	& HP          & HP          & \textbf{LP} & - \\
 \hline
 4         & HP 	& \textbf{LP} & \textbf{LP} & HP          & - \\
 \hline
 5a        & HP 	& HP          & \textbf{LP} & \textbf{LP} & - \\
 \hline
 5b        & HP 	& HP          & \textbf{LP} & \textbf{LP} & HP \\
 \hline
 6         & HP 	& \textbf{LP} & \textbf{LP} & \textbf{LP} & - \\
 \hline
 7         & HP 	& HP          & \textbf{LP} & \emph{MP}   & HP \\
 \end{tabulary}
 \caption{Overview of all conditions with the episodic performance of all usages episodes.
 Non-HP episodes are in bold (\ac{LP}) and italic (\ac{MP}).}
 \label{tab:lab:hypothesesComparison}
\end{table}

H1, \ie, increasing the number of \ac{LP} episodes reduces the following multi-episodic judgment, can be investigated by comparing the results of conditions 3, 5 (a+b) and 6 as well as conditions 2a and 4.
The former presents an increasing number of \ac{LP} episodes directly before the multi-episodic judgment whereas the latter presents the last episode before the multi-episodic judgment in \ac{HP}.

H2 focuses on the position of the \ac{LP} episodes towards the following multi-episodic judgment.
Following the so-called recency effect, it is expected that increasing the number of \ac{HP} episodes before the multi-episodic judgment reduces the negative effect of presented \ac{LP} episodes.
This can be investigated by comparing the multi-episodic judgment of conditions 1, 2, and 3 as well as conditions 4 and 5 (a+b) for one and, respectively, two LP usage episodes.

H3 is similar to H2 as the position of \ac{LP} episodes towards the final multi-episodic judgment is varied, but focuses on recovery between two consecutive multi-episodic judgments due to presentation of additional \ac{HP} episodes.
This can be evaluated by comparing the results of conditions 5b and 7, which are both extended by an additional block of three \ac{HP} episodes.

H4 




\subsection{Applied Performance Level}
In all four experiments similar degradations were 

The presented conditions can be assessed in 

%Describe each used service and usage situation (mobile vs. PC vs. [optional] living room)
%Describe tasks per service and requirements

\subsection{Procedure} % Design

Usage Episodes: number and length (approx): 3-9min
Experiment duration: 60-90min.
Versuchspersonenanzahl per Condition (15)?

Phase 1: Initial Questionnaires
Phase 2: Preparation (Audiometer)
Phase 3: Training - Bewertung Sprachstimuli (Telefoniestoerungen und Video)
Phase 4: Episodic Usage and final Questionnaire

Feedback:
- CCR scale: training, retrospective episodic and retrospective multi-episodic 
- Beschreibung der Stoerungen per Usage

Applied Service performance levels

Telefonie (Papier; PEAK: digital)
- NPS
- Anzahl gestoerter Telefonate (Freitext vs. 0-9)
- Welche (Freitext vs. 0-9)?
- Dauer der Stoerung (Ja/Nein)
-> Telefonpartner bekannt

LST
- NPS
- Wie viele Telefonate waren gestört?
- Falls Sie sich erinnern, welche Telefonat(e) waren gestört?
- Sind die Störungen nur kurzzeitig (also nur Teil eines Telefonates war betroffen) aufgetreten?

LST+Video: Digital
- NPS
- Wie viele Telefonate waren gestört? 
- Falls Sie sich erinnern, welche Telefonat(e) waren gestört? (Freitext)
- Sind die Störungen nur kurzzeitig (also nur Teil eines Telefonates war betroffen) aufgetreten?

- Wie viele Videos waren gestört?
- Falls Sie sich erinnern, welche Videos waren gestört?
- Sind die Störungen nur kurzzeitig innerhalb des Videos aufgetreten?

AOD:
- NPS
- War eine Episode gestoert
- Waren alle Episode gleich lang
- Wie lang war ein Episode?

\section{Results}
%Stichprobenbeschreibung
%Hypothekenweise durch die Studien durchgehen
\section{Discussion}
\begin{table}[h]
	\begin{tabular}{|c|c|c|c|}
	Identifier	& Service type 			& Task 									& Hypotheses \\
	\hline
	S1			& Telephony				& Two-party Conversation (\ac{SCT})	& H1, H2, H3, H5 \\
	S2a			& Telephony				& Third-party Listening	(\ac{SCT})		& H1, H2 \\ 
	S2b			& Telephony and \ac{VoD}& Third-party Listening	(\ac{SCT}) and Video Entertainment & H1, H2 \\  % on mobile device
	S3			& \ac{AoD}				& -										& H1, H2,
	\end{tabular}
\end{table}

\subsection{Details}

