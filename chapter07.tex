\chapter{Experiments: One Session}\label{chap:lab}
\begin{chapter-abstract}
First question: can multi-episodic QoE be studied in "short" laboratory study?
I with my laboratory studies on with one speech telephony service focusing on effects of position(s) of degraded usage episodes (the finished Journal paper) and the already finished extensions.
\end{chapter-abstract}

For the evaluation of the six hypotheses with regard to multi-episodic perceived quality for sequential use in one session four experiments were conducted.
These experiments follow the same procedure, but differ in usage situation and, respectively, service type as shown in \autoref{tab:lab:experiments}.

%Overview on Experiments
\begin{table}[h]
	\begin{tabulary}{\textwidth}{C|C|C|C}
	Experiment	& Service Type 				& Task								& Episodes \\
	\hline
	E1			& Telephony					& Two-party conversation (\ac{SCT})	& 6-9 \\
	\hline
	E2a			& Telephony					& 3rd-party listening (\ac{SCT})	& 6 \\
	\hline
	E2b			& Telephony and \ac{VoD}	& E2a with Movie					& 12 \\
	\hline
	E3			& \ac{AoD}					& Audiobook							& 6 \\
	\end{tabulary}
	\caption{Overview on conducted experiments for multi-episodic perceived quality in one session.}
	\label{tab:lab:experiments}
\end{table}

In the following, first the experimental design that is shared between the four conducted experiments is presented, followed by an overview on the applied conditions to investigate the hypotheses.

\section{Design}
For the investigation of multi-episodic perceived quality in one session service types and usage situations were selected that are based upon audio, \ie, mainly speech.
Using a unimodal rather than a multi-modal service avoids that the integration process of individual modalities into an \emph{overall perceived quality} must be considered and thus eliminates one potential influencing factor.
Varying the service type as well as usage situation, it is not yet known, how if those affect multi-episodic perceived quality.

All experiments consisted of at least 6 usage episodes.
Each usage episode should lead to a minimal usage duration of \unit[2]{min}.
This is, in fact, only an issue for experiment E1 due to the fact that in this experiment, the user behavior affects the duration of an episode, which is not the case in the other three experiments.

For all experiments the first three episodes were always presented with the highest service performance (\ie, \acf{HP}).
This enables participants could experience the service in a well-working setting \citep[\cf,][]{moller_single-call_2011}.
Non-\ac{HP} were only introduced per service for episode 4, 5, and 6.

In all experiments a multi-episodic judgment on the 7-point \ac{CCR} is taken after finishing every third episode.

\subsection{Conditions}
Overall 10 conditions were created that allow to investigate the six hypotheses in detail (\cf, \autoref{chap:towards}).
All conditions are shown in \autoref{tab:lab:hypothesesComparison}.
Differences between conditions can be evaluated by comparing multi-episodic judgments that were taken after the same usage episode, \ie participants experienced the same number of usage episodes so far.

%TODO Abkuerzung aller Conditions C1-C7?

\begin{table}[h]
 \centering
 \begin{tabulary}{\textwidth}{C|C||C|C|C||C||}
 Condition & \multicolumn{5}{c|}{Episodic Performance}        \\
           & 1-3	& 4           & 5           & 6           & 7-9 \\
 \midrule
 1         & HP 	& \textbf{LP} & HP          & HP          & - \\
 \hline
 2a        & HP 	& HP          & \textbf{LP} & HP          & - \\
 \hline
 2b        & HP 	& HP          & \textbf{LP}, long & HP    & - \\
 \hline
 3         & HP 	& HP          & HP          & \textbf{LP} & - \\
 \hline
 4         & HP 	& \textbf{LP} & \textbf{LP} & HP          & - \\
 \hline
 5a        & HP 	& HP          & \textbf{LP} & \textbf{LP} & - \\
 \hline
 5b        & HP 	& HP          & \textbf{LP} & \textbf{LP} & HP \\
 \hline
 6         & HP 	& \textbf{LP} & \textbf{LP} & \textbf{LP} & - \\
 \hline
 7         & HP 	& HP          & \textbf{LP} & \emph{MP}   & HP \\
 \hline
 8         & HP 	& HP          & HP          & HP          & HP \\
 \end{tabulary}
 \caption{Overview of all conditions with the episodic performance of all usages episodes.
 Non-HP episodes are in bold (\ac{LP}) and italic (\ac{MP}).}
 \label{tab:lab:hypothesesComparison}
\end{table}

H1, \ie, increasing the number of \ac{LP} episodes reduces the following multi-episodic judgment, can be investigated by comparing the results of conditions 3, 5 (a+b) and 6 as well as conditions 2a and 4.
The former presents an increasing number of \ac{LP} episodes directly before the multi-episodic judgment whereas the latter presents the last episode before the multi-episodic judgment in \ac{HP}.

H2 focuses on the position of the \ac{LP} episodes towards the following multi-episodic judgment.
Following the so-called recency effect, it is expected that increasing the number of \ac{HP} episodes before the multi-episodic judgment reduces the negative effect of presented \ac{LP} episodes.
This can be investigated by comparing the multi-episodic judgment of conditions 1, 2, and 3 as well as conditions 4 and 5 (a+b) for one and, respectively, two LP usage episodes.

H3 is similar to H2 as the position of \ac{LP} episodes towards the final multi-episodic judgment is varied, but focuses on recovery between two consecutive multi-episodic judgments due to presentation of additional \ac{HP} episodes.
This can be evaluated by comparing the results of conditions 5b and 7, which are both extended by an additional block of three \ac{HP} episodes.

H4 focuses on the impact of duration of one \ac{LP} episode on a following multi-episodic judgment by expecting a higher reduction, if a longer \ac{LP} episode is presented.
This is evaluated by comparing conditions 2a versus 2b, which both provide the 5th usage episode in \ac{LP} but with a different length.
Whereas episodes of condition 2a are presented in a similar length, the \ac{LP} episode in condition 2b is doubled in duration compared.
Presenting the \ac{LP} episode as the 5th episode followed by one \ac{HP} episode is chosen, so that participants are suggested the difference in duration.
Doubling the duration should enforce a measurable effect, if duration is not neglected.
In fact, condition 5b is similar to condition 4 as both present the same overall duration of \ac{LP}, which is split in condition 4 into two episodes, followed by one \ac{HP} episode before the multi-episodic judgment.

H5 focuses on a potential existing peak-effect, \ie a multi-episodic judgment is larger affected by the lowest episodic performance. 
Such an effect has been observed in the retrospective assessment of general experiences, but also for retrospective assessment of episodic perceived quality.
This is investigated by presenting in addition to \ac{HP} and \ac{LP} a third performance level \ac{MP}, which provides a performance between \ac{HP} and \ac{LP}.
H5 can be investigated by comparing condition 7, which presents the 5th episode in \ac{LP} and the 6th episode in \ac{MP}, with condition 5 (a+b) and condition 2a with regard to the multi-episodic judgment after the 6th usage episode.
Those three condition differ only in the performance of the 6th episode presenting it in \ac{HP}, \ac{MP}, and \ac{LP}.
If a peak-effect occurs in the formation process of multi-episodic perceived quality, the result of condition 7 should not differ from the result of condition 2a.

H6 it is hypothesized that multi-episodic perceived quality is judged on a per-service basis, \ie, usage of different, distinct service types in the same usage period does not affect each others multi-episodic judgment.
Following the experimental approach of sequential usage episodes this can be investigated by presenting a second service in the same usage period, where both services are used sequentially.
This is investigated for two overall conditions presenting the first service always in condition 4, \ie the 5th and 6th usage episodes is presented in \ac{LP}.
The 2nd service is presented in two conditions to investigate, if the condition of the 2nd service affect the judgment of the first service.
The second service is presented in condition 8, \ie, all episodes in \ac{HP}, as well as condition 4.

%Describe each used service and usage situation (mobile vs. PC vs. [optional] living room)
%Describe tasks per service and requirements

\subsection{Performance Levels}
In each of the four experiments a speech-only service was used.
By using similar manifestations for the performance levels, the multi-episodic judgments for the same conditions between different experiments can be compared and thus draw conclusion about potential impact of the usage situation as well as service type on the results.

Three performance levels (\ac{HP}, \ac{MP}, and \ac{LP}) must be selected in a way that episodic judgments decrease, and that \ac{LP} is very likely to produce a measurable effect on multi-episodic judgments.
Thus, the \ac{LP} should be not only noticeable but rather present a \emph{severe} reduction in performance. 
However, it must ensured that solving a task is not prevented, if \ac{MP} or \ac{LP} are applied.
All performance level should lead to \emph{constant} impairments rather than fluctuations in one episode over time.
This can be achieved for digital transmission systems in the encoding and compression stage, \eg, selection and configuration of the codec.
Introducing degradations by coding and compression alone enables to apply those performance levels to listening-only as well as a conversational usage situation.

%TODO! BE CAREFUL with MOS here!
\ac{HP} can be properly chosen by selecting a current configuration that provides state-of-the-art performance level.
For speech telephony this is at the time of this writing the transmission in wideband with proper loudness, but without loss, noise, echo, reverberation or other negative factors.
For digital transmission of wideband speech the codec \emph{G.722} with Mode 1 (\unit[64]{kbit/s}) is often used reference condition as it achieves very similar quality judgments like pure wideband filtered speech signals. %TODO Add refs, Moeller, Raake, Marcel
%Furthermore, this codec is although it is rather old, but still in active use.
In subjective experiments G.722 with Mode 1 typically achieves a \ac{MOSLQ} around 4.5 on 5-point \ac{ACR} scale.

For \ac{LP} the speech signal is compressed with the \emph{LPC-10} codec is selected. %TODO REF
This codec is designed for low bitrate radio transmission focusing on intelligibility rather than naturalness.
The re-synthesized speech sounds very unnatural and is often described as robotic, and muddy with a hissing background noise.
LPC-10 achieves a 5-point \ac{MOSLQ} of 2.3~\cite{gibson book}.
In fact, LPC-10 is not designed for speech telephony especially due to the focus on a low bitrate and the resulting degradations of the speech signal.
However, available speech telephony codecs provide a rather \ac{MOS} scores.
Even the worst codec, \ie, \ac{MELP}, only results in a 5-point \ac{MOS} of 3.0. 
Therefore, LPC-10 is well suited for the evaluation of multi-episodic perceived quality although it is a codec that is not used for speech telephony, but provides an intelligble re-synthesized speech signal with severe and thus expected memorable reduction in quality.



\begin{table}
 \centering
 \begin{tabulary}{\columnwidth}{C|C|C|C}
   Performance & Signal bandwidth & Codec & $MOS_{LQO}$ \\
   \midrule
   \ac{HP} & 50...\unit[7000]{Hz}  & G.722, Mode 1 & 4.0 \\ %MOS1-5: 3.9
   \hline
   \ac{MP} & 300...\unit[3400]{Hz} & G.711         & 3.3 \\ %MOS1-5: 3.3
   \hline
   \ac{LP} & 300...\unit[3400]{Hz} & LPC-10        & 1.9 \\ %MOS1-5: 2.0
   \end{tabulary}
   \caption{Details of performance levels (HP, MP and LP) with POLQA prediction (Mode: Super-wideband).
   The prediction was transformed on the continuous 7-pt scale shown in  \autoref{img:chap05:quality-scale} by applying the transformation described in \cite{koster_comparison_2015}.}
   \label{tab:performance}
\end{table}

For a two-party conversation the end-to-end delay is an additional factor influencing perceived quality.
Providing a proper one-way delay up to \unit[100]{ms} is rarely noticeable and is not perceived as degradation~\citep[\cf,][p. 9]{itu-t_g.107:_2005}.





























\subsection{Procedure} % Design

Usage Episodes: number and length (approx): 3-9min
Experiment duration: 60-90min.
Versuchspersonenanzahl per Condition (15)?

Phase 1: Initial Questionnaires
Phase 2: Preparation (Audiometer)
Phase 3: Training - Bewertung Sprachstimuli (Telefoniestoerungen und Video)
Phase 4: Episodic Usage and final Questionnaire

Feedback:
- CCR scale: training, retrospective episodic and retrospective multi-episodic 
- Beschreibung der Stoerungen per Usage

Applied Service performance levels

Telefonie (Papier; PEAK: digital)
- NPS
- Anzahl gestoerter Telefonate (Freitext vs. 0-9)
- Welche (Freitext vs. 0-9)?
- Dauer der Stoerung (Ja/Nein)
-> Telefonpartner bekannt

LST
- NPS
- Wie viele Telefonate waren gestört?
- Falls Sie sich erinnern, welche Telefonat(e) waren gestört?
- Sind die Störungen nur kurzzeitig (also nur Teil eines Telefonates war betroffen) aufgetreten?

LST+Video: Digital
- NPS
- Wie viele Telefonate waren gestört? 
- Falls Sie sich erinnern, welche Telefonat(e) waren gestört? (Freitext)
- Sind die Störungen nur kurzzeitig (also nur Teil eines Telefonates war betroffen) aufgetreten?

- Wie viele Videos waren gestört?
- Falls Sie sich erinnern, welche Videos waren gestört?
- Sind die Störungen nur kurzzeitig innerhalb des Videos aufgetreten?

AOD:
- NPS
- War eine Episode gestoert
- Waren alle Episode gleich lang
- Wie lang war ein Episode?
% gleicher sprecher 

\section{Results}
%Stichprobenbeschreibung
%Hypothekenweise durch die Studien durchgehen
\section{Discussion}
\begin{table}[h]
	\begin{tabular}{|c|c|c|c|}
	Identifier	& Service type 			& Task 									& Hypotheses \\
	\hline
	S1			& Telephony				& Two-party Conversation (\ac{SCT})	& H1, H2, H3, H5 \\
	S2a			& Telephony				& Third-party Listening	(\ac{SCT})		& H1, H2 \\ 
	S2b			& Telephony and \ac{VoD}& Third-party Listening	(\ac{SCT}) and Video Entertainment & H1, H2 \\  % on mobile device
	S3			& \ac{AoD}				& -										& H1, H2,
	\end{tabular}
\end{table}

\subsection{Details}

