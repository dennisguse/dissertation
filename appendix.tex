\chapter{Appendix}\label{chap:appendix}
\begin{chapter-abstract}
In the appendix I include all information about the presented studies.
This includes all details that are necessary for understanding the results, but are important for reproduction (additional data and plots as well as a detailed technical description of the setups).
\end{chapter-abstract}


\section{One-session Experiments}\label{appendix:laboratorySetups}

\subsection{\acs{SCT} used in E1, E2, and E3}
\begin{table}
	\centering
	\begin{tabular}{c|c|c}
	Episode & Title & Page \\
	\hline
	1 & Travel Agent (German: Reisebüro) & p. 16f \\
	2 & Rail Travel Information (German: Bahnauskunft) & p. 18f \\
	3 & Theater Box office (German: Theaterkasse) & p. 30f \\
	4 & Renting a car (German: Autovermietung) & p. 26f \\
	5 & Information on Flights (German: Flugumbuchung) & Based upon p. 24f \\
	6 & Eye Specialist's Appointment (German: Arzttermin) & p. 50f \\
	7 & Pizza Service (German: Pizzaservice) & p. 38f \\
	8 & Flat to Let (German: Wohnungsanzeige) & p. 46f \\
	9 & Booking an Apartment (German: Appartmentreservierung) & p. 36f \\
	\end{tabular}
	\caption{The \acs{SCT} used in E1, E2, and E3 were taken from \cite{itu-t_p.805:_2007} and out-dated information updated. The scenario of episode 4 was changed from a flight information and booking task to a change of booking task.}
	\label{tab:appendix:labsct}
\end{table}

\subsection{Setups}

\subsubsection{Experiment E1: Two-party Conversation}
In E1 the third stage, \ie, presentation of short speech stimuli, was conducted on a tablet computer (\emph{Fujitsu Stylistic ST6012}).
For binaural representation a pair of \emph{AKG K-271} was connected to the internal soundcard of the tablet computer.
The system was calibrated using a \emph{HEAD acoustics} head and torso simulator \emph{HSM II.3} to a sound pressure level of \unit[75]{dB20$\mu$Pa} using babble noise.
In the second part, each participant received a binaural representation with a pair of \emph{Beyerdynamic DT 790 Pro}, which were both connected to one \emph{Edirol UA25-EX}.
Both sides were calibrated to a comfortable listening level.
The two participants were placed into separate rooms according ITU-T P.800~\citep{itu-t_p.800:_1996}.

-> No VAD, No AGC, No denoising necessary, no comfort noise (absolute silence), no loss
No additional IRS filter or similar; no headphone response compensation; no sidetone.

System change: 
1. Edirols  + PJSIP (SLIN16) + Asterisk (internal transcoding); ethernet
2. RME QuadMic II, Edirol, Puredata  


\subsubsection{Experiment E2a: Third-party Listening}

In Study~2 a pair of \emph{Sennheiser HD 25-1} headphones was used for sound reproduction connected to the internal soundcard of a \emph{Microsoft Surface Pro}.
The sound pressure level was calibrated to \unit[75]{dB20$\mu$Pa} using the \emph{HSM II.3}.
This study was conducted in a sound-proof cabin following ITU-T P.800~\citep{itu-t_p.800:_1996}.

\subsubsection{Experiment E2b: Third-party Listening and \ac{VoD}}


\subsubsection{Experiment E3: \ac{AoD}}


\section{Experiments: Multiple Days}



