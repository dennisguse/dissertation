\chapter{Appendix}\label{chap:appendix}
\begin{chapter-abstract}
In the appendix I include all information about the presented studies.
This includes all details that are necessary for understanding the results, but are important for reproduction (additional data and plots as well as a detailed technical description of the setups).
\end{chapter-abstract}


\section{One-session Experiments}

\subsection{\acs{SCT} used in E1, E2, and E3}
\begin{table}[h]
	\centering
	\begin{tabular}{c|c|c}
	Episode & Title & Page \\
	\hline
	1 & Travel Agent (German: Reisebüro) & p. 16f \\
	2 & Rail Travel Information (German: Bahnauskunft) & p. 18f \\
	3 & Theater Box office (German: Theaterkasse) & p. 30f \\
	4 & Renting a car (German: Autovermietung) & p. 26f \\
	5 & Information on Flights (German: Flugumbuchung) & Based upon p. 24f \\
	6 & Eye Specialist's Appointment (German: Arzttermin) & p. 50f \\
	7 & Pizza Service (German: Pizzaservice) & p. 38f \\
	8 & Flat to Let (German: Wohnungsanzeige) & p. 46f \\
	9 & Booking an Apartment (German: Appartmentreservierung) & p. 36f \\
	\end{tabular}
	\caption{The \acs{SCT} used in E1, E2, and E3 were taken from \cite{itu-t_p.805:_2007} and out-dated information updated. The scenario of episode 4 was changed from a flight information and booking task to a change of booking task.}
	\label{tab:appendix:labsct}
\end{table}

<<<<<<< HEAD
\subsection{Setups}

\subsubsection{Experiment E1: Two-party Conversation}
In E1 the third stage, \ie, presentation of short speech stimuli, was conducted on a tablet computer (\emph{Fujitsu Stylistic ST6012}).
For binaural representation a pair of \emph{AKG K-271} was connected to the internal soundcard of the tablet computer.
The system was calibrated using a \emph{HEAD acoustics} head and torso simulator \emph{HSM II.3} to a sound pressure level of \unit[75]{dB20$\mu$Pa} using babble noise.
In the second part, each participant received a binaural representation with a pair of \emph{Beyerdynamic DT 790 Pro}, which were both connected to one \emph{Edirol UA25-EX}.
Both sides were calibrated to a comfortable listening level.
The two participants were placed into separate rooms according ITU-T P.800~\citep{itu-t_p.800:_1996}.

-> No VAD, No AGC, No denoising necessary, no comfort noise (absolute silence), no loss
No additional IRS filter or similar; no headphone response compensation; no sidetone.

System change: 
1. Edirols  + PJSIP (SLIN16) + Asterisk (internal transcoding); ethernet
2. RME QuadMic II, Edirol, Puredata  

=======
\subsection{Setups}\label{appendix:laboratorySetups}

\subsection{Experiment E1: Two-party Conversation}
In E1 the third stage, \ie, quality assessment of short speech stimuli as training, was conducted on a tablet computer (\emph{Fujitsu~Stylistic~ST6012}).
For binaural representation a pair of \emph{AKG~K-271} was connected to the internal soundcard of the tablet computer.
The system was calibrated using a \emph{HEAD acoustics} head and torso simulator \emph{HSM~II.3} to a sound pressure level of \unit[75]{dB20$\mu$Pa} for babble noise.

The multi-episodic assessment, \ie, the fourth stage, a speech telephony system for two-party conversations was required.
In the first part of the experiment a \ac{VoIP}-based system consisting of three computers was used.
Those three computers were connected via Ethernet (CAT-5). 
One computer (\emph{Lenovo X61}) acted as a Server running the open-source telephony server \emph{Asterisk 11}\footnote{\url{http://www.asterisk.org}}.
Two \emph{Fujitsu Siemens TODO} were running each a custom client based-upon \emph{PJSIP~2.1}\footnote{PJSIP is an open-source library for \ac{SIP}-based \ac{VoIP}}, which were connected via \ac{SIP} with the server.
The custom client presented only a minimal \ac{UI} consisting of one button for call initiation and hangup, one checkbox to set own presence status, and an image presenting the present status of the other client.
On each of the two client computers one \emph{Beyerdynamic~DT~790~Pro} headset connected to a \emph{Edirol~UA-25EX} was used for binaural reproduction and recording.
%The headset uses closed headphones ($80 \ohm$, \unit[5...35.000}[MHz]), which provide a ambient noise reduction of approximately \unit[35]{dBA}. https://www.beyerdynamic.de/shop/media/datenblaetter/DAT_DT790_EN_A4.pdf
The output on both systems was calibrated to a comfortable listening level. %TODO REF?
The clients did not send transmit the speech data directly with each other, but rather the signal was relayed via Asterisk.
Transmission of the speech signal between Asterisk and client in both direction was lossless by using the \emph{L16} codec with a sample rate of \unit[16]{kHz} resulting in a unidirectional net transmission rate of \unit[256]{kbit/s}. %TODO CITE RFC3551
Asterisk applied the desired performance level, \ie, codec, by compression the signal and immediately decompressing it before relaying the signal.\footnote{The performance levels were set by using the transcoding capability of Asterisk. On an incoming call coded with L16, Asterisk initiated a call to himself with the desired codec (\ie, G.722, or LPC-10). The later call triggered an outgoing call to the callee encoded in L16. This requires that the call loop detection is disabled in Asterisk.}
This system achieved a one-way end-to-end delay of \unit[120]{ms}.

This system was in the second part replaced by a more elaborate system, which provides easier setup, maintenance, and verification.
For this system only one computer without a network for speech transmission is used.
Rather than transmitting the signals via digital via Ethernet, analogue transmission via audio cables is used.
In this system also the \emph{Beyerdynamic~DT~790~Pro} headsets were used.
The two headsets were connected to the processing computer (\emph{Lenovo X61}) with one \emph{Edirol~UA-25EX}.
As the individual signals were transmitted using analogue audio cables with a length of \unit[10]{m}, the microphone signals were amplified beforehand using a \emph{RME QuadMic II} microphone preamplifier.
On the computer \emph{PureData}, an open-source audio processing program, is used to modify the speech signals.
For this setup PureData was extended by the G.722 as well as LPC-10 as no speech codecs were available in PureData.
It was verified that the PureData setup provides similar characteristics as the Asterisk setup.
For G.722 this system achieved a constant, glitch-free one-way end-to-end delay of \unit[70]{ms}.

In both setups neither comfort noise nor side tone was presented.
Furthermore, no \ac{VAD}, \ac{AGC}, denoising, or echo cancellation was used, because the usage of \emph{Beyerdynamic~DT~790~Pro} in a room according ITU-T P.800~\citep{itu-t_p.800:_1996} is not necessary.
As not a precise reproduction of standardized telephone handsets was not required, audio signals were not IRS filtered.
>>>>>>> 165920a47d9f2adb4b029d444c0660c397bdd1b7

\subsubsection{Experiment E2a: Third-party Listening}
In experiment E2a a pair of \emph{Sennheiser HD 25-1} headphones was used for sound reproduction connected to the internal soundcard of a \emph{Microsoft Surface Pro}.
The sound pressure level was calibrated to \unit[75]{dB20$\mu$Pa} using the \emph{HSM II.3}.
This study was conducted in a sound-proof cabin following ITU-T P.800~\citep{itu-t_p.800:_1996}.

\subsubsection{Experiment E2b: Third-party Listening and \ac{VoD}}


\subsubsection{Experiment E3: \ac{AoD}}


\section{Experiments: Multiple Days}



