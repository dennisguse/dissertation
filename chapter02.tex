\chapter{Quality of Experience}
\section*{Abstract}
Here I give an overview on the concept of Quality of Experience.
This is basically Psychophysics (Perception), Möller, Jekosch and Raake.
I present standard assessment methods (assumption for user studies).
Higher level concepts must be included: ARCU and Geerts + Service Quality
I should here include also some technical examples why QoE is important for IP-based networks.
Specialized methods for assessment of temporal effects / performance fluctuations are presented in \ref{chap:04}.

\begin{itemize}
\item What does this concept include?
\item What are related concepts: QoS, Performance?
\item What is experiencing? (cite book)
\item What is perception?
\item What is psychophysics?
\item What leads to an experience? Perceptual event etc.
\item Definition of performance (put QoS here!)

\item (Physical) Event: An obervable occurance with time, location, character [Whitepaper] -> Duration? [me]
    
\item Perceptual Event (with Sensory Processing [QoE Book Chap 2]): A physical event may(!) trigger a perceptual event.
\item -> When does a physical event trigger a perceptual event? (Attention?, Focus?)
\item -> Behavioral impact of perceptual events: anticipation and matching and explorative actions [QoE Book Chap 2]
    
\item Experience: An experience is an individual's stream of perception and interpretation of one or multiple events [Whitepaper].
\item (Perceptual) Quality: Individual comparison and judgment process with perception, reflection (optional?) and description of the outcome (optional?) [Whitepaper]
\item [Jekosch, Raake, Whitepaper]

\item How is this studied? How is QoE typically assessed? (Methods)?
\item Definitions: QoE
    
\item Expectations: Prior experiences, Desired nature, Internal reference, Contextual Factors, Task

\item Perceptual Quality Features, Quality Elements, Perceptual dimensions [Moeller, Raake]
\item Quality as hygenic factor [Moeller, Wechsung]

\item Semiotic triangle [Raake, Buch p.2]

\item Open questions: How do experience and perceptual event come together? Is an experience a perceptual event?

\item Practical outcome? Evaluation procedures, knowledge and Objective Models (for different purpose)
\end{itemize}

Related concepts
\begin{itemize}
\item What is satisfaction (+money) and acceptance; is their an alternative?
\item Relatedness to Utility [Kahnemann]

\item Types of judgment: momentary and retrospective
\item Quality formation process

\item Practical outcome: objective assessment and 

\item ARCU + Geerts
\item performance fluctuations
\item Service quality
\end{itemize}