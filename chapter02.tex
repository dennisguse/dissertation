\chapter{Quality of Experience}
\section*{Abstract}
Here I give an overview on the concept of Quality of Experience.
This is basically Psychophysics (Perception): Möller, Jekosch and Raake.
I present standard assessment methods (ACR, MUSHRA etc.) and underlying assumption for user studies. %TODO Why here?

I will include higher level concepts (ARCU and Geerts), which propose a more holistic approach towards QoE.
I should here include also some technical examples why QoE is important for IP-based networks.
Specialized methods for assessment of temporal effects / performance fluctuations are presented in Chapter~\ref{chap:04}.

\section{Perception and Psychophysics}
\begin{itemize}
\item What is perception?
\item What is psychophysics? Where does it come from?
\item Perceptual Event (with Sensory Processing [QoE Book Chap 2]): A physical event may(!) trigger a perceptual event.
\item (Physical) Event: An observable occurrence with time, location, character [Whitepaper] -> Duration? [me]
\item Inherent assumption: perception is more or less reproducible: humans perceive similar

\item What is experiencing? (cite book)
\item Experience: An experience is an individual's stream of perception and interpretation of one or multiple events [Whitepaper].
\item Definition of physical observable things may lead to perception AND may be memorized.
\item Behavioral impact of perceptual events: anticipation and matching and explorative actions [QoE Book Chap 2]
\end{itemize}

\section{Perceptual Quality}
\begin{itemize}
\item (Perceptual) Quality: Individual comparison and judgment process with perception, reflection (optional?) and description of the outcome (optional?) [Whitepaper]
\item Quality formation process
\item Assumed quality + recalled quality
\item Expectations: Prior experiences, Desired nature, Internal reference, Contextual Factors, Task [Moeller, Raake]
\item Perceptual Quality Features, Quality Elements, Perceptual dimensions [Moeller, Raake]
\item Semiotic triangle [Raake, Buch p.2]: sign carrier, referent and meaning; (is similar to Shannon)
\item What are related concepts: QoS, Performance?

\item Check Geerts for broader context!
\item REF Gaming taxonomy, if used later!
\end{itemize}

\begin{definition}[Quality of Experience (QoE)]
``is the degree of delight or annoyance of a person whose experiencing involves an application, service, or system. It results from the person’s evaluation of the fulfillment of his or her expectations and needs with respect to the utility and / or enjoyment in the light of the person’s context, personality and current state''~\citep{moller_quality_2014}.
\end{definition}

\section{Application of Quality of Experience}
\begin{itemize}
\item How is this studied? How is QoE typically assessed? (Methods)? 
\item QoE as hygienic factor [Moeller, Wechsung]  
\item Practical outcome? Evaluation procedures, knowledge and Objective Models (for different purpose)

\item Approaches towards modeling: how to one create a model? What are limitations?
\item Types of judgment: momentary and retrospective
\item Performance fluctuations: temporal effects; outlook to next chapter.
\end{itemize}

\section{Related concepts}
\begin{itemize}
\item Definition of Utility [Kahneman and also Moeller]
\item Definition of Acceptability with regard to Service
\item Definition of Satisfaction (if required?)
\item Service quality
\end{itemize}