\chapter{Conclusion}\label{chap:discussion}
%The outlook will include three distinct research directions for follow-up PhDs.
%Highlight that reproducible research is very important

%Research question
In this thesis, I investigated the quality formation process of perceived quality.
Here, I focused on formation of an overall perceived quality over individual, meaningful, and distinct interactions, \ie, usage episodes, with a telecommunication service.
Based upon initial work by \citet{moller_single-call_2011}, I applied the defined-use method to investigate the formation process for two different usage periods.
Here, I investigated one session consisting of several usage episodes and multiple individual usage episodes split over several days.
Considering one session alone allowed to establish a basis for the investigation of multiple days.
Here, most important is the ability to evaluate one session in an experiment under controlled laboratory terms.
This reduces complexity, effort, and thus allows to investigate a more conditions.

Beside the implementation of a prediction model based upon episodic judgments, I evaluated factors affecting the formation process of multi-episodic perceived quality.
To achieve those two goals, I conducted six experiments.
Here, I investigated the number, position, and duration of degraded episodes, and impact of a second service.
In addition, the occurrence of a peak-effect was investigated.

In the empirical experiments, I adapted the defined-use method of \cite{moller_single-call_2011}.
In a similar manner, a constant performance was applied to avoid impact macroscopic performance fluctuations as effects on episodic and multi-episodic judgments is not yet fully understood.
Here, severe, but unrealistic degradations were introduced, so a measurable effect on multi-episodic judgments is created.
Here, \emph{new} services were created to avoid an influence of prior experiences with a service.
%Especially, the necessary between-subject 
In later experiments, typical degradations per service type were presented before the multi-episodic judgment to avoid adjustment in use of the scales over the usage periods.
This might set expectations, but might result in a reduction of noise.
This is especially problematic due to the necessary between-subject design.
Similar to \cite{moller_single-call_2011}, first usage episodes were presented with highest performance.

%Findings
Temporal effects: Duration, Recency, / Saturation
Two services are judged independent
Episodic duration neglect

Most important quantified!

Modeling -> simple, 2 stages
Sufficient prediction performance!

\section{Discussion}
* Complicated experiment, not useful to consider precise degradations
* between-subject design
* defined-use might affect expectations, and perception

* artificial degradations, unrealistic, but severe and reproducible
* only one "task" / content
* limited set of services, mainly speech-only -> also for other services?
	
* episodic evaluation and me evaluation (repeated) -> effect?

\section{Future Work}
Impact of usage situation, task, and task importance
Service failure: insolveability, delayed fulfillment

Macroscopic performance fluctuations, \ie, lewcio?
Position of degradations inside an episode?

one service used on different devices, usage situations?
Is each device judged separate? or not?

Parallel use of two services?