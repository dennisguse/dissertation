\chapter{Conclusion}\label{chap:discussion}
%Why is ME interesting?
The perceived quality of telecommunication services, especially with regard to repeated use, is an important aspect for service providers, because it might affect the user's satisfaction and also future-use behavior.
The formation process of retrospective judgments of general experiences as well as perceived quality is known to be affected by several biases such as recency effect, peak effect, and duration neglect (\cf, \autoref{chap:03}).
These effects describe that not all parts of an experience are equally reflected in the retrospective judgment of this experience.
With regard to perceived quality such effects have been investigated mainly for macroscopic fluctuations in one stimulus (\cf, \autoref{chap:04}).
For interactive situations, such as a telephone call, methods have been developed to suggests a specific user behavior and thus limit the impact of varying user behavior.
For macroscopic fluctuations of telephony most prominent is here the method of \emph{simulated conversations} \citep{weiss_modeling_2009}.
For the assessment of multi\-/episodic perceived quality \citet{moller_single-call_2011} presented the method of defined\-/use, so the multiple participants are exposed to the same multi\-/episodic condition by defining when, how, and for what a service should be used.
This enables to investigated the formation process of multi\-/episodic perceived quality by deriving a \ac{MOS}.
This complements prior work by \citet{duncanson_average_1969}, which allowed free-use of a service focusing on the average perceived quality for a telephone call.
Here, no knowledge about the prior experiences of a user could be determined, and thus could not determine the reasons for the judgments.

In this thesis I investigated the formation process of multi\-/episodic perceived quality using the method of defined-use.
Here, I pursued two goals with regard to multi\-/episodic perceived quality.
First, I wanted to understand the impact of potential factors affecting the quality formation process.
Second, I wanted to implement a model using episodic judgments for the prediction of multi\-/episodic judgments both in terms of \ac{MOS}.
Two different usage periods have been explored: multiple usage episodes in one session of up to \unit[45]{min} and individual usage episodes distributed over several days.
Those two usage periods were evaluated, because it was so far unknown if the time between episodes affects the formation process of multi\-/episodic perceived quality.
In fact, the investigation for one session allowed to conduct experiments under controlled laboratory terms and thus reduces complexity, effort, and limits environmental influence factors.

For the investigation of multi\-/episodic perceived quality I conducted seven experiments.
For one session, I investigated the number, position, duration, and strength of degraded episodes (\E1{}, \EIIa{}, and \E3{}).
This was complemented by investigating the impact of a second service (\EIIb{}).
\E6{} was designed based upon the one-session experiments and the practical findings of 14\,days experiments (\E4{} and \E5{}).
In this experiment a usage period of 6\,days was investigated.
Extending initial work of \citet{moller_single-call_2011} the defined-use method was applied while performance was only varied between usage episodes.
Macroscopic performance fluctuations were not investigated, because the impact on episodic judgments and multi\-/episodic judgments is not yet fully understood.
For the experiments severe but unrealistic reduction in performance was used, \ie, applying \textsc{\lowercase{LPC\=/10}}.
This enabled to achieve a measurable effect on multi\-/episodic judgments.
In addition, participants were presented typical degradations before the multi\-/episodic assessment and also a \emph{new} service was created.
This is expected to reduce the impact of the required between-subject design, because participants could have prior experiences with the service type in general and typical degradations but not with the specific presented service.
In fact, the presentation of typical degradations provides participants with references for the multi\-/episodic assessment.

%Findings
The results of the conducted experiments showed several effects, indicating characteristics of the formation process of multi\-/episodic perceived quality.
The largest effect on multi\-/episodic judgments is observed for an increasing number of degraded episodes.
The multi\-/episodic judgments decrease until three consecutive degraded episodes/days were presented.
Then, no further decrease is observed although the multi\-/judgments remains well above the episodic judgments of degraded episodes, \ie, saturation is observed.
This effect has been observed in one-session experiments and also in a usage period of 6\,days.
Also, the occurrence of a recency effect, which has been observed in prior work, has been investigated.
Such an effect could be observed one session and was indicated in a usage period of 6\,days.
For one session, a difference was observed.
Here, a recency effect could be observed for one degraded episode only in case of two-party conversation while it was not observed for third-party listening.
This is most probably due to the usage situation, \ie, \emph{passive} consumption versus \emph{active} two-party conversation.
In addition, the impact of presenting degraded episodes consecutively and non-consecutively has been investigated.
However, no clear effect could be observed.
The small, potential difference can also be explained by a recency effect.
For one session also the occurrence of a duration neglect for episodic judgments and multi\-/episodic judgments was investigated.
Even doubling the duration did not yield differences these judgments, showing that the actual duration of a degraded episode is not considered for both judgments.
For one session also a peak effect was investigated.
However, the results were inconclusive as no clear effect could be observed.

The large number of conditions in \E1{}, \EIIa{}, and \E6{} enabled to implement a prediction model (\autoref{chap:modeling}).
It was desired to predict the multi\-/episodic \ac{MOS} based upon prior episodic \ac{MOS}.
The modeling approach of \cite{moller_single-call_2011}, \ie, average of all prior episodic judgments, is extended.
Based upon \cite{moller_single-call_2011}, which applied the average of all prior judgments, the weighted average is applied.
In this thesis I evaluated a window function and a linear weighting function.
The evaluation shows that a both weighting functions outperform achieve a better prediction accuracy compared to the unweighted average of all prior episodic judgments.
With regard to prediction accuracy as well as robustness for parameter selection the linear weighting function performs better than the window function.
For one-session experiments (\E1{}, and \EIIa{}) a $\mathit{w}=2$ and for multiple days (\E6{}) a $\mathit{w}=4$ has been found best suitable.
In addition, the observed saturation was accounted for by adjusting the episodic judgments rather than the weighting function (\autoref{pred:saturation}).
If three consecutive \emph{similar} degraded episodes/days occur (\C6{}), then episodic judgment of the first degraded episode/day are set to the average episodic judgments of all prior non-degraded episodes.
This adjustment reduces the prediction accuracy over all conditions, because this shifts the optimum $\mathit{w}$ of this condition to the overall optimal $\mathit{w}$.
It can be concluded that the multi\-/episodic \ac{MOS} can predicted successfully using the episodic \ac{MOS} with rather a simple model type and weighting functions.

\section{Discussion}
The results of the conducted experiments showed that multi\-/episodic perceived quality can be assessed in one-session as well as over multiple days using the defined-use method.
This investigation was limited on purpose to severe degradations, so potential effects are observable.
Especially, the required between-subject design and the complexity of the experiments make this method unsuitable for the precise investigation of non-severe degradations.
In addition, this method has some inherent disadvantages, which might affect the formation process of multi\-/episodic perceived quality.
First, this method forces participants to use a service in a specific manner.
Here, it is defined when, how, for what a service has to be used.
This might affect the formation process, because participants are not free to use a service to fulfill their own needs, \ie, tasks are not necessarily meaningful and important to them.
Second, the formation process of multi\-/episodic perceived quality might be affected by the assessment of episodic perceived quality.
It is possible that taking episodic judgments affects the memorization process of the experiences.
This was not investigated in the experiments, because the episodic judgments were used for verification and also prediction.
The presence of episodic judgments might increase the ability to remember specific information about an experience, \ie, those that could be recalled for the judgment.
In fact, even the judgment process itself might be remembered.
It is not yet known if this affects the formation process of multi\-/episodic perceived quality.
Third, the application of the defined-use method, if applied for multiple days, requires that participants can embed the episodes into their daily life.
This might be complicated and also frustrating, and thus affect multi\-/episodic judgments.
Finally, the results of the experiments are limited to speech-only telecommunication services as only those were investigated in detail.
It is likely that the observed effects can be generalized to other telecommunication services such as video consumption, Internet-based gaming, and web browsing.
However, differences might be observed due to the very different usage situations, expectations, and also types of degradations.

The conducted experiments allowed to successfully investigate the formation process of multi\-/episodic perceived quality.
It could be shown that the formation process of multi\-/episodic perceived quality is affected by several effects resulting in differences of multi\-/episodic judgments.
Here, similar effects could be observed that are known to affect retrospective judgments, \ie, recency effect and duration neglect.
In addition, a saturation effect was observed for the two studied usage period.
This effects has so far not been observed for retrospective judgments of perceived quality.
In fact, the underlying reason(s) for the observed effects cannot be deduced from the experiments.
%Although the model was created only based upon the conducted experiments, it can be used as basis for further investigations.

\section{Future Work}
Although the experiments showed consistent results, the findings are necessarily limited to the evaluated settings.
The here presented findings form a useful basis for further investigation of multi\-/episodic perceived quality.
It seems important if the observed effects can also be observed for other types of telecommunication services, or are specific to mainly investigated speech-based service types.
In fact, the results show that the usage situation seems to affect the multi\-/episodic judgments.
Further investigation are also necessary to evaluate the impact of applied tasks as well as their importance to participants.
The performance of an \emph{important} episode with have a have a higher multi\-/episodic perceived quality than less important episodes.
Another so far not investigated aspect is the inability to fulfill a task due to reduced performance.
The resulting frustration might result in a higher effect than a successful episode.
A conceptual approach towards integrating inability to fulfill a task into \ac{QoE} is presented by \citet{leon-garcia_generalizing_2014}.
This framework might serve as a starting point for multi\-/episodic perceived quality.
Also the impact of macroscopic performance fluctuations on multi\-/episodic judgments has not been investigated so far.
In addition, it is also not known how the multi\-/episodic judgment of a service that is used on multiple, different devices, \eg, a mobile device and a stationary device, is formed.
It is not (yet) known if the multi\-/episodic perceived quality is integrated by device or in such a case rather by service, and if the episodes are weighted equally.
The defined-use method allowed deriving a \ac{MOS}.
However, this necessarily ignores individual differences between participants, \ie, assume a similar formation process with similar characteristics.
In fact, the formation process might also be affect by characteristics of subgroups or even individual participants.
Thus, the knowledge about the \ac{MOS} and its prediction should be complemented by investigating for individual differences.
In addition, the findings should be verified with \emph{real} users that use a service on their own, \ie, free-use, because this allow to verify the findings under ecological valid settings.

The derived knowledge on the formation process of multi\-/episodic perceived quality can be used as input for models on service quality such as \citet{parasuraman_conceptual_1985}.
In fact, knowledge about the business impact of performance fluctuations for repeated use is highly desired aspect for service providers.