\chapter{Conclusion}\label{chap:discussion}
In this thesis, I investigated the formation process of perceived quality in case of multi-episodic use.
The formation of perceived quality for repeated use with meaningful, distinct interactions, \ie, usage episodes, was investigated for telecommunication services.
This complements prior work on perceived quality, which mainly focused on the precise evaluation of the technical conditions and omit repeated use as well as time as influencing factor.
Two different usage periods have been explored: multiple usage episodes in session and single usage episodes distributed over several days.
One session was here used to establish a basis for the investigation of multiple days.
This allowed to conduct experiments under controlled laboratory terms.
This reduces complexity, effort, and thus allows to investigate a more conditions.

Beside the implementation of a prediction model based upon episodic judgments, I evaluated factors affecting the formation process of multi-episodic perceived quality.
I conducted six experiments to achieve those two goals.
Two experiments (\E4{} and \E5{}) focused mainly on feasibility of conducting a field experiment.
In one session, I investigated the number, position, duration, and strength of degraded episodes (\E1{}, \EIIa{}, and \E3{}).
This was complemented by investigating the impact of a second service (\EIIb{}).
Based upon those experiments, \E6{} was designed to investigate a usage period of 7~days.

Those experiments were conducted based upon the defined-use method to extend the initial work by \citet{moller_single-call_2011}.
This method limits the behavioral freedom of participants, but allows to expose multiple participants to the same condition.
Similar to \cite{moller_single-call_2011}, only constant performance was applied for each usage episode.
This avoids potential and not yet fully understood impact of macroscopic performance fluctuations on perceived quality, \ie, episodic judgments and multi-episodic judgments.
For the experiments I  selected severe, but unrealistic reduction in performance.
This allows to achieve a measurable effect on multi-episodic judgments, which is necessary to investigate potential effects.
In addition, \emph{new} services were created for the experiments.
This should avoid the impact of prior experiences with an established, widely used service on episodic judgments and multi-episodic judgments.
Beside use of severe degradations and new services, in later experiments typical degradations were presented to participants before the multi-episodic assessment.
This provides participants with references, which they can use in the multi-episodic assessment for their judgments.
Although, this might set expectations on to be experienced degradations and thus affect multi-episodic perceived quality, it is expected to reduce noise.
This should overcome one limitation of the required between-subject design, \ie, participants have different prior knowledge and might use the scales different.
%Furthermore, a participant might adapt his interpretation of a scale over the usage period due to the experience and thus might different judgments might become incomp
%This is especially problematic due to the necessary between-subject design.
%Following \citet{moller_single-call_2011}, first usage episodes were presented with highest performance.

%Findings
The results of the conducted experiment showed several effects, indicating characteristics of the formation process of multi-episodic perceived quality.
Most notably is the observed saturation, if the number of degraded episodes is increased.
In this case, the multi-episodic judgments decrease until three consecutive degraded episodes/days are presented.
Then, no further decrease is observed, although the judgment remains well above the episodic judgments of degraded episodes.
This has been observed in one-session experiments and also in a usage period of 7~days.
Also the occurrence of a recency effect has been investigated.
Such an effect could be observed in a usage period of 7~days and also one session.
In the later case, an effect diminished for third-party listening and one degraded episode.
This is most probably due to the usage situation, \ie, passive consumption versus active two-party conversation.
In addition, also the presentation of degraded episodes consecutively and non-consecutively has been investigated.
However, no clear effect could be observed in the two considered usage periods.
The small, potential difference can also be explained by a recency effect and thus the number of performance changes seems not to affect multi-episodic judgments.
Beside this also duration neglect of degraded episode was investigated.
Even doubling the duration did not yield differences in episodic judgments and also multi-episodic judgments.
The duration can thus be neglected in the investigated case.
Also, the existence of a peak-effect was investigated in one session.
However, the results were inconclusive as no definite effect could be observed.
One reason for this might be the applied between subject design.

Beside the investigation of potential effects, the large number of conditions are useful to implement a prediction model.
In this thesis, I developed a two-stage prediction model that sufficiently estimates multi-episodic judgments.
In the first stage the episodic judgments are adjusted to account for the observed saturation effect.
If three consecutive \emph{similar} degraded episodes/day occur, then episodic judgments of the first episode/day are set to the average judgments of all prior non-degraded episodes.
This allows to account for the saturation effect without complex manipulation of the weighting function for this specific case.
In the second stage, a weighted average is computed either with a windowed function, or a linear window function.
Both models provide sufficient prediction performance considering the limited number of conditions.
In fact, both weighting functions limit the potential overfitting of more complex models.

\emph{TODO: Modeling: which one is better?}

\section{Discussion}
The results of the conducted experiments showed that multi-episodic perceived quality can be assessed in one-session and over multiple days while using the defined-use method.
This enabled to present multiple participants the same condition and thus enable a \ac{MOS} evaluation.
Applying this method allowed to investigate effects, which affect the formation process of multi-episodic perceived quality.
This investigation should be limited to severe degradations, so potential effects are observable.
Especially the required between-subject design and complexity of experiments make this method unsuited for the precise investigation of realistic degradations.
due to the required between-subject design and complexity of experiments.
However, this method has some inherent disadvantages that might affect might multi-episodic judgments.
This method forces participants to use a service in a specific manner by defining when and how to use a service.
This might affect the formation process, because participants are not free to use a service to fulfill their own needs, \ie, the tasks is not necessarily meaningful and important to them.
In addition, the defined-use method requires that participant can embedded the usage episodes into their daily life.
This might be complicated, and also frustrating and thus affect multi-episodic judgments.
In addition, the formation process of multi-episodic perceived quality might be affected by the assessment of episodic perceived quality.
It is possible that taking such a judgment, affects the memorization process of for an experience.
This was not investigated in the experiments as the episodic judgments were used for verification and also prediction.
The presence of episodic judgments might increase the ability to remember specific information about an experience, \ie, those that could be recalled for the judgment.
In fact, even the judgment can be remembered.
If this affects formation process of multi-episodic perceived quality is not yet known.
In addition, the results of the experiments are limited to speech-only telecommunication services, because only those were investigated in detail.
It is likely that the observed effects can be generalized to different telecommunication services like video consumption, Internet-based gaming, and web browsing, but differences might occur.

Nevertheless, the conducted experiments allowed to investigate the formation process of multi-episodic perceived quality.
It could be shown that the formation process of multi-episodic perceived quality is affected by several effects resulting in differences of multi-episodic judgments.
Here, similar effects could be observed that are known to affect retrospective judgments, \ie, recency effect and duration neglect.
The observed saturation effect has so far not been observed.
The underlying reason for the observed effects is far unknown and must be left for future work.
%Although, the model was created only based upon the conducted experiments, it can be used as basis for further investigations.

\section{Future Work}
The presented results are necessarily limited to the evaluated setting.
However, the findings are useful as a basis for further investigations of multi-episodic perceived quality.
It is necessary to investigate, if the observed effects occur also for other types of telecommunication services, or if (some) effects are specific to speech-based telecommunication services.
In fact, a difference on multi-episodic perceived quality has been observed in the one-session experiments between two-party conversation and third-party listening.
This indicates that the formation process is affected by the usage situation and also the service type.
In addition, the impact of a to be solved task should by investigated.
Especially, the importance for a user of a task is likely affecting episodic perceived quality.
The performance of a usage episode with an important task might have a higher multi-episodic perceived quality than less important tasks.
Beside importance, also the ability to successfully solve a task might affect the formation process.
In a usage episode in which the reduced performance prevents task fulfillment, might have a higher effect than an episode where the task could be fulfilled.
A conceptual approach towards integrating inability to fulfill a task into \ac{QoE} is presented by \citet{leon-garcia_generalizing_2014}.
This framework might serve as a starting point for multi-episodic perceived quality and service unavailability.

Beside those service type and task related aspects, also the impact of macroscopic performance fluctuations is left unsolved here.
An impact of macroscopic fluctuations on retrospective judgments of (sub-)episode could already be shown (\cf, \autoref{chap:04}).
It remains to investigate, if such macroscopic fluctuations affected multi-episodic judgments.

It should also be investigated, how a service is judged that is available on different devices.
For example a video telephony system like Skype is available on computers and also mobile devices.
If a user now interacts with this service repeatedly, he might not always use the same devices.
It is not clear, if the multi-episodic perceived quality is integrated by device, or in this case rather by service.