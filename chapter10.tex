\chapter{Conclusion}\label{chap:discussion}
In this thesis I investigated the formation process of perceived quality in case of multi\-/episodic use.
Here, the so-called multi\-/episodic perceived quality was investigated for speech and video telecommunication services.
This complements prior work on perceived quality, which mainly focused on the precise evaluation of the technical conditions while omitting repeated use.
%add related work here? what are their findings? kahneman + qoe
%In this thesis
Two different usage periods have been explored: multiple usage episodes in one session of up to \unit[45]{min} and individual usage episodes distributed over several days.
One session was used here to establish a basis for the investigation of multiple days.
This allowed to conduct experiments under controlled laboratory terms.
It reduces complexity, effort, and limits environmental influence factors.
It thus enabled to investigate a higher number of conditions.

In this thesis I pursued two goals with regard to multi\-/episodic perceived quality.
First, I wanted to understand the impact of potential factors affecting the quality formation process.
Second, I wanted to implement a model using episodic judgments for the prediction of multi\-/episodic judgments.
To achieve these two goals, I conducted seven experiments.
For one session, I investigated the number, position, duration, and strength of degraded episodes (\E1{}, \EIIa{}, and \E3{}).
This was complemented by investigating the impact of a second service (\EIIb{}).
Based upon those experiments and the practical findings of \E4{} and \E5{}, \E6{} was designed.
In this experiment the formation process of multi\-/episodic perceived quality was investigated in a usage period of 6~days.

Those experiments were conducted based upon the defined-use method, extending initial work by \citet{moller_single-call_2011}.
This method limits the behavioral freedom of participants, but allows to present the same condition to multiple participants.
Thus, the defined-use method allows to conduct a \ac{MOS} evaluation.

In this thesis the impact of constant episodic performance was investigated, \ie, performance was only varied between usage episodes.
This avoids potentially, unknown effects due to macroscopic performance fluctuations on episodic judgments and multi\-/episodic judgments.
For the experiments, I selected a severe but unrealistic reduction in performance, \ie, applying \textsc{\lowercase{LPC\=/10}}.
This enabled to achieve a measurable effect on multi\-/episodic judgments, which is necessary to investigate potential effects.
Beside the use of severe degradations, in later experiments typical degradations were presented to participants before the multi\-/episodic assessment.
This provides participants with references, which can be used for episodic judgments and multi\-/episodic judgments.
Although this might set expectations for to be experienced degradations and thus affect multi\-/episodic perceived quality, it provides a common reference for all participants.
This is expected to reduce the impact of the required between-subject design.
For the same reason \emph{new} services were created for the conducted experiments.
Thus, participants cannot have prior experiences with the service under assessment.
%Furthermore, a participant might adapt his interpretation of a scale over the usage period due to the experience and thus might different judgments might become incomp
%This is especially problematic due to the necessary between-subject design.
%Following \citet{moller_single-call_2011}, first usage episodes were presented with highest performance.

%Findings
The results of the conducted experiments showed several effects, indicating characteristics of the formation process of multi\-/episodic perceived quality.
The largest effect on multi\-/episodic judgments is observed for an increasing number of degraded episodes.
Here, judgments decrease until saturation is reached.
The multi\-/episodic judgments decrease until three consecutive degraded episodes/days are presented.
Here, no further decrease is observed, although the judgment remains well above the episodic judgments of degraded episodes.
This has been observed in one-session experiments and also in a usage period of 6~days.
Also, the occurrence of a recency effect has been investigated.
Such an effect could be observed one session and was indicated in a usage period of 6~days.
For one session, a difference was observed.
Here, a recency effect could be observed for one degraded episode only in case of two-party conversation while it was not observed for third-party listening.
This is most probably due to the usage situation, \ie, \emph{passive} consumption versus \emph{active} two-party conversation.
In addition, the impact of presenting degraded episodes consecutively and non-consecutively has been investigated.
However, no clear effect could be observed.
The small, potential difference can also be explained by a recency effect.
For one session, also the occurrence of a duration neglect was investigated.
Even doubling the duration did not yield differences in episodic judgments and also multi\-/episodic judgments.
For one session, also a peak effect was investigated.
However, the results were inconclusive as no clear effect could be observed.

Beside the investigation of potential effects, the large number of conditions enabled to implement a prediction model.
In this thesis I applied a weighted average to implement a prediction model.
Here, a window function and a linear weighting were proposed.
The results show that a linear weighting function outperforms the window function.
In fact, it provides a higher robustness for parameter selecting.
For one-session experiments (\E1{}, and \EIIa{}), $\mathit{w}=2$ and for multiple days (\E6{}) a $\mathit{w}=4$ has been found best suitable.
In addition, the observed saturation was accounted for by adjusting the episodic judgments rather than the weighting function.
If three consecutive \emph{similar} degraded episodes/days occur, then episodic judgment of the first degraded episode/day are set to the average judgments of all prior non-degraded episodes.
It could be shown that for these three experiments, the prediction accuracy is increased.
Here, it is observed that for this condition the model parameter is adjusted to the optimum $\mathit{w}$ already found.
It can be concluded that the modeling approach is successful while using rather a simple model type.


\section{Discussion}
The results of the conducted experiments showed that multi\-/episodic perceived quality can be assessed in one-session and over multiple days using the defined-use method.
This allowed to investigate potential effects, which affect the formation process of multi\-/episodic perceived quality.
This investigation was limited to severe degradations, so potential effects are observable.
Especially, the required between-subject design and complexity of experiments make this method unsuitable for the precise investigation of realistic degradations.
However, this method has some inherent disadvantages that might affect multi\-/episodic judgments.
The method forces participants to use a service in a specific manner.
Here, it is defined when, how, for what a service has to be used.
This might affect the formation process, because participants are not free to use a service to fulfill their own needs, \ie, tasks are not necessarily meaningful and important to them.
In addition, the defined-use method requires that participants can embed the usage episodes into their daily life.
This might be complicated, and also frustrating, and thus affect multi\-/episodic judgments.
In addition, the formation process of multi\-/episodic perceived quality might be affected by the assessment of episodic perceived quality.
It is possible that taking episodic judgments affects the memorization process of the experiences.
This was not investigated in the experiments, as the episodic judgments were used for verification and also prediction.
The presence of episodic judgments might increase the ability to remember specific information about an experience, \ie, those that could be recalled for the judgment.
In fact, even the judgment process itself might be remembered.
It is not yet known if this affects the formation process of multi\-/episodic perceived quality.
In addition, the results of the experiments are limited to speech-only telecommunication services as only those were investigated in detail.
It is likely that the observed effects can be generalized to other telecommunication services such as video consumption, Internet-based gaming, and web browsing.
However, differences might occur especially due to the very different usage situations.

Nevertheless, the conducted experiments allowed to investigate the formation process of multi\-/episodic perceived quality.
It could be shown that the formation process of multi\-/episodic perceived quality is affected by several effects resulting in differences of multi\-/episodic judgments.
Here, similar effects could be observed that are known to affect retrospective judgments, \ie, recency effect and duration neglect.
The found saturation effect has so far not been observed.
However, the underlying reason(s) for the observed effects are so far unknown and must be left for future work.
%Although the model was created only based upon the conducted experiments, it can be used as basis for further investigations.

\section{Future Work}
The presented results are necessarily limited to the evaluated setting.
However, the findings are useful as a basis for further investigations on multi\-/episodic perceived quality.
It is necessary to investigate, if the observed effects occur also for other types of telecommunication services, or if (some) effects are specific to speech-based telecommunication services.
In fact, a difference on multi\-/episodic perceived quality has been observed in the one-session experiments between two-party conversation and third-party listening.
This indicates that the formation process is affected by the usage situation.
In addition, the impact of tasks needs to be investigated.
Especially, the importance of tasks is likely to affect episodic perceived quality.
The performance of a usage episode with an important task might have a higher multi\-/episodic perceived quality than less important tasks.
Beside importance, also the ability to successfully solve a task might affect the formation process.
An episode in which the reduced performance prevents task fulfillment, might have a higher effect, than a successful episode.
A conceptual approach towards integrating inability to fulfill a task into \ac{QoE} is presented by \citet{leon-garcia_generalizing_2014}.
This framework might serve as a starting point for multi\-/episodic perceived quality.

Beside service types and task related aspects, also the impact of macroscopic performance fluctuations is not yet solved.
An impact of macroscopic fluctuations on retrospective judgments of (sub-)episode could already be shown (see~\autoref{chap:04}).
It remains to investigate if such macroscopic fluctuations affected multi\-/episodic judgments.

It should also be investigated, how a service is judged that is available on different devices.
For example a video telephony system is available on computers and also mobile devices.
If a user now interacts with this service repeatedly, he might not always use the same device.
It is not (yet) known, if the multi\-/episodic perceived quality is integrated by device, or in this case rather by service.