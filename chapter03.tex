\section{Retrospective Judgments of Experiences}\label{chap:03}
For a retrospective judgment of an experience a person can only rely on available information to him.
While experiencing, a person needs to encode and memorize information, so he is later able to recall this information for his judgment.
Already during perception the amount of information is reduced, encoding only a reduced set into the perceptional memory and afterwards into working memory \citep[\cf,][p. 8f.]{raake_speech_2006}.
While those types of memory only have very limited storage time, in the range of seconds up to minutes, the information that becomes encoded into the long-term memory may remain accessible for several years.
Memorized information decays, reducing the amount of \emph{original} information even further.
In addition, recall from long-term memory is not a perfect process.
On recall the available information is complemented by additional available sources like prior knowledge or even previous but not related experiences \citep[\cf,][]{schacter_seven_2003}.
%In fact, specific information might not be accessible at all at the time of recall.

With regard to experiences and their retrospective judgments two aspects are important.
First, how are experiences stored and in which manner is the underlying information grouped together.
Based upon the concept of episodic memory, this leads to the definition of the \emph{episode} that forms the basis for my work on multi-episodic \ac{QoE}.
Second, know and understand effects that determine retrospective judgments of an episodic experience.
Both aspects are presented in the following.

\subsection{Episodic Memory}
Personal experiences and related information are stored in the \emph{episodic memory} \citep{tulving_episodic_1972}.
This memory stores information and their temporal-spatial relationship \citep[][p.~385]{tulving_episodic_1972}.
The information is grouped together by specific events, or so-called episodes.
Information is encoded based upon their autobiographical reference to already stored content \citep[][p.~385f.]{tulving_episodic_1972}.
In addition, temporal and spatial information is stored including information about the situation as well as feelings \citep[][p.~385f.]{tulving_episodic_1972}.
An episode has an explicit start and end while retaining a temporal order of information \citep[][p.~262]{conway_construction_2000}.
Based upon recallable information about an episode, this episode can be re-experienced enabling \emph{mental} time-travel, which is denoted as memory-vividness \citep{conway_construction_2000}.
An episode is successful memorized and retrieved, if the perceptual properties and their approximate temporal relationship to other episodes can be described.
The retrieval process can be supported by providing a cue.

The episodic segmentation process, \ie, constitution in the memory of an actor, is expected to happen during the actual experience \citep[\cf,][]{ezzyat_what_2011, kurby_segmentation_2008}.
Here especially the goal of an actor and temporal closeness of individual events are considered of special importance \citep[\cf,][]{black_episodes_1979}.
It must be noted that first-time episodes and repeated episodes are considered different (\citet{conway_construction_2000} referencing \citet{barsalou_content_1988}).
The latter are linked by a shared theme \citep{robinson_first_1992}, but tend to retain less distinct information about individual episodes.

\subsection{Effects on Retrospective Judgments}
For a \emph{retrospective judgment} a person must rely on information about this experience that has been \emph{a)} perceived, \emph{b)} memorized and \emph{c)} can been recalled while conducting this judgment.
With regard to retrospective judgments major work has been done for the judgment of pain, showing several, even counterintuitive, effects.
It has been found that not all parts of an experience affect a retrospective judgment equally.
Those effects are described in the following.

\paragraph*{Primacy and Recency}
Primacy and recency have first been observed for sequential learning \citep[\cf,][]{murdock_jr._serial_1962}.
In a sequential learning task, it has been found that the likelihood to recall specific items afterwards depends on the position.
Items at the very beginning and at the very end have an increased likelihood to be recalled correctly.
This is denoted as \emph{primacy effect} and \emph{recency effect}.
The former denotes an increased likelihood to recall the beginning and the latter the increased likelihood to recall final items.
With regard to judgments of experiences a recency effect is often observed.
In an episode with varying pain the retrospective judgment is higher, \ie, the episode is judged less painful, if less pain occurs at the end \citep[\cf,][]{kahneman_when_1993, redelmeier_patients_1996}.
Such an effect could also be observed, if an episode is extended by a less painful ending.
An effect of primacy has not been observed for retrospective evaluation of pain, indicating that the beginning of such an experience has a reduced importance on the retrospective judgment.

\paragraph*{Peak}
For retrospective judgments of an experience it has been observed that an outstanding part is overly represented in the retrospective judgment.
This is denoted as \emph{peak} effect.
Such an effect has been observed in retrospective assessment of pain \citep[\cf,][]{kahneman_when_1993, redelmeier_patients_1996}.
Here a spike in \emph{momentary pain} severely results in a reduced retrospective judgment.
It seems as if exceptional parts of an experience can either better memorized, or are likelier to recall.
Peak effects are most often considered for \emph{negative} peaks, \eg, outstanding pain, but rarely for positive peaks, \eg, outstanding pleasure.

\paragraph*{Duration Neglect}
In addition to recency and peak effects also neglect of duration has been observed on retrospective judgments.
Here the actual duration of an experience is neglected rather than considered.
This has been observed for retrospective judgments of pain \citep[\cf,][]{fredrickson_duration_1993, ariely_combining_1998}.
%redelmeier_memories_2003

\paragraph*{Conclusion}
It must be concluded that retrospective judgments of experiences are result of an integration process in which individual parts are assigned a special importance.
%Similar observations have also been made for \citet{hogarth_order_1992}
