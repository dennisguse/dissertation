\chapter{Temporal Aspects of Experiences, Memorization and Recall}\label{chap:03}
\begin{chapter-abstract}
%NOTE Could be merged with chap 2+4?
Here I present how recall is influenced by human memory (and thus retrospective judgments).
I give an multi-disciplinary overview on this topic starting from episodic memory (and thus skipping all short-term/working memory influences).
I focus on temporal effects on learning (as a baseline) and also temporal effects on the recall of experiences (like pain, happiness etc.).
Most important point here is that retrospective judgments always relay on \textit{recalled information}, which must a) not be complete and b) not correct at all.
I will also discuss factors affecting memorization (e.g. attention, importance) and memory failures \citep{schacter_seven_2002}.

I must very carefully explain here the different aspects of time for perception, memorization and retrieval.
This should include perception (and underlying adaptation of perceiving organs) at and also time perception \cite{wittmann_neuropsychologie_2009}.

%One question is: does QoE evolve continuously or on judgment. What does this mean for the judgment process.

Because experiences are grouped and memorized together into an \emph{episode}, the most important aspect of this chapter is the \emph{episodic memory}.
A conceptual definition of \emph{episode} (based on memorization with regard to retrospective assessment) is hinted that forms the basis for the definition of \emph{usage episode} in \autoref{chap:04}.
\end{chapter-abstract}
%TODO Add memory types: working, short-term, long-term / episodic 
%TODO Availability bias
%TODO Importance of content [Baumgartner]: Ads and positive peak effect on remembering content

A person conducting a judgment about an experience can only use those information available to him.
For a \emph{momentary judgment} this person can use his current experiences as perceives it now.
For a \emph{retrospective judgment} this person must rely on information about this experience that has been \emph{a)} perceived, \emph{b)} memorized and \emph{c)} can been recalled while conducting this judgment.
Even if a certain property of the object under consideration has been perceived and encoded properly, this specific information might not be memorized at all, or might change while memorizing as well as while recalling.

%Integration process?
%Memory

\section{Temporal Effects}
%Duration Neglect, Adaption Speed (from low to high vs. high to low), Peak,

For the retrospective assessment of experiences in general several temporal effects have been found.
Major work has been done in the field of learning and assessment of pain.
In the following I present temporal effects that have been to influence retrospective judgments.

\subsubsection*{Primacy and Recency}
Primacy and recency have first been found for sequential learning \citep[cf.,][]{murdock_jr._serial_1962}.
In a sequential learning task, it has been found that the likelihood to recall specific items afterwards depends on the position.
Items at the very beginning and at the very end have an increased likelihood to be recalled correctly.
This is denoted as \emph{primacy effect} and \emph{recency effect} whereas the former denotes the increased likelihood to recall the beginning and the latter the increased likelihood to recall the end.

With regard to the retrospective evaluation of an episode of pain an effect of recency is observed, leading to a better judgments, if less pain occurs at the end \citep[\cf,][]{kahneman_when_1993, redelmeier_patients_1996}.

An effect of primacy has also been for belief-adjustment showing that the outcome depends on the order of received information\todo{REF}.
New information is interpreted while considering the current belief and might thus be interpreted different depending on the order.

Recency has also been observed in perceived quality (\cf, \autoref{chap:04}).

\subsection*{Trend}

TODO Following Kahneman

\subsection*{Peak}
For retrospective judgments of an experience it has been found that the most outstanding part has an increased impact on this judgment.
This is denoted as a \emph{peak effect}.
Such an effect has been observed in retrospective assessment of pain \citep[\cf,][]{kahneman_when_1993, redelmeier_patients_1996}.
Here a spike in momentary pain measurements severely reduces the retrospective assessment.
Peak effects are also been observed in retrospective evaluation of perceived quality (\cf, \autoref{chap:04}).

Peak effects are most often considered for \emph{negative} experiences, \eg, outstanding pain, but not for positive experiences.

\subsection*{Duration Neglect}
For retrospective judgments it is often observed that the duration of an experience is neglected.
Such an effect has been observed in retrospective assessment of pain \citep[\cf,][]{fredrickson_duration_1993, ariely_combining_1998}.

\section{Episodic memory}
Usage episode and episodic memory [Tulving, Black+Bower, Ezzyat]
\begin{itemize}
\item Colloquial:  http://www.merriam-webster.com/dictionary/episode (an event that is distinctive and separate although part of it)
\item remembering vs. knowledge \cite{tulving_concepts_2000}

\item Episodic memory "temporally dated episodes or events and temporal-spatial relations" \cite[385]{tulving_concepts_2000}
\item "A perceptual event can be stored in the episodic system solely in terms of its perceptible properties"\cite[p. 385]{tulving_concepts_2000}
\item Experienced events have a temporal order (Retrieval should also reveal this order!)\cite{tulving_concepts_2000}

\item autobiographical reference, personal identity [Tulving p.10], \cite{conway_construction_????}
\item What? When? Where? Situation+Feelings [Tulving p.10]
\item Explicit start and end [Conway]
\item Time-travel (I must be able to travel back to that situation); memory-vividness \cite{conway_construction_????}

      
\item Successful retrieval requires successful encoding! (successful retrieval: a person can describe perceptual properties AND temporal relations to other events)

\item Events are part of an episode. -> Providing a cue for one episodes allows to retrieve information this episode alone!
\item What is a moment? (Kahneman said a moment is usually 3 seconds?)
\end{itemize}

\section{Event Segmentation Theory}
\cite{black_episodes_1979}\cite{ezzyat_what_2011}\cite{zacks_perceiving_2001}
\begin{itemize}
\item by goal \cite{black_episodes_1979}
\item temporal closeness \cite{black_episodes_1979}\cite{ezzyat_what_2011}
\item "that segments experience (events) into episodes" \cite[p. 248]{ezzyat_what_2011}
\item event segmentation happens while experiencing \cite[p. 248]{ezzyat_what_2011}

\item [Conway] following [Barsalou 1988] \textit{Event-specific knowledge} (ESK)
\item -> general "events": repeated events (e.g. evening hikes) and single events (trip to Paris)
\item -> Series of memories linked to together by a theme
\item -> Goal-attainment knowledge
\item First encounter (better encoded->more important AND seems to set expectations)
\item Repeated encounter
\end{itemize}

%\section{Memory failures}
%Schacter, but not that important as he discusses mainly semantic failures.
%The failures are not about opinions and their recall.

%\begin{itemize}
%\item Failure to encode
%\item Failure to recall
%\item Mis-attribution
%\end{itemize}




