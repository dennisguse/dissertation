\chapter{Temporal aspects for Recall}\label{chap:03}
\section*{Abstract}
<<<<<<< HEAD
Here I present how recall is influenced by human memory (and thus also retrospective judgments).
I give an multi-disciplinary overview on this topic starting from episodic memory (and thus skipping all short-term/working memory influences).
I focus on temporal effects on learning (as a baseline) and also temporal effects on the recall of experiences (like pain, happiness etc.).
Most important point here is that retrospective judgments always relay on \textit{recalled information}, which must a) not be complete and b) not correct at all.
I will also discuss factors affecting memorization (e.g. attention, importance) and memory failures (Schacter).
=======
Here I present all known effects on retrospective judgments of general experiences (like pain, happiness etc.).
This includes also the episodic memory and memory failures.
A definition of \emph{episode} (based on memorization with regard to retrospective assessment) is derived.
%Extended concepts are explained like utility and also importance for Usability and adaptation of behavior.?

I must very carefully explain here the different aspects of time.
This should include perception (and underlying adaptation of perceiving organs) at and also time perception (Wittmann?)

\section{Temporal Effects}
\begin{itemize}
\item Recency
\item Peak
\item Primacy
\item Peak-end
\end{itemize}
>>>>>>> dec5e451247ff114342068ab2cfcba2c6e0e1425

Because experiences are grouped and memorized together into an \textit{episode}, the most important aspect of this chapter is the \textit{episodic memory}.
A conceptual definition of \emph{episode} (based on memorization with regard to retrospective assessment) is hinted that forms the basis for the definition of \textit{usage episode} in Chapter~\ref{chap:04}.

%\bibliography{Bibliography}

\section{Episodic memory}
Usage episode and episodic memory [Tulving, Black+Bower, Ezzyat]
\begin{itemize}
\item Colloquial:  http://www.merriam-webster.com/dictionary/episode (an event that is distinctive and separate although part of it)
\item remembering vs. knowledge \cite{tulving_concepts_2000}

\item Episodic memory "temporally dated episodes or events and temporal-spatial relations" \cite[385]{tulving_concepts_2000}
\item "A perceptual event can be stored in the episodic system solely in terms of its perceptible properties"\cite[385]{tulving_concepts_2000}
\item Experienced events have a temporal order (Retrieval should also reveal this order!)\cite{tulving_concepts_2000}

\item autobiographical reference, personal identity [Tulving p.10], [Conway]
\item -> What? When? Where? Situation+Feelings [Tulving p.10]
\item Explicit start and end [Conway]
\item Time-travel (I must be able to travel back to that situation); memory-vividness [Conway]
      
\item Successful retrieval requires successful encoding! (successful retrieval: a person can describe perceptual properties AND temporal relations to other events)

\item Events are part of an episode. -> Providing a cue for one episodes allows to retrieve information this episode alone!
\item What is a moment? (Kahneman said a moment is usually 3 seconds?)
\end{itemize}

\section{Event Segmentation Theory}
\cite{black_episodes_1979}\cite{ezzyat_what_2011}\cite{zacks_perceiving_2001}
\begin{itemize}
\item by goal \cite{black_episodes_1979}
\item temporal closeness \cite{black_episodes_1979}\cite{ezzyat_what_2011}
\item "that segments experience (events) into episodes" \cite[248]{ezzyat_what_2011}
\item event segmentation happens while experiencing \cite[248]{ezzyat_what_2011}

\item [Conway] following [Barsalou 1988] \textit{Event-specific knowledge} (ESK)
\item -> general "events": repeated events (e.g. evening hikes) and single events (trip to Paris)
\item -> Series of memories linked to together by a theme
\item -> Goal-attainment knowledge
\item First encounter (better encoded->more important AND seems to set expectations)
\item Repeated encounter
\end{itemize}

\section{Utility / Happyiness / Well-being}
\begin{itemize}
\item Utility [Kahnemann]
\item Kahneman: objective happiness
\end{itemize}

\section{Learning}
\begin{itemize}
\item sequential learning
\end{itemize}

\section{Learning}
TODO: I need to do a broad literature review for this topic.
So far I have just a general overview on the temporal effects.


\section{Memory failures}
Schacter, but not that important as he discusses mainly semantic failures.
The failures are not about opinions and their recall.

\begin{itemize}
\item Failure to encode
\item Failure to recall
\item Mis-attribution
\end{itemize}


\section{Belief-Adjustment model}
Describe Hogarth's model and describe found effects!

Importance of content [Baumgartner]: Ads and positive peak effect on remembering content
