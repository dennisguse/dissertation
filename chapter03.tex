\chapter{Prior work on temporal effects}

\section*{Summary}
Here I present all known effects on retrospective judgments of general experiences (like pain, happiness etc.).
This includes also the episodic memory and memory failures.
A definition of \emph{episode} (based on memorization with regard to retrospective assessment) is derived.

\begin{itemize}
    Episode [Tulving, Black+Bower, Ezzyat]:
    - Colloquial:  http://www.merriam-webster.com/dictionary/episode (an event that is distinctive and separate although part of a larger series)
    - remembering vs. knowledge [Tulving]

    - Episodic memory "temporally dated episodes or events and temporal-spatial relations" [Tulving p.385]
    - "A perceptual event can be stored in the episodic system solely in terms of its perceptible properties [Tulving p.385]
    - Experienced events have a temporal order (Retrieval should also reveal this order!) [Tulving]

    - autobiographical reference, personal identiy [Tulving p.10], [Conway]
    -> What? When? Where? Situation+Feelings [Tulving p.10]
    - Explicit start and end [Conway]
    - Time-travel (I must be able to travel back to that situation); memory-vivedness [Conway]
      
    - Successful retrieval requires succesful encoding! (successful retrieval: a person can describe perceptual properties AND temporal relations to other events)

    - Events are part of an episode.
    -> Providing a cue for one episodes allows to retrieve information this episode alone!
    
    Event segmentation theory (EST) [Black+Bower, Ezzyat, Zacks]
    - by goal [Black+Bower]
    - temporal closeness [Black+Bower, Ezzyat p.248]
    - "that segments experience (events) into episodes" [Ezzyat p.248]
    - event segmentation happens while experiencing [Ezzyat p.248]

    - [Conway] following [Barsalou 1988] Event-specific knowledge (ESK)
    -> general "events": repeated events (e.g. evening hikes) and single events (trip to Paris)
    -> Series of memories linked to together by a theme
    -> Goal-attainment knowledege
    --- First encounter (better encoded->more important AND seems to set expectations)
    --- Repeated encounter
    
    - Memory failures [Schacter?]
    -> Failure to recall due encoding
    -> Failure to recall due to retrieve
    -> Mis-attribution

  Utility [Kahnemann]
  Happiness [Kahnemann]
  Learning

  Importance of content [Baumgartner]: Ads and positive peak effect on remembering content

  Definition Effects:
  - Recency
  - Duration neglect [Kahnemann, Frederickson]
  - Peak
  - Peak-end effect
  
  Belief-Adjustment Model [Hogarth]

	Memory failures [Schacter]

	
	(optional) Adjustment-level Theory; NOPE

\end{itemize}
