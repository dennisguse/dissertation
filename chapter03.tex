\section{Retrospective Judgments of Experiences}\label{chap:03}
For a retrospective judgment of an experience, one person can only rely on information available to him.
While experiencing, a person needs to encode and memorize information in order to recall this information for a later judgment.
As early as the perception stage, the amount of information is reduced, because only a subset of information can be encoded successfully into the perceptional memory and afterwards into the working memory \citep[][p.\,8f.]{raake_speech_2006}.
While these types of memory have only a very limited storage time, in the range of seconds to minutes, the information that becomes encoded into the long-term memory may remain accessible for several years.
Memorized information decays, reducing the amount of \emph{original} information even further.
In addition, recall from long-term memory is not a perfect process.
On recall, the original information is complemented by additional, potentially unrelated, information \citep[\cf,][]{schacter_seven_2003}.
In fact, unrelated, recalled information might even prevent the recall of specific information \citep[\cf,][]{schacter_seven_2003}.
Here, prior knowledge and also prior experiences affect the actual recalled information.
%In fact, specific information might not be accessible at all at the time of recall.

Two aspects are important with regard to experiences and their retrospective judgments.
First, how experiences are stored and in which manner is the information about individual experiences grouped together?
Second, what effects/biases determine retrospective judgments of an episodic experience?
Based upon the concept of episodic memory, this leads to the definition of the \emph{usage episode} that forms the basis for my work on multi\-/episodic perceived quality.

\subsection{Episodic Memory}
Personal experiences and related information are stored in the \emph{episodic memory} \citep{tulving_episodic_1972}.
This memory stores items of information and their spatial\-/temporal relationship \citep[][p.\,385]{tulving_episodic_1972}.
The items are grouped together by specific events, or so-called episodes.
These are encoded based upon their autobiographical reference to pre-stored content \citep[][p.\,385f.]{tulving_episodic_1972}.
In addition, temporal and spatial information is stored.
This includes information about the situation and feelings \citep[][p.\,385f.]{tulving_episodic_1972}.
An episode has an explicit start and end while retaining a temporal order of information \citep[][p.\,262]{conway_construction_2000}.
Based upon recallable information about an episode, this episode can be re\=/experienced and thus enabling \emph{mental} time-travel.
This is denoted as memory-vividness \citep{conway_construction_2000}.
An episode is successfully memorized and retrieved if the perceptual properties and their approximate temporal relationship to other episodes can be described \citep{conway_construction_2000}.
%The retrieval process can be supported by providing a cue.

The episodic segmentation process, \ie, constitution in the memory of an actor, is expected to happen during the actual experience \citep[][]{ezzyat_what_2011, kurby_segmentation_2008}.
Here, the goal of an actor and the temporal proximity of individual events are considered of special importance \citep[][]{black_episodes_1979}.
It must be noted that first-time episodes and repeated episodes are considered different with regard to memorization (\citet{conway_construction_2000} referencing \citet{barsalou_content_1988}).
The latter are linked by a shared theme, but tend to retain less specific information about the individual episodes \citep{robinson_first_1992}.

The characteristic of the episodic memory affects retrospective judgments due to the grouping of information and thus the ability to recall.

\subsection{Effects on Retrospective Judgments}\label{chap03:effects}
For a \emph{retrospective judgment} of an experience, a person must rely on information about this experience that has been \emph{a)} perceived and \emph{b)} memorized, and can be \emph{c)} recalled while conducting this judgment.
With regard to retrospective judgments, major work has been done for the judgment of pain, showing several, even counterintuitive, effects.
It has been found that not all parts of an experience affect a retrospective judgment equally.
These effects are described in the following.

\paragraph*{Primacy and Recency}
Primacy and recency have first been observed for sequential learning \citep[][]{murdock_jr._serial_1962}.
In a sequential learning task, it has been found that the likelihood of recalling specific items afterwards depends on the position of an item.
Items at the very beginning and very end have an increased likelihood of being recalled correctly.
These are denoted as \emph{primacy effect} and \emph{recency effect}, respectively.
The former denotes an increased likelihood of recalling items from the beginning and the latter the increased likelihood of recalling final items.

With regard to retrospective judgments of an individual experience, a recency effect is often observed.
In an episode with varying pain, the retrospective judgment is lower, \ie, the episode is judged less painful if less pain occurs at the end \citep[][]{kahneman_when_1993, redelmeier_patients_1996}.
This could also be observed in cases where an episode was extended by a less painful ending.
An effect of primacy has not been observed for retrospective evaluation of pain, indicating that the beginning of such an experience has a reduced importance for the retrospective judgment.

\paragraph*{Peak}
For retrospective judgments of an experience, it has been observed that an outstanding part is overrepresented in the retrospective judgment.
This is denoted as \emph{peak} effect.
Such an effect has been observed in retrospective assessment of pain \citep[][]{kahneman_when_1993, redelmeier_patients_1996}.
Here, a spike in \emph{momentary pain} results in a severely worse retrospective judgment.
It seems as if exceptional parts of an experience can either be better memorized or are more likely to be recalled.
Peak effects are most often considered for \emph{negative} peaks, \eg, outstanding pain, but rarely for positive peaks, \eg, outstanding pleasure.

\paragraph*{Duration Neglect}
In retrospective judgments, a duration neglect has been found.
Here, it has been observed that the actual duration of an experience has only a reduced to no impact on a retrospective judgment.
This has been mainly observed for retrospective judgments of pain \citep[][]{fredrickson_duration_1993, ariely_combining_1998}.
%redelmeier_memories_2003

\subparagraph*{}
This overview shows that retrospective judgments of experiences are affected by characteristics of the episodic memory.
Here, information might not be recallable, not recalled precisely, or not considered with similar importance.
The observed effects/biases indicate that not all parts of an experience are equally important for a retrospective judgment.
Rather, the formation process of retrospective judgments seems to assign a special importance to individual parts.
The observed biases have been used as the basis for investigating retrospective judgments of perceived quality, which is presented in the following.
%Similar observations have also been made for \citet{hogarth_order_1992}
