\section{Retrospective Judgments of Experiences}\label{chap:03} % of Experiences
%\begin{chapter-abstract}
%Here I present how recall is influenced by human memory (and thus retrospective judgments).
%I give an multi-disciplinary overview on this topic starting from episodic memory (and thus skipping all short-term/working memory influences).
%I focus on temporal effects on learning (as a baseline) and also temporal effects on the recall of experiences (like pain, happiness etc.).
%Most important point here is that retrospective judgments always rely on \textit{recalled information}, which must a) not be complete and b) not correct at all.
%I will also discuss factors affecting memorization (\eg,attention, importance) and memory failures \citep{schacter_seven_2002}.
%
%I must very carefully explain here the different aspects of time for perception, memorization and retrieval.
%This should include perception (and underlying adaptation of perceiving organs) at and also time perception \cite{wittmann_neuropsychologie_2009}.
%
%Because experiences are grouped and memorized together into an \emph{episode}, the most important aspect of this chapter is the \emph{episodic memory}.
%A conceptual definition of \emph{episode} (based on memorization with regard to retrospective assessment) is hinted that forms the basis for the definition of \emph{usage episode}.
%\end{chapter-abstract}

%"A perceptual event can be stored in the episodic system solely in terms of its perceptible properties."\cite[p. 385]{tulving_episodic_1972}

%sage episode and episodic memory [Tulving, Black+Bower, Ezzyat]
%\begin{itemize}
%\item remembering vs. knowledge \cite{tulving_concepts_2000}
%\item Episodic memory "temporally dated episodes or events and temporal-spatial relations" \cite[p. 385]{tulving_episodic_1972}
%\item Experienced events have a temporal order (Retrieval should also reveal this order!)\cite{tulving_concepts_2000}
%\item autobiographical reference, personal identity [Tulving p.10], \cite{conway_construction_????}
%\item What? When? Where? Situation+Feelings [Tulving p.10]
%\item Explicit start and end [Conway]
%\item Time-travel (I must be able to travel back to that situation); memory-vividness \cite{conway_construction_????}
%\item Successful retrieval requires successful encoding! (successful retrieval: a person can describe perceptual properties AND temporal relations to other events)
%\item Events are part of an episode. -> Providing a cue for one episodes allows to retrieve information this episode alone!
%\end{itemize}

%\section{Event Segmentation Theory}
%\cite{black_episodes_1979}\cite{ezzyat_what_2011}\cite{zacks_perceiving_2001}
%\begin{itemize}
%\item by goal 
%\item temporal closeness \cite{black_episodes_1979}\cite{ezzyat_what_2011}
%\item "that segments experience (events) into episodes" \cite[p. 248]{ezzyat_what_2011}
%\item event segmentation happens while experiencing 
%\end{itemize}
%
%\section{Episode}\label{def:episode}
%Previous work on \ac{QoE} focused on the assessment of individual stimuli or interactions and omitted the integration process over multiple episodes with a service or system.

For a retrospective judgment of an experience a person can only rely on available information to him.
While experiencing a person needs to encode and memorize information, so he is later able to recall this information for his judgment.
Already during perception the amount of information is heavily reduced, encoding only a reduced set into the perceptional memory and afterwards into working memory \cite[\cf, ][p. 8f.]{raake_speech_2006}.
While those types of memory only have very limited storage time in the range of seconds up to minutes, the information that become encoded into the long-term memory may remain accessible for several years.
Memorized information decays, reducing the amount of "original" information even further.
In addition, to reduction and conversion of information until it might be finally stored in the long-term memory, recall is not a perfect process.
On recall the available information is complemented by additional available sources like prior knowledge or even previous experiences \citep{schacter_seven_2003}.
%In fact, specific information might not be accessible at all at the time of recall.

With regard to experiences and their retrospective judgments two aspects are important.
First, how are experiences are stored and in which manner is the underlying information grouped together.
Based upon the concept of episodic memory, this leads to the definition of the \emph{episode} that forms the basis for my work on multi-episodic \ac{QoE}.
Second, known effects affecting retrospective judgments of experiences are presented.

\subsection{Episodic Memory}
Personal experiences and related information are stored in the \emph{episodic memory} \citep{tulving_episodic_1972}.
This memory stores information and their temporal-spatial relationship \citep[][p. 385]{tulving_episodic_1972}.
The information is grouped together by specific events, or so-called episodes.
Information is encoded based upon their autobiographical reference to already stored content \citep[][p. 385f.]{tulving_episodic_1972}.
In addition, temporal and spatial information is stored including information about the situation as well as feelings \citep[][p. 385f.]{tulving_episodic_1972}.
An episode has an explicit start and end while retaining a temporal order of information \citep[][p. 262]{conway_construction_2000}.
Based upon recallable information about an episode, this episode can be re-experienced enabling \emph{mental} time-travel, which is denoted as memory-vividness \citep{conway_construction_2000}.
An episode is successful retrieved, if the perceptual properties and their approximate temporal relationship to other episodes can be described.
If memorized a cue can be used for easier recall.
It must be noted that first-time episodes and repeated episodes are considered different (\citet{conway_construction_2000} referencing \citet{barsalou_construction_1988}).
The latter are linked by a shared theme \citep{robinson_first_1992}, but tend to retain be less distinct information about specific episodes.

The actual episodic segmentation process, \ie, constitution in the memory of an actor, is expected to happen during the actual experience \citep{ezzyat_what_2011, kurby_segmentation_2008}.
Here especially the goal of an actor and temporal closeness of individual events are considered of special importance \cite{black_episodes_1979}.

\subsection{Effects on Retrospective Judgments}
For a \emph{retrospective judgment} this person must rely on information about this experience that has been \emph{a)} perceived, \emph{b)} memorized and \emph{c)} can been recalled while conducting this judgment.
With regard to retrospective judgments major work has been done for the judgment of pain, showing several, even counterintuitive effects.

\paragraph*{Primacy and Recency}
Primacy and recency have first observed for sequential learning \citep[\cf,][]{murdock_jr._serial_1962}.
In a sequential learning task, it has been found that the likelihood to recall specific items afterwards depends on the position.
Items at the very beginning and at the very end have an increased likelihood to be recalled correctly.
This is denoted as \emph{primacy effect} and \emph{recency effect}.
The former denotes an increased likelihood to recall the beginning and the latter the increased likelihood to recall final items.
With regard to judgments of experience a recency effect is often observed.
In an episode with varying pain the retrospective judgment is better, \ie, the episode is judged less painful, if less pain occurs at the end \citep[\cf,][]{kahneman_when_1993, redelmeier_patients_1996}.
Such an effect could also be observed, when an episode is extended by a less painful ending.
An effect of primacy has not been observed for retrospective evaluation of pain, indicating that the beginning of such an experience has reduced importance on a retrospective judgment.

\paragraph*{Peak}
For retrospective judgments of an experience a \emph{peak} effect has been observed.
Here the most outstanding part has an increased impact on the retrospective judgment.
Such an effect has been observed in retrospective assessment of pain \citep[\cf,][]{kahneman_when_1993, redelmeier_patients_1996}.
Here a spike in momentary pain measurements severely reduces the retrospective judgment.
A peak effect could also been observed in retrospective evaluation of perceived quality (\cf, \autoref{chap:04}).
Peak effects are most often considered for \emph{negative} peaks, \eg, outstanding pain, but rarely for positive peaks.

\paragraph*{Duration Neglect}
In addition to recency and peak effects also neglect of duration has been observed on retrospective judgments.
Here the actual duration of an experience is neglected rather than considered.
This has been observed for retrospective judgments of pain \citep[\cf,][]{fredrickson_duration_1993, ariely_combining_1998}.
%redelmeier_memories_2003