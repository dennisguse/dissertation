\chapter{Temporal aspects for Recall and Experiences}\label{chap:03}
\section*{Abstract}
Here I present all known effects on retrospective judgments of general experiences (like pain, happiness etc.).
This includes also the episodic memory and memory failures.
A definition of \emph{episode} (based on memorization with regard to retrospective assessment) is derived.
%Extended concepts are explained like utility and also importance for Usability and adaptation of behavior.?

I must very carefully explain here the different aspects of time.
This should include perception (and underlying adaptation of perceiving organs) at and also time perception (Wittmann?)

\section{Temporal Effects}
\begin{itemize}
\item Recency
\item Peak
\item Primacy
\item Duration neglect
\item Adaptation speed (from low to high vs. high to low)
\end{itemize}

\section{Epsiodic memory}
Usage episode and episodic memory [Tulving, Black+Bower, Ezzyat]
\begin{itemize}
\item Colloquial:  http://www.merriam-webster.com/dictionary/episode (an event that is distinctive and separate although part of \item remembering vs. knowledge [Tulving]

\item Episodic memory "temporally dated episodes or events and temporal-spatial relations" [Tulving p.385]
\item "A perceptual event can be stored in the episodic system solely in terms of its perceptible properties [Tulving p.385]
\item Experienced events have a temporal order (Retrieval should also reveal this order!) [Tulving, Kahneman: objective happiness]

\item autobiographical reference, personal identity [Tulving p.10], [Conway]
\item -> What? When? Where? Situation+Feelings [Tulving p.10]
\item Explicit start and end [Conway]
\item Time-travel (I must be able to travel back to that situation); memory-vivedness [Conway]
      
\item Successful retrieval requires successful encoding! (successful retrieval: a person can describe perceptual properties AND temporal relations to other events)

\item Events are part of an episode. -> Providing a cue for one episodes allows to retrieve information this episode alone!
\item What is a moment? (Kahneman said a moment is usually 3 seconds?)
\end{itemize}

\section{Event segmentation theory}
[Black+Bower, Ezzyat, Zacks]
\begin{itemize}
\item by goal [Black+Bower]
\item temporal closeness [Black+Bower, Ezzyat p.248]
\item "that segments experience (events) into episodes" [Ezzyat p.248]
\item event segmentation happens while experiencing [Ezzyat p.248]

\item [Conway] following [Barsalou 1988] Event-specific knowledge (ESK)
\item -> general "events": repeated events (e.g. evening hikes) and single events (trip to Paris)
\item -> Series of memories linked to together by a theme
\item -> Goal-attainment knowledege
\item First encounter (better encoded->more important AND seems to set expectations)
\item Repeated encounter
\end{itemize}

\section{Utility / Happyiness / Well-being}
Kahneman?

\section{Learning}
\begin{itemize}
\item sequential learning
\end{itemize}

\section{Memory failures}
Schacter, but not that important as he discusses mainly semantic failures.
The failures are not about opinions and their recall.

\begin{itemize}
\item Failure to encode
\item Failure to recall
\item Mis-attribution
\end{itemize}


\section{Belief-Adjustment model}
Describe Hogarth's model and describe found effects!

Importance of content [Baumgartner]: Ads and positive peak effect on remembering content