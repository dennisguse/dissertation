\section{Perceived Quality and Macroscopic Fluctuations}\label{chap:04}
The performance of telecommunication services is in general not constant, but rather varies over time.
Performance fluctuations can occur due to varying network transmission, but also due to applied lossy compression.
With regard to perceived quality, only those performance fluctuations must be considered that affect the quality formation process of a user.

Fluctuations that affect the quality formation process are distinguished as \emph{microscopic} and \emph{macroscopic} \citep[][p.\,72]{raake_short-_2006}.\footnote{\citet{raake_short-_2006} distinguishes microscopic and macroscopic with regard to packet-loss behavior for speech telephony. The notation used here is generalized to be independent of the actual source of fluctuations.}
This differentiation focuses on performance fluctuations in one stimulus or episode.
The former is a fluctuation that is not perceived as variation in perceived quality.
Such fluctuations are often rather short, such as non-bursty packet-loss in a \ac{VoIP} call \citep[\cf,][p.\,72]{raake_short-_2006}.
In contrast, macroscopic fluctuations are perceived and judged as variation in perceived quality.
An example of macroscopic fluctuations is a noticeable change in video encoding bandwidth.

The impact of \emph{macroscopic} performance fluctuations on retrospective judgments of the perceived quality has already received some attention for telecommunication services.
Although some approaches have been undertaken, the impact of \emph{varying perceived quality} on a retrospective judgment and the prediction of such judgments are not yet completely solved.
An overview on the state of the art is given in the following, starting with the assessment methods for varying \emph{macroscopic} performance.
Subsequently, an overview on observed effects is given, followed by a short presentation of modeling approaches of retrospective judgments.

\subsection{Quality Assessment of \emph{Macroscopic} Fluctuations}
The perceived quality for \emph{macroscopic} performance fluctuations can be assessed by requesting a user to judge the quality of this experience in retrospection.
A retrospective judgment can be used to deduce the actual experience, but especially in the case of longer experiences, a retrospective judgment may not contain all desired information about an experience.
A final retrospective judgment can be complemented by \emph{momentary} judgments and \emph{intermediate} retrospective judgments.

For momentary judgments, the \emph{current} perceived quality is assessed continuously during the experience.
This allows the investigation of the noticeability of fluctuations, which might not be deducible from a retrospective judgment alone.
This method is called \ac{SSCQE} and is standardized for video quality assessment~\citep[][]{itu-r_recommendation_bt.500-13_methodology_2012}, but has also been applied for the evaluation of speech-only stimuli~\citep[\eg,][]{gros_instantaneous_2001}.
While experiencing, the momentary perceived quality should be judged by adjusting a slider.
The position of the slider should reflect the currently perceived quality.
It has been observed that a reduction in performance almost instantaneously leads to a reduction in the momentary judgment, but that adaptation due to improvements are delayed~\citep[\eg,][]{hands_recency_2001, gros_instantaneous_2001, hamberg_time-varying_1999}.
\citet{borowiak_long_2013} extended the \ac{SSCQE} method by not assessing momentary judgments.
Instead, a participant is allowed to react to macroscopic fluctuations by adjusting the performance to the desired level.
%In fact, momentary judgments have the inherent limitation that they affect the actual experience as an additional task must be conducted while experiencing.

The impact of macroscopic fluctuations can also be assessed by intermediate retrospective judgments.
Here, a stimulus is split into individual parts.
Each part is presented and a retrospective judgment, representing an intermediate judgment, is taken individually.
The intermediate judgments allow a fine-grained analysis of the impact of the fluctuations.

With regard to the investigation of macroscopic fluctuation, the impact of varying user behavior is an issue.
A \ac{MOS} can only be derived using those judgments, which are based on identical or very similar stimuli and thus are assumed to lead to similar experiences.
Because perception and experience are influenced by an actor's behavior, varying usage behavior limits the applicability of the \ac{MOS} computation.
This can be overcome by either limiting the user behavior completely, \ie, permitting passive consumption and assessment only or by enforcing a certain behavior.
The latter can be achieved by providing instructions to participants, or letting them solve a task that can only be solved in a limited number of ways.
For the evaluation of conversational speech telephony, for example, \acp{SCS} have been developed.
Here, the information that is to be exchanged is defined.
Although a conversational structure is suggested, the exact timing is not enforced, and thus the assessment of macroscopic performance fluctuations is limited.
An alternative is the method of \emph{simulated conversations} \citep{weiss_modeling_2009, berger_estimation_2008}.
This method enforces a \emph{realistic} and reproducible user behavior including speaker changes, and is standardized as ETSI\,102506 \citep{etsi_speech_2011}.
Here, a telephone conversation is split into individual parts of listening and speaking.
Speaking parts and listening parts are then concatenated alternatingly in a meaningful order to create a simulated conversation.
For speaking parts, predefined questions should be answered by the participant.
This has been done orally as well as written.
This should enable an otherwise passive listener enable to feel as though he is taking part in a real conversation.
This method allows the presentation of a comparable stimulus, except the exact speaking phases, to multiple participants including precisely timed degradations.

\subsection{Effects on Retrospective Judgments}
The impact of macroscopic performance fluctuations on retrospective judgments has been investigated mainly for video transmission and speech telephony.
Here, similar effects to retrospective judgments of general experiences have been observed (\cf, \autoref{chap:03}).
Most often, a recency effect, and in some cases a peak effect, is observed.
Duration neglect has received only limited attention, but could be observed in some cases.
A primacy effect has not been observed for perceived quality.

Although effects could be observed, this is not always the case, and the reasons for this are still under investigation.
%One practical goal of research on \ac{QoE} is the actual prediction of \ac{QoE} judgments often for very specific case including technology and resulting performance fluctuations such as \ac{VoIP}.
Fundamental work was conducted by \citet{hands_recency_2001}.
Their work is based on the \emph{belief-adjustment model}~\citep{hogarth_order_1992}.
This model explains the occurrence of recency effect, primacy effect, and duration neglect for the integration of new information into one's belief. % depending on the complexity of new information and the underlying formation process.
For \unit[30]{s} video sequences, \citet{hands_recency_2001} could observe a recency effect as well as a duration neglect.
The duration neglect was shown by presenting either \unit[5]{s} or \unit[10]{s} of reduced performance, but no impact on retrospective judgments was observed.
This effect occurred although participants were able to assess the duration closely. %(\unit[5]{s}: \unit[6.1]{s}; \unit[10]{s}: \unit[11.1]{s}).
In fact, the recency effect could be observed only if no intermediate judgments were taken.
If momentary judgments were taken additionally, a recency effect could not be observed.
This indicates that the momentary judgments affect the quality formation process due to the presence of the explicit assessment.
\citet{hamberg_time-varying_1999} also observed both effects for video sequences of up to \unit[180]{s} while varying impairment duration from \unit[2]{s} to \unit[10]{s}.
With regard to shorter stimuli, effects on retrospective judgments are rarely observed.
In fact, such effects seem to diminish if the length of an experience is reduced.
For example, \citet{ninassi_considering_2009} did not find a recency effect for \unit[8]{s} videos.
%It must be noted that smooth changes in performance seem yield for video transmission better retrospective judgments than abrupt changes \citep[\eg,][]{egger_impact_2014}.
%Beside variation in performance presentation also stalling has been recently investigated.
%It has been found that initial stalling is affecting a final retrospective judgment less than stalling while playback \citep[\cf,][]{hossfeld_pippi_2013}.
%Here it was furthermore observed that increasing the number of stallings results in higher reduction in retrospective judgments than the increasing the duration of one stalling event.

Beside video transmission, the impact of macroscopic performance fluctuations has been investigated for (speech) telephony. %In difference to \ac{PSTN}-based telephony 
Here, a recency effect could also be observed \citep[\eg,][]{rosenbluth_testing_1998, hamberg_time-varying_1999, gros_instantaneous_2001, gros_effects_2004, belmudez_audiovisual_2015, weiss_modeling_2009, lewcio_management_2014} whereas a negative peak effect has been less often observed \citep[\eg,][]{weiss_modeling_2009, belmudez_audiovisual_2015, lewcio_management_2014}.
The work of \citet{weiss_modeling_2009, lewcio_management_2014, belmudez_audiovisual_2015} is based on the method of \emph{simulated conversations} \citep{etsi_speech_2011}.
Here, the perceived quality of a simulated conversation is complemented by judgments of the individual listening parts and speaking parts.
Analyzing the relationship between the intermediate judgments and the retrospective judgment, enables the investigation of potential effects.
In addition, \citet{rosenbluth_testing_1998} investigated and observed a duration neglect for speech telephony.

With regard to macroscopic performance fluctuations and their impact on a retrospective judgment of the perceived quality, the state of the art is rather limited.
Although recency effect, peak effect, and duration neglect have been observed, it is not known under which circumstances these occur.
In fact, the characteristics of these effects are not yet fully understood, \eg, window of increased importance due to a recency effect for specific cases.
One reason for this is the incomparability of the conducted experiments.
Therefore, effects were observed repeatedly, but exact characteristics can hardly be derived.
%Although a negative peak effect could be observed, a positive peak effect has to my knowledge not been investigated.

\subsection{Prediction of Retrospective Judgments}
One practical goal of research on \ac{QoE} is the prediction of retrospective judgments.
Retrospective judgments can be predicted using either momentary or intermediate judgments as well as predictions of these judgments.\footnote{If the momentary or intermediate judgments are different to the final retrospective judgment, for example using a different scale or assessing something different, these judgments must be first transformed before predicting the retrospective judgment.}
An alternative is to omit the prediction for these judgments and use a parametric description of the complete stimulus to predict the final retrospective judgment directly.

The \emph{baseline model} for temporal integration is based on the assumption that no effects occur, \ie, that all individual parts of an experience are equally important.
This can be represented by the \emph{unweighted} arithmetic mean of all momentary or intermediate judgments.
This model can be improved by accounting for observed effects that result in a deviation between the prediction and the judgment that is to be predicted.
The baseline model can be extended by using a \emph{weighted arithmetic mean} and a weight function.
Here, a recency effect can be modeled by increasing the weight of later parts \citep[][]{rosenbluth_testing_1998, weiss_modeling_2009, hamberg_time-varying_1999}.
In a similar manner, a peak effect can be modeled.
%With regard to peak effect \citet{weiss_modeling_2009} takes a different approach by modeling a \emph{relative peak effect}.
%Here, the peak effect is modeled as difference between the average intermediate judgments and lowest intermediate judgment and subtracted from the weighted average.
However, the implemented prediction models in the state of the art for retrospective judgments of single stimuli or single episodes are very specific to the experimental findings.

\subsection{Conclusion}
Retrospective judgments of perceived quality show effects similar to the retrospective judgments of experiences in general.
However, the findings with regard to perceived quality remain so far inconclusive, as effects are regularly observed but rarely quantified.
Here, the major focus lies on a \emph{sufficiently} precise prediction independent of the underlying reason.
For example, a recency effect could be regularly observed, but it is not (yet) known under which circumstances it occurs, \eg, the minimal duration of an experience tending to show a recency effect.
Furthermore, it is not known if recency is affected by the usage situation or modality (\eg, is visually presented content affected in a similar way to auditory content?) etc.
In addition to a recency effect, a peak effect could be observed, while duration neglect only received limited attention.

In fact, research on \ac{QoE} has been and will probably remain a mainly technology\-/driven.
In particular, the wide variety of applications, technology, and fast-pacing technological changes limit the comparison and derivation of knowledge about the formation process of judgments on perceived quality.
Nevertheless, the state of the art shows that not all parts of an experience affect a retrospective judgment equally.
