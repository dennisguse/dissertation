\chapter{Effects of Performance fluctuations on episodic QoE}\label{chap:04}
\begin{chapter-abstract}
This is an overview chapter on \textit{related work}.
Here I will present the all prior work on QoE with regard to performance fluctuations/changes in a continuous usage (sub-episode/episode).
Performance fluctuations is not everything that yields varying, non-constant perception, but focuses on a noticeable change in an episode or sub-episode (e.g. codec/parameter changes, packet-loss, interactive delay, NOT speech induced noise or NOT broken echo canceler).
The major point is here to present an overview on temporal effects that affect \textit{retrospective QoE judgments for (sub-)episodes}.
In the second part, modeling methods for those effects are present and broken down on their essential parts (model parts) and, if possible, found parameters.

%Here I will present the all prior work on QoE with regard to performance fluctuations: speech telephony, video consumption, web-browsing.
%This will include assessments methods (ACR, SSCQ, task-driven vs. quality-driven assessment etc.)
%Assessments Methods that are basis for multi-episodic QoE are described in great detail otherwise only state.
%Assumption: Temporal perception AND memorization comparable over persons.

%State the implicit assumption: Temporal perception AND memorization comparable over persons.

%Definition usage episode!

%Start with a definition of usage episode; especially focus on "meaningfulness of an interaction"

%Describe temporal effects that were found in QoE research (include potential reasons, if known) and discuss relation-ship to human memory.

%TODO: Determine, if the belief-adjustment model can be applied! (Journal Paper idea)
%TODO: Service quality studies could be at least referred here (e.g., call into call center)
\end{chapter-abstract}




\section{Excurse: Performance fluctuations in sub-episode} %NOTE: Should be very small!
\begin{itemize}
\item listening-only speech transmission quality (non-stationary impairments)
\item video-only (Neige-Garcia et al.)
\item Web (one-page view only)
\item \cite{hands_recency_2001}
\end{itemize}


QoE assessment of part of an episode (something that is in itself not complete like listening to one sentence).
For those no "real" (meaningful) tasks should be solved, but rather the task is to \textit{experiencing} and judge \cite{egger_qoe_2014}.

\section{Performance fluctuations in one episode}
Here task-driven assessment methodologies are presented, which focus "usage" as part of an experience.
This includes also passive (task-free) media (audio or video) studies where the task is watch a movie that is in itself meaningful (longer than some seconds).

\begin{itemize}
\item Call Quality, Conversational assessment
\item "long-term" Video
\item Put Web-session in here?
\end{itemize}


\section{Modeling of Performance Fluctuations}
This section can be either split into two, if differences with regard to modeling are found.
Here I will show how the found temporal effects are modeled and use this as basis for modeling approaches in Chapter~\ref{chap:08}.

Keyword: Temporal pooling
%TODO Try to apply belief-adjustment model here ;)