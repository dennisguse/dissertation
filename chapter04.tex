\chapter{Effects of Performance fluctuations on (sub-)episodic QoE}\label{chap:04}
\section*{Abstract}
Here I will present the all prior work on QoE with regard to performance fluctuations: speech telephony, video consumption, web-browsing.
This will include assessments methods (ACR, SSCQ, task-driven vs. quality-driven assessment etc.)
Assessments Methods that are basis for multi-episodic QoE are described in great detail otherwise only state.
%Assumption: Temporal perception AND memorization comparable over persons.

Start with a definition of usage episode; especially focus on "meaningfulness of an interaction"

Describe temporal effects that were found in QoE research (include potential reasons, if known) and discuss relation-ship to human memory.

TODO: Determine, if the belief-adjustment model can be applied! (Journal Paper idea)
TODO: Service quality studies could be at least referred here (e.g., call into call center)

\section{Performance fluctuations in sub-episode}
\begin{itemize}
\item listening-only speech transmission quality (non-stationary impairments)
\item video-only (Neige-Garcia et al.)
\item Web (one-page view only)
\end{itemize}

\section{Performance fluctuations in one episode}
\begin{itemize}
\item Call Quality, Conversational assessment
\item "long-term" Video
\item Put Web-session in here?
\item 
\end{itemize}