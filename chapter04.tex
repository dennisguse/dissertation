\newpage
\section{Performance Fluctuations and Perceived Quality}\label{chap:04}
%This is an overview chapter on \textit{related work}.
%Here I will present the all prior work on QoE with regard to performance fluctuations/changes in a continuous usage (sub-episode/episode).
%Performance fluctuations is not everything that yields varying, non-constant perception, but focuses on a noticeable change in an episode or sub-episode (e.g. codec/parameter changes, packet-loss, interactive delay, NOT speech induced noise or NOT broken echo canceler).
%The major point is here to present an overview on temporal effects that affect \textit{retrospective QoE judgments for (sub-)episodes}.
%In the second part, modeling methods for those effects are present and broken down on their essential parts (model parts) and, if possible, found parameters.

%Here I will present the all prior work on QoE with regard to performance fluctuations: speech telephony, video consumption, web-browsing.
%Assumption: Temporal perception AND memorization comparable over persons.

%State the implicit assumption: Temporal perception AND memorization comparable over persons.
%Describe temporal effects that were found in QoE research (include potential reasons, if known) and discuss relation-ship to human memory.

The performance of telecommunication services is in general not constant, but often varies over time.
Performance fluctuations can occur due to varying network transmission, but also due to applied lossy compression, or even signal processing.
With regard to perceived quality only those fluctuations must be considered that lead to a noticeable difference and thus affect the quality formation process.
The impact of varying performance on retrospective judgments of the perceived quality for rather short stimuli as well as individual episodes has already received some attention for different telecommunication services.
Although some approaches have been undertaken, the impact of \emph{varying perceived quality} on a retrospective judgment and its prediction is not yet completely solved.

In the following the state of the art on in-episodic varying performance with regard to a final retrospective judgment and its prediction is presented.
This chapter closes with an overview on potential 

\subsection{Quality Assessment of Fluctuations}
Perceived quality of performance fluctuations is also assessed by requesting a user to judge the quality of experience in retrospective.
The retrospective description of the \emph{complete experience} is based upon memorized and recallable information.
However, the final retrospective description often does not contain all information about the experience especially with especially regard to noticeability of certain fluctuations.
A final retrospective judgment can be completed by \emph{momentary} judgments and \emph{intermediate} retrospective judgments.
For momentary judgments the \emph{current} perceived quality is assessed continuously while experiencing.
For intermediate judgments the experience is split into smaller parts and those assessed individually.
This allows to derive detailed information, but taken info account that the experience is affected by taking such judgments while experiencing and thus likely affects the quality formation process.
Those judgments can also be used to predict the retrospective judgment.





















\subsection{Performance fluctuations in sub-episode}
\begin{itemize}
\item listening-only speech transmission quality (non-stationary impairments)
\item video-only (Neige-Garcia et al.)
\item Web (one-page view only)
\item \cite{hands_recency_2001}
\end{itemize}

\cite{garcia_parametric_2014}


QoE assessment of part of an episode (something that is in itself not complete like listening to one sentence).
For those no "real" (meaningful) tasks should be solved, but rather the task is to \textit{experiencing} and judge \cite{egger_qoe_2014}.

\subsection{Performance fluctuations in one episode}
Here task-driven assessment methodologies are presented, which focus "usage" as part of an experience.
This includes also passive (task-free) media (audio or video) studies where the task is watch a movie that is in itself meaningful (longer than some seconds).

\begin{itemize}
\item Call Quality, Conversational assessment
\item "long-term" Video
\item Put Web-session in here?
\end{itemize}


\subsection{Prediction of Retrospective Judgments}
%Here I will show how the found temporal effects are modeled and use this as basis for modeling approaches in \autoref{chap:09}.Keyword: Temporal pooling