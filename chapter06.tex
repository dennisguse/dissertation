\chapter{Multi-episodic QoE in 1 hour}\label{chap:06}
\begin{chapter-abstract}
Here I describe my approach towards understanding multi-episodic QoE.
First question: can multi-episodic QoE be studied in "short" laboratory study?
I with my laboratory studies on with one speech telephony service focusing on effects of position(s) of degraded usage episodes (the finished Journal paper) and the already finished extensions.

Then the parallel-use studies are presented (impact of a 2nd service used in parallel).
This chapter closes with the description of cross-service QoE (using results of studies with two services) (However, results are limited).
%Important point: argue that must degradations must be severe!
\end{chapter-abstract}

\section{One Service}
Studied aspects using speech telephony

Describe study design in the beginning as it is similar for all sequential studies

\begin{itemize}
\item Position (Conversation and Listening): yes/no, found for conversation
\item Length of degradation (Conversation and Listening): yes, similar: saturation
\item Task (Conversation vs. Listening): their might be an effect due to mental capacity
\item Strength of degradation (Conversation): yes, supports peak
\item Length of degraded episode (Extension of Study 2): results not yet known %TODO
\item Does episodic quality judgment impact multi-episodic QoE? %TODO
\end{itemize}

\section{Two Services}
\subsection{Sequential use}
\begin{itemize}
\item Second \textit{unimpaired} service: reduced negative effect of LP episodes
\item Second \textit{impaired} service: same effect
\end{itemize}

\subsection{Parallel use}
Present here that using two services in parallel only has a slight impact on Web-QoE.
First present degraded web-only and then web+TV.

\subsection{Multi-episodic over multiple Systems} %TODO Cross-service as extra chapter?