\chapter{Multi-episodic Perceived Quality in one Session}\label{chap:06}
\begin{chapter-abstract}
Here I describe my approach towards understanding multi-episodic QoE.
First question: can multi-episodic QoE be studied in "short" laboratory study?
I with my laboratory studies on with one speech telephony service focusing on effects of position(s) of degraded usage episodes (the finished Journal paper) and the already finished extensions.

Then the parallel-use studies are presented (impact of a 2nd service used in parallel).
This chapter closes with the description of cross-service QoE (using results of studies with two services) (However, results are limited).
%Important point: argue that must degradations must be severe!
\end{chapter-abstract}

Initial work on multi-episodic perceived quality with \emph{defined use} focused on a usage period over several days (cf., \autoref{chap:05}). 
This period has been expected to allow new users of a service to determine its future use \citep[cf.,][]{moller_single-call_2011}.
However, it is so far unknown, if the length of usage period affects the formation process of multi-episodic perceived quality.
Whereas \cite{moller_single-call_2011} focused on a usage period of multiple days, multi-episodic perceived quality also occurs in one session that is divided into several distinct usage episodes.
This limits the duration of usage period to avoid influence factors like fatigue affect the formation process.
Typical experiments on perceived quality do not exceed \unit[90]{min}\todo{REF} to avoid such influence, but if a longer duration are required such experiments are split into multiple sessions.
Studying multi-episodic perceived quality in one session alone allows to conduct the experiment under controlled laboratory conditions similar to standardized experiments on perceived quality.
This avoids potential sources of noise for example due to differences in the presentation environment over several participants.
In fact, reducing the usage period to such a short time-frame compared to multiple days, allows to study a greater number on conditions in detail.
Results on multi-episodic perceived quality in one session can then be used as a starting point to investigate multi-episodic perceived quality over several days.



\section{One Service}
%Follow Moeller: sequential use -> seperate usage episodes
%Audio-only (avoid multi-modality)
%Controlled environment, same system for all

%Assume well working-system in the beginning for adaptation.
%Present peformance modes AND non-failure!

\subsection{Hypotheses}

\subsubsection*{\emph{H1}: Number of Degraded Episodes}
%Length of degradation (Conversation and Listening): yes, similar: saturation

\subsubsection*{\emph{H2}: Position of one Degraded}
%Position (Conversation and Listening): yes/no, found for conversation

\subsubsection*{\emph{H3}: Recovery}
%Telephony only

\subsubsection*{\emph{H4}: Length of Degraded Episode}
%Length of degraded episode (Extension of Study 2): none
%Aod only

\subsubsection*{\emph{H5}: Strength of one Degraded Episode}
%Strength of degradation (Conversation): yes, supports peak

\subsubsection*{\emph{H6}: Episodic Perceived Quality Judgment }
%Does the episodic quality judgment impact multi-episodic QoE???? AOD to be prepared! %TODO

\begin{table}
 \centering
 \begin{tabulary}{\textwidth}{C|C|C|C||C|C|C||C|C|C||C|C|C|C|C|C}
 Condition & \multicolumn{9}{c|}{Episodic Performance}        & \multicolumn{6}{c}{Hypotheses} \\
           & 1  & 2  & 3  & 4           & 5           & 6           & 7  & 8  & 9  & H1 & H2  & H3  & H4  & H5 & H6 \\
 \midrule
 1         & HP & HP & HP & \textbf{LP} & HP          & HP          & -  & -  & -  & -  &  *  &  -  &  * \\
 \hline
 2a        & HP & HP & HP & HP          & \textbf{LP} & HP          & -  & -  & -  &  * &  *  &  -  &  - \\
 \hline
 2b        & HP & HP & HP & HP          & \textbf{long LP} & HP          & -  & -  & -  &  * &  *  &  -  &  - \\
 \hline
 3         & HP & HP & HP & HP          & HP          & \textbf{LP} & -  & -  & -  &  $\star$ &  *  &  -  &  - \\
 \hline
 4         & HP & HP & HP & \textbf{LP} & \textbf{LP} & HP          & -  & -  & -  &  * &  $\star$  &  -  &  * \\
 \hline
 5a        & HP & HP & HP & HP          & \textbf{LP} & \textbf{LP} & -  & -  & -  &  $\star$ &  $\star$  &  -  &  - \\
 \hline
 5b        & HP & HP & HP & HP          & \textbf{LP} & \textbf{LP} & HP & HP & HP &  $\star$ &  $\star$  &  *  &  - \\
 \hline
 6         & HP & HP & HP & \textbf{LP} & \textbf{LP} & \textbf{LP} & -  & -  & -  &  $\star$ &  -  &  -  &  - \\
 \hline
 7         & HP & HP & HP & HP          & \textbf{LP} & \emph{MP}   & HP & HP & HP & -  &  -  &  *  &  * \\
 \end{tabulary}
 \caption{Overview of all conditions with the episodic performance of all usages episodes and showing which conditions are compared for each of the three hypotheses.
 Non-HP episodes are in bold (LP) and italic (MP).
 Position of multi-episodic QoE judgments are marked by \emph{double vertical lines}.
 All conditions are evaluated in Study~1.
 In Study~2 all conditions except 5b and 7 are applied.}
 \label{tab:chap06:hypothesesComparison}
\end{table}

\subsection{Experimental Design}
%Feedback: scale + questions
%System description Appendix?
%Digitial feedback, paper allows to go back in time...
%Task: Conversation vs. Listening (Note taking) vs. Listening (retrospective content)
%Type of services
%Shared characteristics HP vs. LP (Table!) with MOS_LQO!

\begin{table}
 \centering
 \begin{tabulary}{\columnwidth}{C|C|C|C}
   Performance & Signal bandwidth & Codec & $MOS_{LQO}$ \\
   \midrule
   HP & 50 to \unit[7000]{Hz}  & G.722, Mode 1 & 4.0 \\ %MOS1-5: 3.9
   \hline
   MP & 300 to \unit[3400]{Hz} & G.711         & 3.3 \\ %MOS1-5: 3.3
   \hline
   LP & 300 to \unit[3400]{Hz} & LPC-10        & 1.9 \\ %MOS1-5: 2.0
   \end{tabulary}
   \caption{Details of performance levels (\ac{HP}, \ac{MP} and \ac{LP}) with \ac{POLQA} prediction (Mode: Super-wideband). The prediction was transformed on the continuous 7-pt scale shown in \autoref{img:chap05:quality-scale} by applying the transformation described by  \cite{koster_comparison_2015}.}
   \label{tab:performance}
\end{table}

\subsection{Experiments}


\subsection{Results}

\subsubsection*{Aspect: Task influence}
%Task (Conversation vs. Listening): their might be an effect due to mental capacity
%No Conversation related degradation...


\section{Two Services}
\subsection{Sequential use}
\begin{itemize}
\item Second \textit{unimpaired} service: reduced negative effect of LP episodes
\item Second \textit{impaired} service: same effect
\end{itemize}

\subsection{Excurse: Parallel use}
Present here that using two services in parallel only has a slight impact on Web-QoE.
First present degraded web-only and then web+TV.