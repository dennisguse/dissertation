\chapter{Towards Multi-episodic Perceived Quality}\label{chap:towards}

\begin{chapter-abstract}
Here I describe my approach towards understanding multi-episodic \ac{QoE}.
Based upon \citet{moller_single-call_2011}, I approach multi-episodic perceived quality in two periods: one session and multiple days.

\begin{itemize}
\item Only do limited assumptions! (we don't know anything)
\item Find anomalies that are not reflected by "averaging"
\item What do I need for modeling?
\item Where are limits?
\end{itemize}

I present the tasks/usage situation, performance levels and hypotheses in detail.

\textbf{TODO}: Conditions should be moved from the following Chapter to this one!
\end{chapter-abstract}

First work on multi-episodic perceived quality with a defined-use methodology was conducted by \citet{moller_single-call_2011}.
Although only limited results were obtained, the conducted experiment showed that this assessment methodology can be successfully applied.
This experiment indicated that the formation process of multi-episodic perceived quality is unlikely to be resembled by an averaging process of the perceived quality of all prior usage episodes but more likely by a more complex process.
Based upon this and prior work on temporal effects on retrospective judgment, I conducted a series of experiments to examine the formation process in detail to derive potential model components and create a prediction model based upon episodic perceived quality judgments.

\section{Research Method}
\subsection{Prior Experiences}
\citet{moller_single-call_2011} investigated multi-episodic perceived quality for the case, in which a service delivers highest performance for the very first usage episodes and latter episodes are potentially presented with reduced performance.
This allows a person to experience the service under the best performance, and familiarize this performance level and the service characteristics.
When reduced performance occurs this person can compare his current experience with his prior experiences with said service.
This avoids that the person compares the current experience resulting from reduced performance to prior experiences that do not result of experiences with said service.
In the experiment by \citet{moller_single-call_2011} at least four episodes were presented with highest performance.
In my experiments I follow this concept.

\subsection{Judgments}
Episodic perceived quality judgments are collected, which are in the following denoted as \emph{episodic judgments}.
The episodic judgment allows to determine, how the performance of a usage episode was perceived.
Those judgments are collected directly after finishing a usage episode.
%This ensures that the judgment reflects the perceived quality as precise as possible.

The investigation of multi-episodic perceived quality will be based on the retrospective judgment, which is in the following denoted \emph{multi-episodic judgment}.
For this judgment a person is requested to assess his \emph{perceived quality of all prior usage episodes with the service}.
It is assumed that different conditions, if those are different enough, manifest in consistent variations in the multi-episodic judgments.
Here \emph{condition} refers to the order of episodes with defined performance level and their occurrence in time.

Similar to episodic judgments, multi-episodic judgments are acquired after finishing a usage episode.
If a multi-episodic judgment is required, it is always assessed after the episodic judgment to prevent an influence on the episodic judgment due to the presentation of an additional question occurring only in some cases.
In fact, assessing multi-episodic \ac{QoE} directly after finishing including the episodic judgment might increase the impact of the this very episode on this multi-episodic judgment.
However, it is not yet known, if such an effect occurs and how much time should be between the episodic judgment and multi-episodic judgment.
Following \citet{moller_single-call_2011} the 7-point \ac{CCR} scale is applied for all judgments about perceived quality (\cf, \autoref{img:chap05:quality-scale}, \autopageref{img:chap05:quality-scale}).
The 7-point \ac{CCR} allows for more fine-grained judgments than the 5-point \ac{ACR} scale.
This enables to observe small differences between different conditions, but can be a source of noise, also in-subject, as more rating possibilities are provided.
Using the same scale for episodic judgments and multi-episodic judgments enables a direct comparison without scale conversion.

\subsection{Performance Levels}
\citet{moller_single-call_2011} applied per condition two performance levels \acf{HP}, and \acf{LP}.
\ac{HP} denotes the highest performance that should lead a higher episodic judgment than\ac{LP}, which denotes the lowest performance.
For both performance levels the overall maximum transmission rate was limited only.
This performance limitation was selected as a near \emph{constant} quality per episode desired. %"stable quality judgment"
This avoids effects due to in-episodic quality fluctuations on episodic judgments and multi-episodic judgments as those are not yet fully understood (\cf, \autoref{chap:04}).

The results of \cite{moller_single-call_2011} show that episodic judgments are in line with the defined performance levels as those show a clear difference of episodic judgments between \ac{HP} and \ac{LP}.
However, the impact on the multi-episodic judgments due to \ac{LP} usage episodes is very limited (\cf \autoref{chap:04}).
This indicates that the selected performance level for \ac{LP} did produce degradations that were perceived, but were not severe enough to produce a clear, observable effect on multi-episodic judgments.
Thus, \ac{LP} must be selected in such a way that degradations are \emph{severe} enough to produce an observable effect on multi-episodic perceived quality, but does still allow successful task fulfillment.

Per condition two performance levels alone do not allow to investigate the existence of a peak effect (\cf \autoref{chap:state-of-the-art}).
This effect states that the worst part of an experience, has a higher impact on a retrospective judgment.
For this investigation at least a third performance level is required that provides a higher episodic quality than \ac{LP}, but lower than \ac{HP}.
This performance level is in the following denoted as \acf{MP}.

\subsection{Usage Periods}
%TODO Make two aspects CLEAR: SESSION and multiple days!
Initial work on multi-episodic perceived quality with the \emph{defined use} methodology focused on a usage period over several days (\cf, \autoref{chap:state-of-the-art}). 
This period has been expected to allow new users of a service to determine its future use \citep[\cf,][]{moller_single-call_2011}.
However, it is so far not known, if the length of the overall usage period affects the formation process of multi-episodic perceived quality.

Whereas \citet{moller_single-call_2011} focused on a usage period of multiple days, multi-episodic perceived quality also occurs in one session, if this session consists of multiple, distinct episodes.
A session could for example consists of several separate telephone calls in which each is a usage episode.
%This limits the duration of usage period to avoid influence factors like fatigue affect the formation process.
%Typical experiments on perceived quality do not exceed \unit[90]{min} to avoid such influence.
Studying multi-episodic perceived quality in one session alone allows to conduct the experiment under controlled laboratory conditions similar to standardized experiments on sub-episodic as well as episodic perceived quality.
Results on multi-episodic perceived quality in one session can be used as a starting point to investigate multi-episodic perceived quality over several days.
%In fact, reducing the usage period to such a short time-frame compared to multiple days reduces the required effort, allows to study a higher number on conditions in detail.
%This avoids potential sources of noise for example due to differences in the presentation environment over several participants.


\subsubsection{Service Types}
With regard to multi-episodic perceived quality for telecommunication services, those service types are of higher interest that are widely and also frequently used in daily life.
In the following the different services types are presented, which are in the experiments applied to investigate multi-episodic perceived quality.
Different service types must be considered, because it is not known, if the formation process of multi-episodic perceived quality is affected by the service type.
For multi-episodic perceived quality tasks should be selected in such way that those are meaningful enough to result in an episodic experience.

%TODO USAGE EPISODE!!!!
%TODO usage episode length (my desired length)?
%TODO similar usage behavior: non-interactive services

\subsubsection*{Telephony}
Speech telephony services provide the live communication of between two or more remote parties for spoken interaction.
This is a well established and, in fact, classical service for telecommunication providers.
In fact, the quality perception and underlying influence factors for speech telephony are well understood and standardized evaluation methods have been developed for evaluation of perceived quality.

\subsubsection*{Two-party Conversation}
Speech telephony is most often used for the communication between two remote parties, who engage in an conversation.
A conversation is in itself interactive as in general roles of speaker and listener are changing while caller and callee speak with each to exchange information \citet[\eg,][]{hopper_telephone_2002}.%\footnote{\citet{gueguin_evaluation_2008} presented a model describing speaker different phases in a telephone conversation.}
The interaction behavior of caller and callee, in fact, can affect the perceived quality for both parties \citep[\cf,][]{schoenenberg_why_2014}.

For the evaluation of perceived quality standardized methods have been developed to achieve a comparable interaction behavior in a conversation and thus limited the impact of noise due to differences in behavior.
The \acf{SCT} have been especially developed to simulate typical conversational tasks that caller and callee need to solve together \cite[\cf,][p. 76]{moller_assessment_2000}.
Here caller and callee need to exchange a defined set of information while a conversation structure is suggested.
The \acs{SCT} mimic a common situation in which the caller has a demand with specific requirements, which he tries to fulfill by initiating the telephone conversation and informing the callee about his needs.
Based upon this information the callee selects the appropriate option or information and presents this to the caller.
If the presented option or information fulfills the requirements of the caller and is confirmed by him, a second information transfer is initiated in which the callee provides the caller information, so that he can finally fulfill the requirement using the selected option.
The standardized \acs{SCT} \citep{itu-t_p.805:_2007} usually result in a conversation duration of \unit[3]{min} to \unit[7]{min}.

While conversations and the resulting perceived quality is affected by varying user behavior, it allows to study the whole range of degradations for speech telephony and also determine how those degradations \emph{affect} the usage behavior.
Furthermore, an active conversation allows to investigate the impact of degradations in a setting, in which a speech telephony service is actually used.

Beside the advantages the evaluation of perceived quality in an active conversation requires a large effort, because \emph{a)} a service with desired performance levels must be available and \emph{b)} the varying user behavior is an additional factor influencing in quality judgments.
%Social impact?

\subsubsection*{Third-party Listening}
Perceived quality of speech telephony can be assessed to a certain degree in a passive situation.
A subject listens to a conversation and thus not an actual part of this conversation.
This procedure is denoted as third-party listening \citep[\cf, ][p. 13]{itu-t_p.832:_2000}.
As the subject is not part of the conversation, his behavior cannot affect the conversation.\cite{skowro
This is, in fact, an uncommon situation as it cannot occur in a two-party situation, but is not a uncommon situation in multi-party conferencing \todo{REF JANTO, Kathrin oder so?}.

Elimination the influence of a subjects behavior on a conversation allows to use recordings of conversations and thus present the exact same material to multiple subjects.
If the degradations under considerations do not affect the behavior of the people participating in a conversation, the degradations can be inserted in a post-processing stage to the recordings of the conversation.
If recordings are used alone, no system is necessary that can provide required service performance in real-time.

As a subject only experiences a listening-only situation alone, the impact of degradations on the speaking phase cannot be assessed.
Another limitation is that the usage situation is different.
A passive observer is not forced to follow a conversation, if it is known to him beforehand that the content of the conversation is not relevant for him.
This can be avoided by applying a task, which requires a subject to follow the conversation.

If conversations are based upon \acs{SCT}, a note taking task can be applied.
Here the task is to write down all information exchanged between caller and callee, and vice versa.
As standardized \acs{SCT} \citep{itu-t_p.805:_2007} contain similar content, this task can be structured by asking content-specific questions like \emph{"What is the name of the caller/callee?"}, or \emph{"What does the callee want?"}.
%In a \ac{SCT} induced conversation those question are implicitly answered by caller and callee.

For the assessment of multi-episodic perceived quality, third-party listening has some advantages over two-party conversation although the usage situation is artificial.
Most important the task can be solved by one subject alone and thus eliminating the need of a conversation partner.
Furthermore, the same conversation with preprocessed degradations can be presented to multiple subjects in exactly the same manner including varying performance levels.
In this case also the length of a usage episode is known beforehand, allowing to investigate the impact episode duration on retrospective judgments.

\subsubsection{Entertainment Media Consumption: Audio and Audiovisual Services}
Telecommunication services for media consumption provide a unidirectional transmission of (multi-)media content to a user on his request.
This can be unimodal content like audio, or speech, as well as multi-modal content like audiovisual content.

A typical usage scenario is the provision of media content for entertainment purpose.
Services that provide content on-demand are denoted \ac{AoD} for audio-only and \ac{VoD} for audiovisual content.
Here a user can select from available content, the currently desired item, which is transmitted to him, in the best case, almost immediately.
%A common use case for media-on-demand services is the consumption of a series content-related parts like a TV series.

While the media selection procedure is often interactive, because a user can select the desired content, the actual media consumption only provides limited to no interactivity.
Former, a service might allow to pause, or seek whereas latter the only option for the user is to abort consumption at all.
In fact, a service that does not allow media selection but rather provides media in a defined order, this service can be considered non-interactive.

In fact, a third-party listening could be considered as a special case of media consumption, \ie, consumption of recorded two-party speech conversation.
However, third-party listening focuses on the simulation of multi-party conferencing in which the observer takes a passive part of the conversation whereas media consumption focus on the media consumption alone.
In difference to telephony, media consumption focuses on the consumption of pre-produced content.
This allows to use high-end recording equipment, adequate post-processing, and efficient but time consuming compression.
This, in general, limits the sources of severe degradations in general to the transmission as well as reproduction.
Recently, media-on-demand services have been widely deployed that adapt to the current network conditions to avoid stalling at a reduced service performance, \eg, limiting transmission bandwidth.
For a usage episode this can either happen while content is consumed resulting in performance fluctuations, or beforehand trying to achieve a constant service performance.

Media-on-demand services are well-suited for investigation multi-episodic perceived quality, because those can be setup in a non-interactive way and thus avoid impact of user behavior especially on duration of usage episodes.
Furthermore, media-on-demand services are nowadays widely used in mobile as well as in stationary usage situations. 

%\subsection{Web-browsing}

\section{Hypotheses}
Based upon prior work on retrospective experiences (\cf, \autoref{chap:03}) and quality assessment (\cf, \autoref{chap:04}), I derived \unit[7]{hypotheses} to provide information about the formation process.
Those hypotheses are presented in the following.

\subsection{\emph{Hypothesis}: Number of Consecutive \acl{LP} Episodes}
\begin{hypothesis}[\autoref{hypo:number}]\label{hypo:number}
Increasing the number of degraded episodes before a multi-episodic judgment decreases this judgment, because more \ac{LP} episodes and thus more time of \ac{LP} is experienced.
\end{hypothesis}

The presentation of \ac{LP} episode(s) is expected to result in a reduction in multi-episodic judgment compared to the presentation of those usage episode(s) in \ac{HP}.
When presenting all episodes in \ac{HP}, then the multi-episodic judgment should be sufficiently reflected by averaging all prior episodic judgments as predicted by \citet{moller_single-call_2011}.
The more \ac{LP} episode(s) are presented, the higher is the expected reduction until multi-episodic judgments reach the same level as the episodic judgments of \ac{LP} usage episodes, which is expected to be a lower boundary.
%For this hypothesis to be valid, no effect of \emph{duration neglect} over several usage episodes should be observed (\cf, \autoref{chap:03}).

\subsection{\emph{Hypothesis}: Position of \acl{LP} Episode(s)}
\begin{hypothesis}[\autoref{hypo:position}]\label{hypo:position}
Increasing the number of \ac{HP} episode(s), which follow one or more \ac{LP} episodes, results in a lower reduction of the multi-episodic judgment.
\end{hypothesis}

A \emph{recency effect} has been observed in sequential learning, retrospective judgments of experiences in general, and for perceived quality with one stimulus of varying performance (\cf, \autoref{chap:03}).
If an effect of recency affect multi-episodic QoE, episodes with close temporal proximity to this judgment must have a higher impact on it.
By varying the position of \ac{LP} episode(s) before this judgment, the existence of a recency effect can be observed.
If recency occurs, conditions that present \ac{LP} episode(s) closer to the multi-episodic judgment will have a lower score than those conditions that present more \ac{HP} episodes following the \ac{LP} episode(s).

\subsection{\emph{Hypothesis}: Consecutive vs. Non-consecutive \ac{LP} Episodes}
\begin{hypothesis}[\autoref{hypo:consecutive}]\label{hypo:consecutive}
The presentation of non-consecutive \ac{LP} episodes lead a higher reduction in multi-episodic judgments than the number as consecutive \ac{LP} episodes.
\end{hypothesis}

Presenting consecutive \ac{LP} episodes is expected to produce a higher multi-episodic judgment than the same number non-consecutive \ac{LP} episodes, because in the latter case the service performance varies more often.
It is thus not reliable and less predictable for the user than in the former case.

\subsection{\emph{Hypothesis}: Recovery after \ac{LP} Episodes}
\begin{hypothesis}[\autoref{hypo:recovery}]\label{hypo:recovery}
Presenting additional \ac{HP} episodes after a negatively affected multi-episodic judgment, increases the following multi-episodic judgment.
\end{hypothesis}

This hypothesis is similar to \autoref{hypo:position}, but focuses on the recovery of an negatively affected multi-episodic judgments.
%The presentation of \ac{LP} episodes is expected to reduce the following multi-episodic judgment.
Presenting after this judgment and before the next multi-episodic judgment only \ac{HP} episodes, should result in an increase of the last judgment.
If enough \ac{HP} episodes are presented, the last judgment is expected to reach the same level as if no \ac{LP} episodes were presented at all.

\subsection{\emph{Hypothesis}: Duration of \ac{LP} Episode}
\begin{hypothesis}[\autoref{hypo:duration}]\label{hypo:duration}
\ac{LP} episodes with a longer duration have a higher impact on multi-episodic perceived quality than shorter \ac{LP} episode, because the experienced duration the performance level \ac{LP} is longer.
\end{hypothesis}

For retrospective judgments of episodic experiences with variations a \emph{duration neglect} could be observed, \ie, the actual duration of a variations in an episodic experience is not reflected in the retrospective judgment (\cf, \autoref{chap:03}).
Such an effect must not necessarily occur for multi-episodic perceived quality.
In \autoref{hypo:number} the number of \ac{LP} episodes and the impact on multi-episodic judgment is investigated.
For an increasing number of \ac{LP} episodes, a longer overall duration of \ac{LP} is experienced.
This, however, leaves open, if the formation process of multi-episodic perceived quality relies \emph{a)} on the overall duration of \ac{LP} episode(s), \emph{b)} the number of \ac{LP} episode(s), or \emph{c)} both.
In fact, for near constant performance the actual duration has been found to have a constant positive but rather small effect on a retrospective judgments if the duration is longer than \unit[30]{sec} \citep[\cf,][]{frohlich_qoe_2012}.
However, the shift is constant suggesting an impact of usage situation rather than an encoding of duration into the episodic judgment.
In case of a) and c) episodic judgments alone would not contain all information for prediction of the multi-episodic judgment.\footnote{\citet[p. 2]{moller_single-call_2011} did not define the duration of \ac{LP} usage episodes and thus implicitly assumed a duration neglect in the formation process of multi-episodic perceived quality.}

The impact of the duration of \ac{LP} episodes on multi-episodic perceived quality can be investigated by comparing the multi-episodic judgment of at least two conditions with the same number and positions of \ac{LP} episode(s) with varying duration of the \ac{LP} episode(s).

\subsection{\emph{Hypothesis}: Strength of one \ac{LP} Episode}
\begin{hypothesis}[\autoref{hypo:strength}]\label{hypo:strength}
The \emph{lowest} experienced episodic performance has a higher impact on multi-episodic judgment.
\end{hypothesis}

The so-called peak-effect, which has been observed in retrospective assessment of episodic experiences, denotes the higher impact of the worst part of an experience on this judgment (\cf, \autoref{chap:03}).
As such an effect has been observed for perceived quality of one stimulus and also for one usage episode, and therefore might also affect the quality formation process of multi-episodic perceived quality.
It can be investigating by presenting more than two performance levels that result in different episodic judgments and investigate differences in multi-episodic judgments.

\subsection{\emph{Hypothesis}: Services are Judged Independent}
\begin{hypothesis}[\autoref{hypo:independent}]\label{hypo:independent}
The multi-episodic judgment for one service is not affected, if a second service is assessed with regard to multi-episodic QoE in same usage period.
\end{hypothesis}

The quality formation process is affected by expectations, knowledge, and prior experiences, which also change the encoding of the perceived quality into the judgment.
%In perceptual experiments order effects that one stimulus affect the judgments of a following stimulus are in general countered by randomization.
With regard to multi-episodic perceived quality, however, the use of another service might affect the multi-episodic judgment of the assessed service.
Studying multi-episodic perceived quality in one session alone such an effect can be prevent by using one service alone.
In a usage period covering several days, this can hardly be prevented.
It is therefore important to know, if the use of other services affect this multi-episodic judgment for the service under assessment.

%WEB + TV parallel
%In certain situations multiple services might be actively used in a concurrent manner.
%For example a person might watch a full-length movie on one device while using a second device.
%The second device could be used to acquire additional information about the movie, or communicate with other persons.
%If such a case occurs, the user must split his attention between the concurrently used services, which affects his perception.
%This might also affect formation process of multi-episodic perceived quality.

%\subsection*{\emph{H8}: Episodic Judgment} %optional!
%\begin{hypothesis}[H8]\label{hypo:judgment}
%Does the episodic quality judgment impact multi-episodic QoE???? AOD to be prepared! %TODO
%\end{hypothesis}

%TODO \section{Conclusion}
