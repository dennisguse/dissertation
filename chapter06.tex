\chapter{Towards Multi-episodic Perceived Quality}\label{chap:towards}
First work on multi-episodic perceived quality with the  defined-use method was conducted by \citet{moller_single-call_2011}.
Although only limited results were obtained with regard to multi-episodic perceived quality, the conducted experiment showed that this assessment method can be successfully applied.
Based upon this and prior work on effects of retrospective judgments, I designed and conducted a series of experiments to examine the formation process of multi-episodic perceived quality. 
Those experiments form the basis to implement a prediction model for judgments of multi-episodic perceived quality.
One implicit requirement of the defined-use method is that a between-design needs to be applied.
In the following, I first present details that form the basis for the experimental application of the defined-use method.
This includes judgments, performance levels, usage periods as well as used service types and applied tasks.
Then, I describe the hypotheses I am going to investigate for multi-episodic perceived quality.

\section{Aspects}

\subsection{Initial Experiences}
\citet{moller_single-call_2011} investigated multi-episodic perceived quality for a service, which provides first several episodes of highest performance.
Reduction in performance is only presented for later usage episodes.
This allows a user to familiarize with the highest performance level and his resulting perceived quality.
When reduced performance occurs, the user can compare his current experience with his prior experiences with the service.
This should avoid that the user compares his current experience, resulting from a reduced performance, to prior experiences with another services.
In the experiment by \citet{moller_single-call_2011} at least four episodes were presented with highest performance per condition.
For my experiments, I follow this concept.

In the experiment of \citet{moller_single-call_2011}, participants were not presented typical performance reductions that might occur for the service type.
This might lead to change in use of the scale over the usage period due to the encounter of degradations.
Highest performance episodes might be judged better after the reduced performance episodes have been experienced.
However, such an effect has not been observed in this experiment.
One reason for this might be the use of Skype and participants were required to be experienced with this service.
The impact of prior experiences with a service, and thus an effect on multi-episodic perceived quality, can be avoided by creating a service that is not known to participants beforehand.
An adaption of scale can be limited by presenting the range of degradations beforehand to participants.
In experiments on perceived quality, this is in general done by presenting and assessing short stimuli.
Although this might set expectations for the performance to be experienced, it gives participants a shared basis for their judgments and thus might reduce variance.
This is in the following denoted as training.
\citet{moller_single-call_2011} did not conduct a training, but relied on prior (pre-experiment) experiences of the participants.
Only the first two of the to be presented experiments (\E4{} and \E5{}) were conducted without a training.

\subsection{Judgments}
Perceived quality judgments are taken similar to \citet{moller_single-call_2011}.
For each usage episode the retrospective perceived quality is assessed.
This so-called \emph{episodic judgments} allows to determine, how the performance of a usage episode was perceived.
Those judgments are collected directly after finishing a usage episode.
%This ensures that the episodic judgment reflects the perceived quality as precise as possible.

The investigation of multi-episodic perceived quality will be based on the retrospective judgment, which is in the following denoted \emph{multi-episodic judgment}.
For this judgment a participant is requested to assess his \emph{perceived quality of all prior usage episodes with the service}.
It is assumed that difference between conditions, if those are different enough, manifest in consistent variations in the multi-episodic judgments.
Here, \emph{condition} refers to the presented order of usage episodes with defined performance level and their defined occurrence over the usage period.
Similar to episodic judgments, multi-episodic judgments are acquired after finishing a usage episode.
If a multi-episodic judgment is required, it is assessed after the episodic judgment.
This should prevent an influence on the episodic judgment due to the assessment of the multi-episodic perceived quality.
In fact, assessing multi-episodic perceived quality directly after the episodic judgment might increase the impact of this very episode on the multi-episodic judgment.
However, it is not yet known, if such an effect occurs and must be left for future work.

Following \citet{moller_single-call_2011}, the 7-point \acf{CCR} scale is used for the episodic judgments and multi-episodic judgments (\cf, \autoref{img:chap05:quality-scale}, \autopageref{img:chap05:quality-scale}).
The 7-point \ac{CCR} allows for more fine-grained judgments than the 5-point \ac{ACR} scale.
This might enable to observe small differences between conditions.
However, this might also be a source of unexplained variance as more rating possibilities are provided.
Using the same scale for both judgments enables a direct comparison between both judgments without requiring a conversion between scales.

\subsection{Performance Levels}
For the investigation of multi-episodic perceived quality, three performance levels are used.
Those are denoted as \acf{HP}, \acf{MP}, and \acf{LP}.
All three performance levels should be clearly distinguishable.
\ac{HP} denotes the highest performance and should lead to higher episodic judgments than \ac{MP}.
Both should be judged better than \ac{LP}.
In line with \citet{moller_single-call_2011}, performance levels are selected to provide no macroscopic performance fluctuations.
This avoids potential effects due to within-episodic fluctuations, because their impact on episodic judgments and multi-episodic judgments are not yet fully understood (\cf, \autoref{chap:04}).
The presentation of all three performance levels in one condition allows to investigate the existence of a peak-effect, which cannot be quantified using two performance levels alone (\cf, \autoref{chap:state-of-the-art}).
\ac{HP} and \ac{LP} in all conditions of an experiment allows to compare the episodic judgments between conditions.
This is useful to investigate the impact of the necessary between-subject design.
%In difference to \citet{moller_single-call_2011} in all conditions \ac{HP} as well as \ac{LP} are presented.

The results of \citet{moller_single-call_2011} show that episodic judgments are in line with the defined performance levels, because those show a clear difference of episodic judgments between \ac{HP} and \ac{LP}.
However, the impact on the multi-episodic judgments due to \ac{LP} usage episodes is very limited (\cf, \autoref{prior:moeller}).
This indicates that the selected performance level for \ac{LP} did produce degradations that were perceived and judged, but were not severe enough to produce a clear, observable effect on multi-episodic judgments.
Thus, \ac{LP} must be selected in such a way that degradations are \emph{severe} enough to produce an \emph{observable} effect on multi-episodic perceived quality.
However, successful task fulfillment must be guaranteed to avoid frustration due to task failure.

\subsection{Usage Periods}
\citet{moller_single-call_2011} applied the defined-use method in a usage period of 12~days. 
%This period has been selected as it is expected to be a typical period for service adaptation of new users. %TODO REF
%However, it is so far not known, if the length of the usage period affects the formation process of multi-episodic perceived quality.
Whereas \citet{moller_single-call_2011} focused on a usage period of multiple days, multi-episodic perceived quality also occurs in one session, if a session consists of multiple, distinct usage episodes.
Studying multi-episodic perceived quality in one session alone, allows to conduct experiments in a controlled laboratory environment. % similar to standardized experiments on sub-episodic as well as episodic perceived quality.
Typical experiments on perceived quality do not exceed \unit[90]{min} for one session to avoid an influence of fatigue.
In fact, limiting the usage period to such a short time-frame reduces the required effort and thus allows to study a higher number of conditions in detail.
Furthermore, the environment and equipment can be kept constant for all participants and thus should not affect the judgments.
Beside the reduced effort for the investigation of multi-episodic perceived quality, the findings form a meaningful starting point for investigate multi-episodic perceived quality over several days.
Three usage periods are investigated in this thesis: one~session, \unit[6]{days}, and~\unit[14]{days}.

\subsection{Service Types}
In this thesis two types of telecommunication services are investigated.
Here, services are of special interest that are frequently used and enable rather short but meaningful and self-contained usage episodes.
Different service types must be considered as it is not yet known, if the formation process of multi-episodic perceived quality is affected by the service type.
For each service type, a task must be selected in such a way that it results in an episodic experience.
In the following, I present the service types and tasks, which I will use in my investigation of multi-episodic perceived quality.

\subsubsection*{Speech Telephony}\label{method:sct}
Speech telephony services provide live communication between two or more remote parties for spoken interaction.
This is a well established and, in fact, classic telecommunication service.
The quality perception and underlying influence factors for speech telephony are well understood and standardized evaluation methods have been developed for evaluation of perceived quality \citep[\eg,][]{itu_handbook_1992}.
%In addition, first work on multi-episodic perceived quality was conducted with video telephony
%Furthermore, first work on multi-episodic perceived quality was conducted by \citet{moller_single-call_2011} for conversational system.

\paragraph*{Two-party Conversation}
Speech telephony is most often used for the communication between two remote parties, who engage in a conversation.
A telephony conversation is an interactive exchange of information with changing roles of speaker and listener between caller and callee \citep[][]{hopper_telephone_1992}.
The interaction behavior of caller and callee can affect the perceived quality for both parties \citep[\eg,][]{schoenenberg_why_2014, egger_it_2010}.

Methods have been developed to achieve a comparable interaction behavior in a conversation and thus limit the impact of different behavior.
Most prominent are the \acp{SCS}, where caller and callee need to solve a typical task together \citep[][p.~76]{moller_assessment_2000}.
Here, caller and callee need to exchange a defined set of information while a conversational structure is suggested.
A common situation is mimicked in which the caller has a demand with specific requirements, which he tries to fulfill by initiating the conversation and informing the callee about his demand.
Based upon this information, the callee selects an appropriate pre-defined option, or information and presents this to the caller.
If this fulfills the requirements of the caller, a second information transfer is initiated.
Here, the callee provides information to the caller, so that the caller can finally fulfill this requirement.
For this method, standardized scenarios are defined in \citet{itu-t_recommendation_p.805_subjective_2007}.
The standardized \acp{SCS} usually result in a conversation duration of 3~min to 7~min.

While a conversation and the resulting perceived quality is affected by varying user behavior, it allows to study the whole range of degradations for speech telephony.
Here, also degradations can be assessed that \emph{affect} the usage behavior. %REF?
Furthermore, an active conversation allows to investigate the impact of degradations in a setting, in which a speech telephony service is actually used.
Beside the advantages, the evaluation of perceived quality in an active conversation requires a large effort.
First, a service must be available that can provide desired performance.
This is especially problematic in non-laboratory environment (\cf, experiment \E4{}).
Second, variations in user behavior can affect quality perception and thus quality judgments.
This additional influence might be problematic for the investigation of multi-episodic perceived quality.

\paragraph*{Third-party Listening}
Perceived quality of speech telephony can be assessed  to a certain degree in a passive situation.
Here, a participant listens to a recorded conversation and thus is not an actual part of this conversation, \ie, his behavior cannot affect the conversation.
If a two-party conversation is used, this is denoted as third-party listening \citep[][p.~13]{itu-t_recommendation_p.832_subjective_2000}.
This is, in fact, an artificial situation, because it cannot occur in a two-party conversation.
Here, only a monologue of one conversation partner might occur as part of a conversation.
For multi-party conferencing, however, this is a likely situation.
Anyhow, using recordings of complete two-party conversations is here considered to be a usage episode.

The elimination of user behavior allows to use recordings of conversations and thus to present the exact same stimuli to multiple participants.
If degradations do not affect the behavior, \eg, Lombard speech \citep[][p.~161]{moller_assessment_2000}, those can be inserted in a post-processing.

A listening-only experiment has one major limitation beside the inability to assess the impact of degradations on the speaking phase \citep{gueguin_evaluation_2008}.
In fact, a passive observer is not even forced to follow a conversation, because he does not have to react to the content.
This can be avoided by applying a task, which requires to follow the conversation.
For conversations based upon \acp{SCS}, a note taking task can be applied.
In fact, for \ac{SCS} the actual task of caller and callee is to exchange a specific set of information, \ie, each one needs to answer specific questions.
Those implicit questions are similar for the standardized \acp{SCS}.
These questions can be used as a task in a listening-only situation while presenting recordings of \acp{SCS}.
This forces participants to follow conversation to successfully solve the task.
%those questions like \emph{"What is the name of the caller/callee?"}, or \emph{"What does the callee want?"}.
%In a \ac{SCS} induced conversation those question are implicitly answered by caller and callee.

For the assessment of multi-episodic perceived quality, third-party listening has some advantages over two-party conversation although the usage situation is artificial.
Most importantly, the task can be solved by a participant alone.
This eliminates the need of a conversation partner and the same conversation with preprocessed performance can be presented to multiple participants in exactly the same manner.
Here, even the duration of usage episodes can be defined beforehand without user behavior specific variations.
Also, the technical complexity is reduced, because neither live transmission nor processing is required.
%In this case also the length of a usage episode is known beforehand, allowing to investigate the impact episode duration on retrospective judgments.

\subsubsection*{Entertainment Media Consumption}
Telecommunication services for media consumption provide a unidirectional transmission of (multi-)media content to a user on his request.
This can be unimodal content like audio, or even speech, and multi-modal content like audiovisual content.

A typical usage scenario is the provision of media content for entertainment purpose.
Services that provide media content on-demand are denoted \acf{AoD} for audio-only content and \acf{VoD} for audiovisual content.
For such a service, a user can select from available content the currently desired one, which is then transmitted to him.
%A common use case for media-on-demand services is the consumption of a series content-related parts like a TV series.
While the media selection procedure is often interactive, the actual media consumption only provides limited interactivity.
Here, a user might be allowed to pause, seek or abort the consumption.
In fact, the interactivity of an on-demand service can be completely limited by presenting predefined content and not allow in-presentation interaction.
%In fact, a third-party listening could be considered as a special case of media consumption, \ie, consumption of recorded two-party speech conversation.
%However, third-party listening focuses on the simulation of multi-party conferencing in which the observer takes a passive part of the conversation whereas media consumption focus on the media consumption alone.
In difference to telephony, media consumption focuses on the consumption of pre-produced content.
This allows to use high-end recording equipment and adequate post-processing. %, and efficient but time consuming compression.
This limits the sources of severe degradations in general to transmission and reproduction.
%Recently, media-on-demand services have been widely deployed that adapt to the current network conditions to avoid stalling at a reduced service performance, \eg, limiting transmission bandwidth.
%For a usage episode this can either happen while content is consumed resulting in performance fluctuations, or beforehand trying to achieve a constant service performance.

Media-on-demand services are well-suited for investigating multi-episodic perceived quality.
First, such a service can be set up in a non-interactive way and thus avoids impact of user behavior.
Second, content can be provided that is interesting for participants and thus limit the effort for them.
And also stimuli can be pre-processed completely and thus limiting technical complexity.

\section{Hypotheses}
Based upon prior work on retrospective experiences (\cf, \autoref{chap:03}) and quality assessment (\cf, \autoref{chap:04}), I derived 7~hypotheses to derive knowledge about the formation process.
The experimental investigation of those hypotheses will be used for the implementation of a prediction model for multi-episodic perceived quality in \autoref{chap:modeling}.
The major goal here is to investigate, if a weighted average model is sufficient, or a more sophisticated model is required.
The hypotheses are presented in the following.

\subsection[\autoref{hypo:number}: Number of Consecutive \acs{LP} Episodes]{\emph{Hypothesis}: Number of Consecutive \acs{LP} Episodes}
\begin{hypothesis}[\autoref{hypo:number}]\label{hypo:number}
Increasing the number of \ac{LP} episodes before a multi-episodic judgment decreases this judgment.
\end{hypothesis}

The presentation of \ac{LP} episode(s) is expected to result in a reduction in multi-episodic judgment compared to the presentation of those usage episode(s) in \ac{HP}.
When presenting all episodes in \ac{HP}, then the multi-episodic judgment should be sufficiently reflected by averaging all prior episodic judgments \citep[\cf,][]{moller_single-call_2011}.
The more \ac{LP} episode(s) are presented, the higher is the expected reduction of multi-episodic judgments.
Here, a lower boundary is expected to be set by the episodic judgments for \ac{LP} episodes.

Based upon this hypothesis also the impact of \ac{LP} episodes can be quantified.
This is important for the parametrization of a prediction model.


\subsection[\autoref{hypo:position}: Position of \acs{LP} Episode(s)]{\emph{Hypothesis}: Position of \acs{LP} Episode(s)}
\begin{hypothesis}[\autoref{hypo:position}]\label{hypo:position}
Increasing the number of \ac{HP} episode(s), which follow one or more \ac{LP} episode(s), results in a lower reduction of the multi-episodic judgment.
\end{hypothesis}

A \emph{recency effect} has been observed in sequential learning and retrospective judgments of episodic experiences.
With regard to perceived quality an effect of recency could be observed in stimuli with macroscopic performance fluctuations, if a stimuli is long enough.
If an effect of recency occurs for multi-episodic perceived quality, then usage episodes with close temporal proximity to a multi-episodic judgment have a higher impact than episodes that occurred earlier.
By varying the position of a defined number of \ac{LP} episode(s) before a multi-episodic judgment, the existence of a recency effect can be investigated.
If a recency effect occurs, conditions that present \ac{LP} episode(s) closer to the multi-episodic judgment will result in a lower multi-episodic judgment than those conditions that present more \ac{HP} episodes following the \ac{LP} episode(s).

\subsection[\autoref{hypo:consecutive}: Non-Consecutive vs. Consecutive \acs{LP} Episodes]{\emph{Hypothesis}: Non-Consecutive vs. Consecutive \acs{LP} Episodes}
\begin{hypothesis}[\autoref{hypo:consecutive}]\label{hypo:consecutive}
The presentation of non-consecutive \ac{LP} episodes leads to a higher reduction of multi-episodic judgments than presenting the number of \ac{LP} episodes consecutively.
\end{hypothesis}

With this hypothesis it is investigated, if the number of performance changes between episodes affects a multi-episodic judgment.
Here, it is expected that increasing the number of changes results in higher reduction, because the performance is less predictable for participants.
This can be investigated by presenting the same number of \ac{LP} episodes either consecutively or non-consecutively before a multi-episodic judgment.
In fact, if the impact is small or even not existing, then it might be hidden by a recency effect.
This is because the position before the multi-episodic judgment must be varied, \ie, in the non-consecutive case \ac{LP} must be presented closer to this judgment than in the consecutive case.

\subsection[\autoref{hypo:strength}: Strength of Degradation]{\emph{Hypothesis}: Strength of Degradation}
\begin{hypothesis}[\autoref{hypo:strength}]\label{hypo:strength}
The lowest experienced episodic performance has an increased impact on a multi-episodic judgment whereas less severe degradations are less important.
\end{hypothesis}

The so-called peak effect, which has been observed in retrospective assessment of episodic experiences, denotes a higher impact of the worst part of an experience on a retrospective judgment (\cf, \autoref{chap:03}).
Such an effect has been observed for perceived quality of one stimulus and for one usage episode affecting the quality formation process (\cf, \autoref{chap:04}).
The existence of such an an effect has not been investigated for multi-episodic perceived quality.
It can be investigated by presenting more than two performance levels.
In fact, those performance levels must result in a different perceived quality.
For this, the performance level \ac{MP} is used in addition to \ac{LP} and \ac{HP}.
If a peak-effect exists, then a condition, which presents some usage episode in \ac{MP}, should be judged similar to a condition, which presents those episode in \ac{HP} and all other with equal performance level.

%In fact, a peak-effect has found been very useful for implementation of prediction models for perceived quality of macroscopic performance fluctuations.

\subsection[\autoref{hypo:recovery}: Recovery after \acs{LP} Episodes]{\emph{Hypothesis}: Recovery after \acs{LP} Episodes}
\begin{hypothesis}[\autoref{hypo:recovery}]\label{hypo:recovery}
Presenting additional \ac{HP} episodes after a negatively affected multi-episodic judgment, results in an increase of the following multi-episodic judgment.
\end{hypothesis}

This hypothesis is similar to \autoref{hypo:position}, but focuses on the recovery after a negatively affected multi-episodic judgment.
Recovery can be investigated by presenting only \ac{HP} episodes after an affected multi-episodic judgment.
This should result in an increase of the final multi-episodic judgment.
If enough \ac{HP} episodes are presented, the final judgment should reach a similar level as if no \ac{LP} episodes were presented at all.
%The expected increase might be the result of a recency effect, or the increased number of \ac{HP} episodes.

\subsection[\autoref{hypo:duration}: Duration of a \acs{LP} Episode]{\emph{Hypothesis}: Duration of a \acs{LP} Episode}
\begin{hypothesis}[\autoref{hypo:duration}]\label{hypo:duration}
\ac{LP} episodes with a much longer duration result in higher reduction of multi-episodic judgments than shorter \ac{LP} episodes.
\end{hypothesis}

For retrospective judgments of episodic experiences with variations a \emph{duration neglect} could be observed (\cf, \autoref{chap:03}).
However, such an effect must not necessarily occur for multi-episodic perceived quality.
In fact, in \autoref{hypo:number} the number of \ac{LP} episodes and the impact on multi-episodic judgment is investigated.
For an increasing number of \ac{LP} episodes, a longer overall duration of \ac{LP} is experienced.
This, however, leaves open, if the formation process of multi-episodic perceived quality relies \emph{a)} on the overall duration of \ac{LP} episode(s), \emph{b)} the number of \ac{LP} episode(s), or \emph{c)} both.
%In fact, for constant performance the actual duration has been found to have a constant positive but rather small effect on a retrospective judgments, if the duration is longer than \unit[30]{sec} \citep[\cf,][]{frohlich_qoe_2012}.
%However, the shift is constant suggesting an impact of usage situation rather than an encoding of duration into the episodic judgment.
In case of a) and c), episodic judgments alone would not contain all information for the prediction of the multi-episodic judgment.
%\citet[p.~2]{moller_single-call_2011} did not define the duration of \ac{LP} usage episodes and thus implicitly assumed a duration neglect in the formation process of multi-episodic perceived quality.
A duration neglect can be investigated by comparing the results of two conditions that are similar with regard to performance level while varying the duration of the \ac{LP} episode(s).

\subsection[\autoref{hypo:independent}: Services are Judged Independent]{\emph{Hypothesis}: Services are Judged Independent}
\begin{hypothesis}[\autoref{hypo:independent}]\label{hypo:independent}
The multi-episodic judgment for one service is not affected by the presentation of a second service in the same usage period.
\end{hypothesis}

Multi-episodic perceived quality can only be assessed in retrospective by evaluating prior, recallable experiences of a service and form the judgment.
If in the same usage period multiple, similar services have been used this might be problematic.
The multi-episodic judgment of a service could be affected by the presence of other service(s).
Here, an assessor might fail to attribute perceived quality correctly to a service, or other service(s) might affect expectations.
For investigating multi-episodic perceived quality in one session alone, this is not problematic, because the exposure to other services can be controlled.
However, for a usage period covering several days, this can hardly be prevented.
It is therefore important to understand, if the use of other services affects this multi-episodic judgment of a service.

\section{Conclusion}
In this chapter, I first presented the service types, tasks, and performance levels that will be used for the investigation of multi-episodic perceived quality.
Then, I presented my hypotheses of the quality formation process for multi-episodic judgments.
Those form the basis for the experimental investigation.
With regard to modeling most important seems to be \autoref{hypo:number}, because this hypothesis should result in a large effect.
In the following chapter, the experiments on multi-episodic perceived quality in one session are presented.
In \autoref{chap:field}, I present the experiments covering a usage period of multiple days. 
