\chapter{Towards Multi-episodic Perceived Quality}\label{chap:towards}
Initial work on multi\-/episodic perceived quality with the  defined-use method was conducted by \citet{moller_single-call_2011}.
Although only limited results were obtained with regard to multi\-/episodic perceived quality, the results of this experiment show that this assessment method can be successfully applied in field experiments.
Based on this and prior work on effects of retrospective judgments, I designed and conducted a series of experiments to examine the formation process of multi\-/episodic perceived quality. 
These experiments form the basis for the development of prediction models for multi\-/episodic judgments.
%One implicit requirement of the defined-use method is that a between-subject design must be applied.
In the following, I first present the here considered aspects for the application of the defined-use method.
This includes judgments, performance levels, usage periods, service types, and tasks.
Subsequently, I describe the hypotheses I am going to investigate about the formation process of multi\-/episodic perceived quality.

\section{Aspects}

\subsection{Initial Experiences}
\citet{moller_single-call_2011} investigated multi\-/episodic perceived quality for a service which provides the first episodes with highest performance.
Reduction in performance was only presented for later usage episodes.
This allows a participant to familiarize himself with the highest performance level and his resulting perceived quality.
When reduced performance occurs, the user can compare his current experience with his prior experiences with the service.
This should avoid that a participant needs to compare his current experience, resulting from a reduced performance, to prior experiences with another services to derive a judgment.
In the experiment by \citet{moller_single-call_2011}, at least four episodes were presented with the highest performance level.
For my experiments, I follow this approach.

In the experiment of \citet{moller_single-call_2011}, no anchor stimuli were presented to participants before conducting the multi\-/episodic part of this experiment.
Thus, each participant could only rely on his individual prior experiences for his judgments of the episodic and multi\-/episodic perceived quality.
This might lead to an effect that the highest performance is judged to be better after reduced performance episodes have been experienced, because the latter might lead to an adjustment of the internal reference.
However, such an effect has not been observed in this experiment.
One reason for this might be the use of Skype and participants being required to have prior experiences with this service.
The impact of prior experiences with a service, and thus potential, unknown effects on the formation process of multi\-/episodic perceived quality, can be avoided by creating a \emph{new} service that is not known to participants in advance.
Besides creating a new service, the evaluation of perceived quality of short anchor stimuli is conducted with each participant before presenting the multi\-/episodic condition.
Although this might set expectations for the performance to be experienced, it gives participants a shared basis for their judgments.
The perceived quality assessment of anchor stimuli is denoted in the following as \emph{training}.

%An adaptation of scale can be limited by presenting the range of degradations beforehand to participants.
%In experiments on perceived quality, this is in general done by presenting and assessing short stimuli.
%\citet{moller_single-call_2011} did not conduct a training, but relied on prior (pre-experiment) experiences of the participants.
%Only the first two of the to be presented experiments (\E4{} and \E5{}) were conducted without a training.

\subsection{Judgments}
Perceived quality judgments are taken similarly to \citet{moller_single-call_2011}.
For each usage episode, the retrospective perceived quality is assessed.
These so-called \emph{episodic judgments} allow the determination of how the performance of each usage episode was perceived.
Episodic judgments are taken directly after finishing a usage episode.
%This ensures that the episodic judgment reflects the perceived quality as precise as possible.

The investigation of multi\-/episodic perceived quality will be based on retrospective judgments, which are in the following denoted as \emph{multi\-/episodic judgments}.
For a multi\-/episodic judgment, a participant is requested to assess his \emph{perceived quality of all prior usage episodes with the service}.
It is assumed that differences in performance between multi\-/episodic conditions manifest as consistent differences in the multi\-/episodic judgments.
Here, \emph{multi\-/episodic condition} refers to the presentation order of usage episodes with defined performance level and their defined occurrence over the usage period.
Similar to episodic judgments, multi\-/episodic judgments are acquired after finishing a usage episode.
If a multi\-/episodic judgment is required, it is assessed after the episodic judgment.
This should prevent an influence on the episodic judgment due to the assessment of the multi\-/episodic perceived quality.
In fact, assessing multi\-/episodic perceived quality directly after the episodic judgment might increase the impact of this very episode on the multi\-/episodic judgment.
It is not yet known, however, if such an effect occurs and must be left for future work.

Following \citet{moller_single-call_2011}, the 7\=/point \acf{CoCR} scale is used for the episodic judgments and multi\-/episodic judgments (\cf{} \autoref{img:chap05:quality-scale}, \autopageref{img:chap05:quality-scale}).
The 7\=/point \ac{CoCR} allows for more fine-grained judgments than the 5\=/point \ac{ACR} scale.
This might enable the observation of small differences between conditions but might also introduce small deviations, as more rating possibilities are provided.
Using the same scale for both judgments enables a direct comparison between both judgments without requiring a conversion between scales.

\subsection{Performance Levels}
For the investigation of multi\-/episodic perceived quality, three performance levels are used.
These are denoted as \acf{HP}, \acf{MP}, and \acf{LP}.
All three performance levels should be clearly distinguishable with regard to perceived quality.
\ac{HP} denotes the highest performance and should lead to higher episodic judgments than \ac{MP}.
Both should be judged to be better than \ac{LP}.
In line with \citet{moller_single-call_2011}, performance levels are selected, so almost no macroscopic fluctuations of perceived quality occur.
This avoids potential effects due to within-episodic fluctuations, as their impact on episodic judgments and multi\-/episodic judgments are not yet fully understood (\cf{} \autoref{chap:04}).
The presentation of all three performance levels in one condition allows the investigation of the existence of a peak effect, which cannot be quantified using two performance levels alone (\cf{} \autoref{chap:state-of-the-art}).
\ac{HP} and \ac{LP} were presented in all conditions per experiment, so the judgments of the conditions are directly comparable and are not affected potentially by differences in worst and best performance levels.
This is useful for the investigation of the impact of the necessary between-subject design.
%In difference to \citet{moller_single-call_2011} in all conditions \ac{HP} as well as \ac{LP} are presented.

The results of \citet{moller_single-call_2011} show that episodic judgments are in line with the defined performance levels, because these show a clear difference of episodic judgments between \ac{HP} and \ac{LP}.
However, the impact on multi\-/episodic judgments due to \ac{LP} usage episodes was very limited (\cf{} \autoref{prior:moeller}).
This indicates that the selected performance level for \ac{LP} did, in fact, produce degradations that were perceived and judged, but were not severe enough to produce a clear, observable effect on multi\-/episodic judgments.
Thus, \ac{LP} must be selected in such a way that degradations are \emph{severe} enough to produce an \emph{observable} effect on multi\-/episodic perceived quality.
As the inability to fulfill a task due to overly severe degradations and the impact on perceived quality due to frustration is so far unknown, successful task fulfillment is required throughout this thesis for all applied performance levels.

\subsection{Usage Periods}
\citet{moller_single-call_2011} applied the defined-use method in a usage period of \unit[12]{days}. 
%This period has been selected as it is expected to be a typical period for service adaptation of new users. %TODO REF
%However, it is so far not known, if the length of the usage period affects the formation process of multi\-/episodic perceived quality.
Whereas \citet{moller_single-call_2011} focused on a usage period of multiple days, multi\-/episodic perceived quality also occurs in one session if a session consists of multiple usage episodes.
Studying multi\-/episodic perceived quality in one session alone allows one to conduct experiments in a controlled laboratory environment. % similar to standardized experiments on sub-episodic as well as episodic perceived quality.
Typical experiments on perceived quality do not exceed \unit[90]{min} to avoid an influence of fatigue.
In fact, limiting the usage period to such a short time frame reduces the required effort and thus allows the investigation of a higher number of multi\-/episodic conditions in detail.
Furthermore, the environment and equipment can be kept constant for all participants and thus must not be considered as confounding factors.

In addition to the reduced effort for the investigation of multi\-/episodic perceived quality, the findings form a meaningful starting point for the investigation of multi\-/episodic perceived quality over several days.
Three usage periods are investigated in this thesis: one~session, \unit[6]{days}, and~\unit[14]{days}.

\subsection{Service Types}
In this thesis, two types of telecommunication services are used.
Here, services are of special interest that are frequently used and enable rather short usage episodes.
Different service types must be considered, as it is not yet known if the formation process of multi\-/episodic perceived quality is affected by the service type.
For each service type, a generic task needed to be selected in such a way that solving one such task in one interaction results in a usage episode.
In the following, I present the service types and tasks which I will use for the investigation of multi\-/episodic perceived quality.

\subsubsection*{Speech Telephony}\label{method:sct}
Speech telephony services provide live communication between two or more remote parties for spoken interaction.
This is a well established and, in fact, classic telecommunication service.
The quality perception and underlying influence factors for speech telephony are well understood, and standardized evaluation methods have been developed for the evaluation of perceived quality \citep[\eg,][]{itu_handbook_1992}.
%In addition, first work on multi\-/episodic perceived quality was conducted with video telephony
%Furthermore, first work on multi\-/episodic perceived quality was conducted by \citet{moller_single-call_2011} for conversational system.

\paragraph*{Two-party Conversation}
Speech telephony is most often used for communication between two remote parties who engage in a conversation.
A telephony conversation is an interactive exchange of information with changing roles of speaker and listener between caller and callee \citep[][]{hopper_telephone_1992}.
The interaction behavior of caller and callee can affect the perceived quality for both parties \citep[\eg,][]{schoenenberg_why_2014, egger_it_2010}.

Methods have been developed to achieve a comparable interaction behavior in a conversation and thus limit the impact of different behavior on perceived quality.
Most prominent are the \acfp{SCS} in which caller and callee need to solve a typical two\=/party telephony task together \citep[][p.\,76]{moller_assessment_2000}.
Here, caller and callee need to exchange a defined set of information while a conversational structure is suggested.
A common situation is mimicked in which the caller has a demand with specific requirements, which he tries to fulfill by initiating the conversation and informing the callee about his demand.
Based on this information, the callee selects an appropriate predefined option or information and presents this to the caller.
If this fulfills the requirements of the caller, a second information transfer is initiated.
Here, the callee provides information to the caller, so that the caller can finally fulfill his requirement.
For this method, standardized scenarios are defined in the \citet{itu-t_recommendation_p.805_subjective_2007}.
The standardized \acp{SCS} usually result in a conversational duration of \unit[3]{min} to \unit[7]{min}.
This allows the investigation of the whole range of degradations for speech telephony.
Here, also those degradations that \emph{affect} the usage behavior can be evaluated. %REF?
Furthermore, an active conversation allows to investigate the impact of degradations in a setting in which speech telephony is actually used.
Besides the advantages, the evaluation of perceived quality in an active conversation requires a great effort.
First, a service/system must be available that can provide the desired performance levels.
This is especially a problematic in a field experiment.
Second, variations in user behavior can affect quality perception and thus quality judgments.
This might be problematic for the investigation of multi\-/episodic perceived quality.

\paragraph*{Third-party Listening}
Perceived quality of speech telephony can be assessed  to a certain degree in a passive situation.
Here, a participant listens to a recorded conversation and thus is not an actual part of this conversation, \ie, his behavior cannot affect the conversation.
If the recording of a two\=/party conversation is presented, this is denoted as third\=/party listening \citep[][p.\,13]{itu-t_recommendation_p.832_subjective_2000}.
This is, in fact, an artificial situation, as it cannot occur in a two\=/party conversation.
Here, only a monologue of one conversational partner might occur as part of a conversation.
For multi\=/party conferencing, however, this is a likely situation.
In either case, using recordings of a complete two\=/party conversation is here considered to represent a usage episode if it contains a meaningful conversation.

The elimination of user behavior allows the use of recordings of conversations and thus the presentation of the exact same stimuli to multiple participants.
If the desired degradations do not affect the behavior of caller and callee, the degradations can even be inserted via post\=/processing the recordings.

A listening-only experiment has one major limitation besides the inability to assess the impact of degradations on the speaking phase \citep{gueguin_evaluation_2008}.
In fact, a passive observer is not forced to follow a conversation, as he does not have to react to a conversational partner.
This can be avoided by applying a task which requires following the conversation.
For conversations based on \acp{SCS}, a note-taking task can be applied.
Indeed, for \ac{SCS} the actual task of caller and callee is to exchange a specific set of information, \ie, each one needs to answer specific questions.
These implicit questions are similar for the standardized \acp{SCS}.
These questions can be used as a task in a listening-only situation while presenting recordings of \acp{SCS}.
This forces participants to follow the conversation to successfully solve the task.
%those questions such as \emph{"What is the name of the caller/callee?"}, or \emph{"What does the callee want?"}.
%In a \ac{SCS} induced conversation those question are implicitly answered by caller and callee.

For the assessment of multi\-/episodic perceived quality, third\=/party listening has some advantages over two\=/party conversation although the usage situation is artificial.
Most importantly, the task can be solved by a participant alone.
This eliminates the need for a conversational partner, and the same stimulus (\ie, a conversation including degradations) can be presented to multiple participants.
Here, even the duration of usage episodes can be defined beforehand.
Also, the technical complexity is reduced, as neither live transmission nor live processing is required.
%In this case also the length of a usage episode is known beforehand, allowing to investigate the impact episode duration on retrospective judgments.

\subsubsection*{Entertainment Media Consumption}
Telecommunication services for media consumption provide a unidirectional transmission of (multi-)media content to a user on his request.
This can be unimodal content, such as audio, and multi-modal content, such as audiovisual content.

A typical usage scenario is the provision of media content for entertainment purposes.
Services that provide media content on\=/demand are denoted as \acf{AoD} for audio-only content and as \acf{VoD} for audiovisual content.
For such a service, a user can select from the available content that item which he currently desires.
The desired item is then transmitted to him.
%A common use case for media\=/on\=/demand services is the consumption of a series content-related parts such as a TV series.
While the media selection procedure is often interactive, actual media consumption provides only limited interactivity.
Here, a user might be allowed to pause, seek, or abort the consumption.
In fact, the interactivity of an on\=/demand service can be completely limited by presenting predefined content and not allow in\=/presentation interaction.
%In fact, a third-party listening could be considered as a special case of media consumption, \ie, consumption of recorded two-party speech conversation.
%However, third-party listening focuses on the simulation of multi-party conferencing in which the observer takes a passive part of the conversation whereas media consumption focus on the media consumption alone.
In contrast to telephony, media entertainment focuses on the consumption of pre\=/produced content.
This allows the use of high-end recording equipment and adequate post\=/processing.
Thus, limiting in general the sources of severe degradations to the transmission and the reproduction.
%Recently, media-on-demand services have been widely deployed that adapt to the current network conditions to avoid stalling at a reduced service performance, \eg, limiting transmission bandwidth.
%For a usage episode this can either happen while content is consumed resulting in performance fluctuations, or beforehand trying to achieve a constant service performance.

Media-on-demand services are well-suited for investigating multi\-/episodic perceived quality.
First, such a service can be set up in a non\-/interactive way and thus avoiding the impact of varying user behavior.
Second, content can be provided that is interesting for participants and thus motivating them to participate in such an experiment.
Besides this, the content can be pre\=/processed in advance, limiting technical complexity.

\section{Hypotheses}
Based on prior work on retrospective experiences (\cf{} \autoref{chap:03}) and quality assessment (\cf{} \autoref{chap:04}), I developed 7~hypotheses to investigate the formation process of multi-episodic perceived quality.
The experimental investigation of these hypotheses will be used for the implementation of a prediction model for multi\-/episodic judgments in \autoref{chap:modeling}.
The major goal here is to investigate if an average model is sufficient or a more sophisticated model is required due to observable effects.
The hypotheses are presented in the following.

\subsection[H1: Number of Consecutive \acs{LP} Episodes]{\emph{Hypothesis}: Number of Consecutive \acs{LP} Episodes}
\begin{hypothesis}[\autoref{hypo:number}]\label{hypo:number}
Increasing the number of \ac{LP} episodes before a multi\-/episodic judgment decreases this judgment.
\end{hypothesis}

The presentation of \ac{LP} episode(s) is expected to result in a reduction in multi\-/episodic judgments compared to the presentation of these usage episode(s) in \ac{HP}.
When presenting all episodes in \ac{HP}, the multi\-/episodic judgment should be sufficiently reflected by averaging all prior episodic judgments \citep[\cf{}][]{moller_single-call_2011}.
The more \ac{LP} episode(s) are presented, the higher should be the expected reduction of multi\-/episodic judgments.
Here, a lower boundary is expected to be set by the episodic judgments for \ac{LP} episodes.
Based on this hypothesis, the impact of \ac{LP} episodes can also be quantified, which is important for the implementation of a prediction model.


\subsection[H2: Position of \acs{LP} Episode(s)]{\emph{Hypothesis}: Position of \acs{LP} Episode(s)}
\begin{hypothesis}[\autoref{hypo:position}]\label{hypo:position}
%Increasing the number of \ac{HP} episode(s), which follow one or more \ac{LP} episode(s), results in a lower reduction of the multi\-/episodic judgment.
The more \ac{HP} episodes are presented directly before a multi\-/episodic judgment, the lower is the negative impact of earlier presented \ac{LP} episodes.
%The negative impact of \ac{LP} episode(s) on a following multi\-/episodic judgment is reduced, the more \ac{HP} episode(s) are presented directly before this judgment.
\end{hypothesis}

A \emph{recency effect} has been observed in sequential learning and retrospective judgments of episodic experiences.
With regard to perceived quality, an effect of recency could be observed in stimuli with macroscopic performance fluctuations if stimuli were long enough.
If an effect of recency occurs for multi\-/episodic perceived quality, then usage episodes with close temporal proximity to a multi\-/episodic judgment have a higher impact than episodes that occurred earlier.
By varying the position of \ac{LP} episode(s) before a multi\-/episodic judgment, the existence of a recency effect can be investigated.
If a recency effect occurs, conditions that present \ac{LP} episode(s) closer to the multi\-/episodic judgment will result in a lower multi\-/episodic judgment than those conditions that present more \ac{HP} episodes following the same number of \ac{LP} episode(s).

\subsection[H3: Non-Consecutive vs. Consecutive \acs{LP} Episodes]{\emph{Hypothesis}: Non-Consecutive vs. Consecutive \acs{LP} Episodes}
\begin{hypothesis}[\autoref{hypo:consecutive}]\label{hypo:consecutive}
The presentation of non-consecutive \ac{LP} episodes leads to a higher reduction of multi\-/episodic judgments than presenting the same number of \ac{LP} episodes consecutively.
\end{hypothesis}

Here, it is investigated whether the number of performance changes between episodes affects multi\-/episodic judgments.
It is expected that increasing the number of changes results in a higher reduction, as the performance is less predictable for participants.
This can be investigated by presenting the same number of \ac{LP} episodes either consecutively or non-consecutively before a multi\-/episodic judgment.
If the effect is small or non-existing, then it might be hidden by a recency effect.
Indeed, keeping the number of \ac{HP} and \ac{LP} episodes constant, \ac{LP} episodes must be separated by \ac{HP} episode(s) in non-consecutive cases.
Thus, some \ac{LP} episodes are presented earlier than in consecutive cases.

\subsection[H4: Strength of Degradation]{\emph{Hypothesis}: Strength of Degradation}
\begin{hypothesis}[\autoref{hypo:strength}]\label{hypo:strength}
The lowest experienced episodic performance has an increased impact on multi\-/episodic judgments, whereas less severe degradations are less important.
\end{hypothesis}

The so-called peak effect, which has been observed in retrospective assessment of episodic experiences, denotes a higher impact of the worst part of an experience and a lower impact of less displeasing parts of an experience on a retrospective judgment of this experience (\cf{} \autoref{chap:03}).
Such an effect has been observed for perceived quality affecting the quality formation process (\cf{} \autoref{chap:04}).
The existence of such an an effect has not been investigated for multi\-/episodic perceived quality.
It can be investigated by presenting more than two performance levels and analyzing their impact on multi\-/episodic judgments.
In fact, these performance levels must result in a different perceived quality.
If a peak effect exists, then the episode(s) presented with the worst performance level should have a higher impact on the multi\-/episodic judgment than episodes which provide a better episodic experience.
This is investigated here by introducing a third performance level denoted as \ac{MP} in addition to \ac{HP} and \ac{LP}.
Here, \ac{MP} must be selected, so it achieves a better perceived quality than \ac{LP} but is inferior to \ac{HP}.
This allows the determination of the impact of \ac{MP} episodes on multi\-/episodic judgments compared to \ac{LP} episode(s) and thus the investigation of the existence of a peak effect.

%If a peak effect exists, then a multi\-/episodic condition, which presents some usage episode in \ac{MP}, should be judged similar to a condition, which presents those episode in \ac{HP} and all other with equal performance level.

%In fact, a peak effect has found been very useful for implementation of prediction models for perceived quality of macroscopic performance fluctuations.

\subsection[H5: Recovery after \acs{LP} Episodes]{\emph{Hypothesis}: Recovery after \acs{LP} Episodes}
\begin{hypothesis}[\autoref{hypo:recovery}]\label{hypo:recovery}
Presenting additional \ac{HP} episodes after a negatively affected multi\-/episodic judgment results in an increase of the following multi\-/episodic judgment.
\end{hypothesis}

This hypothesis is similar to \autoref{hypo:position}, but focuses on the recovery after a negatively affected multi\-/episodic judgment.
Recovery can be investigated by presenting only \ac{HP} episodes after the negatively affected multi\-/episodic judgment.
This should result in an increase of the additional multi\-/episodic judgment.
If enough \ac{HP} episodes are presented, this final judgment should reach a similar level as if no \ac{LP} episodes were presented at all.
%The expected increase might be the result of a recency effect, or the increased number of \ac{HP} episodes.

\subsection[H6: Duration of a \acs{LP} Episode]{\emph{Hypothesis}: Duration of a \acs{LP} Episode}
\begin{hypothesis}[\autoref{hypo:duration}]\label{hypo:duration}
\ac{LP} episodes with a much longer duration result in higher reduction of multi\-/episodic judgments than shorter \ac{LP} episodes.
\end{hypothesis}

For retrospective judgments of episodic experiences with macroscopic fluctuations, a \emph{duration neglect} could be observed (\cf{} \autoref{chap:03}).
However, such an effect must not necessarily occur for multi\-/episodic perceived quality.
In fact, in \autoref{hypo:number}, the number of \ac{LP} episodes and the impact on multi\-/episodic judgments is investigated.
Thus, for an increasing number of \ac{LP} episodes, a longer overall duration of \ac{LP} is experienced.
This, however, leaves open to conjecture whether the formation process of multi\-/episodic perceived quality relies \emph{a)} on the overall duration of \ac{LP} episode(s), \emph{b)} the number of \ac{LP} episode(s), or \emph{c)} both.
%In fact, for constant performance the actual duration has been found to have a constant positive but rather small effect on a retrospective judgments, if the duration is longer than \unit[30]{s} \citep[\cf{}][]{frohlich_qoe_2012}.
%However, the shift is constant suggesting an impact of usage situation rather than an encoding of duration into the episodic judgment.
In case of a) and c), episodic judgments alone would not contain all the required information for the prediction of multi\-/episodic judgments.
%\citet[p.\,2]{moller_single-call_2011} did not define the duration of \ac{LP} usage episodes and thus implicitly assumed a duration neglect in the formation process of multi\-/episodic perceived quality.
A duration neglect can be investigated by comparing the results of conditions that are similar with regard to performance level while varying the duration of the \ac{LP} episode(s).

\subsection[H7: Services are Judged Independent]{\emph{Hypothesis}: Services are Judged Independent}
\begin{hypothesis}[\autoref{hypo:independent}]\label{hypo:independent}
The multi\-/episodic judgment for one service is not affected by the presentation of a second service in the same usage period.
\end{hypothesis}

Multi-episodic perceived quality can only be assessed in retrospect by evaluating prior, recallable experiences of a service and derive the judgment.
If in the same usage period multiple services have been used, this might be problematic.
The multi\-/episodic judgment of a service could be affected by the presence of other service(s).
Here, an assessor might fail to attribute perceived quality correctly to a service, or the other service(s) might affect expectations.
For investigating multi\-/episodic perceived quality in one session alone, this is not problematic, as the exposure to other services within the multi\-/episodic conditions can be controlled.
However, for a usage period spanning several days, this can hardly be prevented.
It is therefore important to understand if the use of other services affects the multi\-/episodic judgment of a service to be judged.

\section{Conclusion}
In this chapter, I first presented the aspects that might affect the formation process of multi\-/episodic perceived quality.
Subsequently, I presented the service types, the tasks, and the performance levels that will be used for the investigation of multi\-/episodic perceived quality.
Then, I presented my hypotheses of the quality formation process for multi\-/episodic judgments.
These form the basis of the experimental investigation.
With regard to modeling, most important seems to be \autoref{hypo:number}, as this hypothesis is expected to result in a large effect.
All other hypotheses provide knowledge about edge cases of the formation process of multi-episodic perceived quality.
In the following chapter, the experiments on multi\-/episodic perceived quality in one session are presented.
In \autoref{chap:field}, I present the experiments with usage periods of multiple days. 
