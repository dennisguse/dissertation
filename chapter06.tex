\chapter{Towards Multi-episodic Perceived Quality}\label{chap:towards}
\begin{chapter-abstract}
Here I describe my approach towards understanding multi-episodic QoE.
Based upon \cite{moller_single-call_2011}, I approach multi-episodic perceived quality in two periods: one session and multiple days.

What do I need for modeling?
Where are limits?
Only do limited assumptions! (we don't know anything)

Find anomalies that are not reflected by "averaging"
Hypotheses in detail

Experiments share the same stuff:
%Important point: argue that must degradations must be severe!
\end{chapter-abstract}

First work on multi-episodic perceived quality with a defined-use methodology was conducted by \cite{moller_single-call_2011}.
Although only limited results were obtained, the conducted experiment showed that this assessment methodology can be successfully applied.
I follow the work of \cite{moller_single-call_2011} by investigating multi-episodic perceived quality with a defined-use methodology.
The experiment of \cite{moller_single-call_2011} indicated that formation process of multi-episodic perceived quality is unlikely to be an averaging of the perceived quality of all prior usage episodes.
Based upon this and prior work on temporal effects on retrospective judgment, I conducted a series of experiments to examine the formation process in detail to derive potential model components and create a prediction model based upon episodic perceived quality judgments.

\section{Approach}
\subsection{Prior Experiences}
\cite{moller_single-call_2011} investigated multi-episodic perceived quality for the case, in which a service delivers highest performance in the beginning but latter reduced performance is sometimes only provided.
This allows a person to experience the service under the best performance, and experience it.
When reduced performance occurs this person can compare his prior experiences with said service.
This avoids that the person compares reduced performance of a usage episode to prior experiences, which do not result of the experiences said service.
\cite{moller_single-call_2011} applied for this four episodes (\ie, two days).
In my experiments I followed this approach.

\subsection{Judgments}
The investigation of multi-episodic perceived quality will be based on the retrospective judgment, which is in the following denoted \emph{multi-episodic judgment}.
For this judgment a person is requested to judge his \emph{perceived quality of all prior usage episodes with the service}.\todo{REF to Appendix!}
It is assumed that the experiences of different conditions, if those are different enough, manifest in differences in multi-episodic judgments.

In addition, episodic perceived quality judgments are collected, which are in the following denoted as \emph{episodic judgments}.
This judgment is collected directly after finishing a usage episode.
This ensures that the judgment reflects the perceived quality as precise as possible.
The episodic judgment allows to determine, how the performance of a usage episode was perceived.

The multi-episodic judgment is always taken after the episodic judgment.
It is assumed that this order prevents an effect of the multi-episodic judgment on the episodic judgment.
For both judgments the same scale is used.
This enables a direct mapping from episodic judgments to a multi-episodic judgment without converting the scale and makes the judgments directly comparable.

Following \cite{moller_single-call_2011} the 7-point \ac{ACR} scale is applied for all judgments about perceived quality (\cf, \autoref{img:chap05:quality-scale}, \autopageref{img:chap05:quality-scale}).
Using a continuous scale over the discrete 5-point \ac{ACR} scales allows for more fine-grained judgments between two categories.
This enables to observe small differences between different conditions, but can be a source of noise, also in-subject, as more rating possibilities are provided.

\subsection{Performance Levels}
\cite{moller_single-call_2011} applied per condition two performance levels \acf{HP}, and \acf{LP}.
\ac{HP} denotes the highest performance that should lead a higher episodic judgment than\ac{LP}, which denotes the lowest performance.
For both performance levels \cite{moller_single-call_2011} limited the overall maximum transmission rate only.
This performance limitation was selected as a near \emph{constant} quality per episode desired. %"stable quality judgment"
This avoids effects due to in-episodic quality fluctuations on episodic judgments and multi-episodic judgments as those are not yet fully understood (\cf, \autoref{chap:04}).

The results of \cite{moller_single-call_2011} show that episodic judgments are in line with the defined performance levels as those show a clear difference between \ac{HP} and \ac{LP} episodic judgments.
However, the impact on the multi-episodic judgments due to \ac{LP} usage episodes is very limited (\cf \autoref{chap:04}).
This indicates that the selected performance level for \ac{LP} did produce degradations that were perceived, but were not severe enough to produce a clear, observable effect on multi-episodic judgments.
Thus, \ac{LP} must be selected in such a way that degradations are \emph{severe} enough to produce an observable effect on multi-episodic perceived quality, but still provide sufficient performance to use the service to successfully solve defined tasks.

Per condition two performance levels alone do not allow to investigate the existence of a peak effect (\cf \autoref{chap:state-of-the-art}).
This effect states that the worst part of an experience, has a higher impact on a retrospective judgment.
For this investigation at least a third performance level is required that provides a higher episodic quality than \ac{LP}, but lower than \ac{HP}.
This performance level is in the following denoted as \acf{MP}.

\section{Hypotheses}
Based upon prior work on retrospective experiences (\cf, \autoref{chap:03}) and quality assessment (\cf, \autoref{chap:04}), I derived \unit[6]{hypotheses} to provide information about the formation process.
In the following the \unit[6]{hypotheses} are presented.

\subsection*{\emph{Hypothesis 1 (H1)}: Number of Consecutive \ac{LP} Episodes}
\begin{hypothesis}[H1]
Increasing the number of degraded episodes before a multi-episodic judgment decreases this judgment, because more \ac{LP} episodes and thus more time of \ac{LP} is experienced.
\end{hypothesis}

The presentation of \ac{LP} episode(s) is expected to result in a reduction in multi-episodic judgment compared to presenting the same usage episode(s) in \ac{HP}.
Presenting all episodes in \ac{HP} the multi-episodic judgment should be sufficiently reflected by averaging all prior episodic judgments as found by \cite{moller_single-call_2011}.
The more \ac{LP} episode(s) are presented, the higher is the expected reduction until the multi-episodic judgment reaches the same level as the episodic judgments of \ac{LP} usage episodes, which is considered a lower boundary.

For this hypothesis to be valid, no effect of \emph{duration neglect} should be observed (\cf, \autoref{chap:03}).

%Length of degradation (Conversation and Listening): yes, similar: saturation

\subsection*{\emph{Hypothesis 2 (H2)}: Position of \ac{LP} Episode(s)}
\begin{hypothesis}[H2]
Increasing the number of \ac{HP} episode(s), which follow one or more \ac{LP} episodes, results a lower decrease of the multi-episodic judgment.
\end{hypothesis}

%Position (Conversation and Listening): yes/no, found for conversation

\subsection*{\emph{Hypothesis 3 (H3)}: Recovery after \ac{LP} Episodes}
\begin{hypothesis}[H3]
Presenting additional \ac{HP} episodes after a negatively affected multi-episodic judgment judgment, increases the following multi-episodic judgment.
\end{hypothesis}

%Telephony only

\subsection*{\emph{Hypothesis 4 (H4)}: Duration of \ac{LP} Episode}
\begin{hypothesis}[H5]
Usage episodes with a longer duration have a higher impact on multi-episodic perceived quality than shorter usage episode, because more time the experience the is longer.
\end{hypothesis}
%Length of degraded episode (Extension of Study 2): none
%Aod only

\subsection*{\emph{Hypothesis 5 (H5)}: Strength of one \ac{LP} Episode}
\begin{hypothesis}[H5]
The \emph{lowest} experienced episodic performance has a higher impact on multi-episodic judgment than smaller degradations.
\end{hypothesis}
%Strength of degradation (Conversation): yes, supports peak

\subsection*{\emph{Hypothesis 6 (H6)}: Different Services are Independent}
\begin{hypothesis}[H6]
The multi-episodic judgment is not affect, if multiple services are used.
\end{hypothesis}

%\subsection*{\emph{H7}: Episodic Judgment} %optional!
%Does the episodic quality judgment impact multi-episodic QoE???? AOD to be prepared! %TODO

\section{Usage periods}
Initial work on multi-episodic perceived quality with \emph{defined use} focused on a usage period over several days (\cf, \autoref{chap:state-of-the-art}). 
This period has been expected to allow new users of a service to determine its future use \citep[\cf,][]{moller_single-call_2011}.
However, it is so far unknown, if the length of usage period affects the formation process of multi-episodic perceived quality.
Whereas \cite{moller_single-call_2011} focused on a usage period of multiple days, multi-episodic perceived quality also occurs in one session that is divided into several distinct usage episodes.
This limits the duration of usage period to avoid influence factors like fatigue affect the formation process.
Typical experiments on perceived quality do not exceed \unit[90]{min}\todo{REF} to avoid such influence, but if a longer duration are required such experiments are split into multiple sessions.
Studying multi-episodic perceived quality in one session alone allows to conduct the experiment under controlled laboratory conditions similar to standardized experiments on perceived quality.
This avoids potential sources of noise for example due to differences in the presentation environment over several participants.
In fact, reducing the usage period to such a short time-frame compared to multiple days, allows to study a greater number on conditions in detail.
Results on multi-episodic perceived quality in one session can then be used as a starting point to investigate multi-episodic perceived quality over several days.

\section{Service Types}

\subsection{Tasks}

\subsection{Service-specific Performance Levels}

\section{Overview on Conducted Experiments}

%\section{Scales}
