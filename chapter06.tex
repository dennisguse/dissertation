\chapter{Towards Multi-episodic Perceived Quality}\label{chap:towards}

\begin{chapter-abstract}
Here I describe my approach towards understanding multi-episodic QoE.
Based upon \cite{moller_single-call_2011}, I approach multi-episodic perceived quality in two periods: one session and multiple days.

What do I need for modeling?
Where are limits?
Only do limited assumptions! (we don't know anything)

Find anomalies that are not reflected by "averaging"
Hypotheses in detail

Experiments share the same stuff:
%Important point: argue that must degradations must be severe!
\end{chapter-abstract}

First work on multi-episodic perceived quality with a defined-use methodology was conducted by \cite{moller_single-call_2011}.
Although only limited results were obtained, the conducted experiment showed that this assessment methodology can be successfully applied.
I follow the work of \cite{moller_single-call_2011} by investigating multi-episodic perceived quality with a defined-use methodology.
The experiment of \cite{moller_single-call_2011} indicated that formation process of multi-episodic perceived quality is unlikely to be an averaging of the perceived quality of all prior usage episodes.
Based upon this and prior work on temporal effects on retrospective judgment, I conducted a series of experiments to examine the formation process in detail to derive potential model components and create a prediction model based upon episodic perceived quality judgments.

\section{Approach}
\subsection{Prior Experiences}
\cite{moller_single-call_2011} investigated multi-episodic perceived quality for the case, in which a service delivers highest performance in the beginning but latter reduced performance is sometimes only provided.
This allows a person to experience the service under the best performance, and experience it.
When reduced performance occurs this person can compare his prior experiences with said service.
This avoids that the person compares reduced performance of a usage episode to prior experiences, which do not result of the experiences said service.
\cite{moller_single-call_2011} applied for this four episodes (\ie, two days).
In my experiments I followed this approach.

\subsection{Judgments}
The investigation of multi-episodic perceived quality will be based on the retrospective judgment, which is in the following denoted \emph{multi-episodic judgment}.
For this judgment a person is requested to judge his \emph{perceived quality of all prior usage episodes with the service}.\todo{REF to Appendix!}
It is assumed that the experiences of different conditions, if those are different enough, manifest in differences in multi-episodic judgments.

In addition, episodic perceived quality judgments are collected, which are in the following denoted as \emph{episodic judgments}.
This judgment is collected directly after finishing a usage episode.
This ensures that the judgment reflects the perceived quality as precise as possible.
The episodic judgment allows to determine, how the performance of a usage episode was perceived.

The multi-episodic judgment is always taken after the episodic judgment.
It is assumed that this order prevents an effect of the multi-episodic judgment on the episodic judgment.
For both judgments the same scale is used.
This enables a direct mapping from episodic judgments to a multi-episodic judgment without converting the scale and makes the judgments directly comparable.

Following \cite{moller_single-call_2011} the 7-point \ac{ACR} scale is applied for all judgments about perceived quality (\cf, \autoref{img:chap05:quality-scale}, \autopageref{img:chap05:quality-scale}).
Using a continuous scale over the discrete 5-point \ac{ACR} scales allows for more fine-grained judgments between two categories.
This enables to observe small differences between different conditions, but can be a source of noise, also in-subject, as more rating possibilities are provided.

\subsection{Performance Levels}
\cite{moller_single-call_2011} applied per condition two performance levels \acf{HP}, and \acf{LP}.
\ac{HP} denotes the highest performance that should lead a higher episodic judgment than\ac{LP}, which denotes the lowest performance.
For both performance levels \cite{moller_single-call_2011} limited the overall maximum transmission rate only.
This performance limitation was selected as a near \emph{constant} quality per episode desired. %"stable quality judgment"
This avoids effects due to in-episodic quality fluctuations on episodic judgments and multi-episodic judgments as those are not yet fully understood (\cf, \autoref{chap:04}).

The results of \cite{moller_single-call_2011} show that episodic judgments are in line with the defined performance levels as those show a clear difference between \ac{HP} and \ac{LP} episodic judgments.
However, the impact on the multi-episodic judgments due to \ac{LP} usage episodes is very limited (\cf \autoref{chap:04}).
This indicates that the selected performance level for \ac{LP} did produce degradations that were perceived, but were not severe enough to produce a clear, observable effect on multi-episodic judgments.
Thus, \ac{LP} must be selected in such a way that degradations are \emph{severe} enough to produce an observable effect on multi-episodic perceived quality, but still provide sufficient performance to use the service to successfully solve defined tasks.

Per condition two performance levels alone do not allow to investigate the existence of a peak effect (\cf \autoref{chap:state-of-the-art}).
This effect states that the worst part of an experience, has a higher impact on a retrospective judgment.
For this investigation at least a third performance level is required that provides a higher episodic quality than \ac{LP}, but lower than \ac{HP}.
This performance level is in the following denoted as \acf{MP}.

\section{Usage Periods}
Initial work on multi-episodic perceived quality with \emph{defined use} focused on a usage period over several days (\cf, \autoref{chap:state-of-the-art}). 
This period has been expected to allow new users of a service to determine its future use \citep[\cf,][]{moller_single-call_2011}.
However, it is so far unknown, if the length of usage period affects the formation process of multi-episodic perceived quality.

Whereas \cite{moller_single-call_2011} focused on a usage period of multiple days, multi-episodic perceived quality also occurs in one session that is divided into several distinct usage episodes.
This limits the duration of usage period to avoid influence factors like fatigue affect the formation process.
Typical experiments on perceived quality do not exceed \unit[90]{min}\todo{REF} to avoid such influence, but if a longer duration are required such experiments are split into multiple sessions.
Studying multi-episodic perceived quality in one session alone allows to conduct the experiment under controlled laboratory conditions similar to standardized experiments on perceived quality.
This avoids potential sources of noise for example due to differences in the presentation environment over several participants.
In fact, reducing the usage period to such a short time-frame compared to multiple days, allows to study a greater number on conditions in detail.
Results on multi-episodic perceived quality in one session can then be used as a starting point to investigate multi-episodic perceived quality over several days.

\section{Hypotheses}
Based upon prior work on retrospective experiences (\cf, \autoref{chap:03}) and quality assessment (\cf, \autoref{chap:04}), I derived \unit[6]{hypotheses} to provide information about the formation process.
In the following the \unit[6]{hypotheses} are presented.

\subsection*{\emph{Hypothesis 1 (H1)}: Number of Consecutive \acl{LP} Episodes}
\begin{hypothesis}[H1]\label{hypo:number}
Increasing the number of degraded episodes before a multi-episodic judgment decreases this judgment, because more \ac{LP} episodes and thus more time of \ac{LP} is experienced.
\end{hypothesis}

The presentation of \ac{LP} episode(s) is expected to result in a reduction in multi-episodic judgment compared to the presentation of those usage episode(s) in \ac{HP}.
Presenting all episodes in \ac{HP} the multi-episodic judgment should be sufficiently reflected by averaging all prior episodic judgments as found by \cite{moller_single-call_2011}.
The more \ac{LP} episode(s) are presented, the higher is the reduction is expected until the multi-episodic judgment reaches the same level as the episodic judgments of \ac{LP} usage episodes, which is considered a lower boundary.

For this hypothesis to be valid, no effect of \emph{duration neglect} should be observed (\cf, \autoref{chap:03}).
%Length of degradation (Conversation and Listening): yes, similar: saturation

\subsection*{\emph{Hypothesis 2 (H2)}: Position of \acl{LP} Episode(s)}
\begin{hypothesis}[H2]\label{hypo:position}
Increasing the number of \ac{HP} episode(s), which follow one or more \ac{LP} episodes, results in a decreased reduction of the multi-episodic judgment.
\end{hypothesis}

A \emph{recency effect} has been observed in sequential learning, and retrospective judgments of experiences in general as well as for perceived quality with one stimulus with varying performance (\cf, \autoref{chap:03}).
If an effect of recency can be observed in a multi-episodic judgment, usage episodes with close temporal proximity to this judgment must have a higher impact.
By varying the position of \ac{LP} episode(s) before this judgment, the existence of recency can be observed.
If recency occurs, conditions that present \ac{LP} episode(s) closer to the multi-episodic judgment will achieve a lower score.

%Position (Conversation and Listening): yes/no, found for conversation

\subsection*{\emph{Hypothesis 3 (H3)}: Recovery after \acl{LP} Episodes}
\begin{hypothesis}[H3]\label{hypo:recovery}
Presenting additional \ac{HP} episodes after a negatively affected multi-episodic judgment, increases the following multi-episodic judgment.
\end{hypothesis}

This hypothesis is similar to H2, but focuses on the recovery of multi-episodic judgments.
The presentation of \ac{LP} episodes is expected to reduce the following multi-episodic judgment.
Presenting after this judgment and before the next multi-episodic judgment only \ac{HP} episodes, should result in an increase of the last judgment.
If enough \ac{HP} episodes are presented, the last judgment is expected to reach the same level as if no \ac{LP} episodes were presented at all.

%Telephony only

\subsection*{\emph{Hypothesis 4 (H4)}: Duration of \acl{LP} Episode}
\begin{hypothesis}[H4]\label{hypo:duration}
\ac{LP} episodes with a longer duration have a higher impact on multi-episodic perceived quality than shorter \ac{LP} episode, because the experience of \ac{LP} is longer.
\end{hypothesis}

For retrospective judgments of episodic experiences with variation a \emph{duration neglect} could be observed, \ie, the actual duration of a variations in an episodic experience is not reflected in the retrospective judgment.
Such an effect must not necessarily occur for multi-episodic perceived quality.
In \autoref{hypo:number} the number of \ac{LP} episodes and the impact on multi-episodic judgment is investigated.
For an increasing number of \ac{LP} episodes, a longer overall duration of \ac{LP} is experienced.
This, however, leaves open, if the formation process of multi-episodic perceived quality relies \emph{a)} on the overall duration of \ac{LP} episode(s), \emph{b)} the number of \ac{LP} episode(s), or \emph{c)} both.
In case of a) episodic judgments alone does contain not all information encoded for prediction of the multi-episodic judgment.\footnote{\cite[p. 2]{moller_single-call_2011} did not define the duration of \ac{LP} usage episodes and thus implicitly assumed a duration neglect in the formation process of multi-episodic perceived quality.}

The impact of the duration of \ac{LP} episodes on multi-episodic perceived quality can be investigated by comparing the multi-episodic judgment of at least two conditions with the same number and positions of \ac{LP} episode(s) with varying duration of \ac{LP} episodes.

%AOD only: Length of degraded episode (Extension of Study 2): none

\subsection*{\emph{Hypothesis 5 (H5)}: Strength of one \acl{LP} Episode}
\begin{hypothesis}[H5]\label{hypo:strength}
The \emph{lowest} experienced episodic performance has a higher impact on multi-episodic judgment.
\end{hypothesis}

The so-called peak-effect, which has been observed in retrospective assessment of episodic experiences, denotes the higher impact of the worst part of an experience on this judgment (\cf, \autoref{chap:03}).
As such an effect has been observed for perceived quality of one stimulus and also for one usage episode, and therefore might also affect the quality formation process of multi-episodic perceived quality.
%Strength of degradation (Conversation): yes, supports peak

\subsection*{\emph{Hypothesis 6 (H6)}: Services are Judged Independent}
\begin{hypothesis}[H6]\label{hypo:services}
The multi-episodic judgment for one service is not affected, if a second service is assessed with regard to multi-episodic judgment in same usage period.
\end{hypothesis}

The quality formation process is affected by expectations, knowledge, and prior experiences, which also change the encoding of the perceived quality into the judgment.
In perceptual experiments order effects that one stimulus affect the judgments of a following stimulus are countered by randomization.
With regard to multi-episodic perceived quality, however, the use of another service might affect the multi-episodic judgment of the assessed service.
Studying multi-episodic perceived quality in one session alone such an effect can be prevent by using one service alone.
In a usage period covering several days, this can hardly be prevented.
It is therefore important to know, if how the use of other services affect this multi-episodic judgment for the service under assessment.

%WEB + TV parallel
%In certain situations multiple services might be actively used in a concurrent manner.
%For example a person might watch a full-length movie on one device while using a second device.
%The second device could be used to acquire additional information about the movie, or communicate with other persons.
%If such a case occurs, the user must split his attention between the concurrently used services, which affects his perception.
%This might also affect formation process of multi-episodic perceived quality.

%\subsection*{\emph{H7}: Episodic Judgment} %optional!
%\begin{hypothesis}[H7]\label{hypo:judgment}
%Does the episodic quality judgment impact multi-episodic QoE???? AOD to be prepared! %TODO
%\end{hypothesis}

\section{Conditions}
The presented conditions can be assessed in 
%Measurement after 6th usage episode OR DAY

\begin{table}[h]
 \centering
 \begin{tabulary}{\textwidth}{C|C||C|C|C|C|C|C|}
 Condition & \multicolumn{7}{c|}{Episodic Performance}        \\
           & 1-3	& 4           & 5           & 6           & 7  & 8  & 9 \\
 \midrule
 1         & HP 	& \textbf{LP} & HP          & HP          & -  & -  & - \\
 \hline
 2a        & HP 	& HP          & \textbf{LP} & HP          & -  & -  & - \\
 \hline
 2b        & HP 	& HP          & \textbf{LP}, long & HP     & -  & -  & - \\
 \hline
 3         & HP 	& HP          & HP          & \textbf{LP} & -  & -  & - \\
 \hline
 4         & HP 	& \textbf{LP} & \textbf{LP} & HP          & -  & -  & - \\
 \hline
 5a        & HP 	& HP          & \textbf{LP} & \textbf{LP} & -  & -  & - \\
 \hline
 5b        & HP 	& HP          & \textbf{LP} & \textbf{LP} & HP & HP & HP \\
 \hline
 6         & HP 	& \textbf{LP} & \textbf{LP} & \textbf{LP} & -  & -  & - \\
 \hline
 7         & HP 	& HP          & \textbf{LP} & \emph{MP}   & HP & HP & HP \\
 \end{tabulary}
 \caption{Overview of all conditions with the episodic performance of all usages episodes and showing which conditions are compared for each of the three hypotheses.
 Non-HP episodes are in bold (\ac{LP}) and italic (\ac{MP}).}
 \label{tab:chap06:hypothesesComparison}
\end{table}



\section{Service Types}
With regard to multi-episodic perceived quality for telecommunication services, those service types are of higher interest that are widely and also frequently used in daily life.
In the following the different services types are presented, which are in the experiments applied to investigate multi-episodic perceived quality in detail.
Different service types must be considered, because it is not known, if the formation process of multi-episodic perceived quality is affected by the service type, the service type specific interaction, and the to be solved task.

For multi-episodic perceived quality tasks should be selected in such way that those are meaningful enough to result in an episodic experience.
%TODO USAGE EPISODE!!!!

\subsection{Telephony}
Speech telephony services provide the live communication of between two or more remote parties for spoken interaction.
This is a well established and, in fact, classical service for telecommunication providers.
In fact, the quality perception and underlying influence factors for speech telephony are well understood and standardized evaluation methods have been developed for evaluation of perceived quality.

\subsubsection{Two-party Conversation}
Speech telephony is most often used for the communication between two remote parties, who engage in an conversation.
A conversation is in itself interactive as in general roles of speaker and listener are changing while caller and callee speak with each to exchange information.
The interaction behavior of caller and callee, in fact, can affect the perceived quality for both parties.

For the evaluation of perceived quality standardized methods have been developed to achieve a comparable interaction behavior in a conversation and thus limited the impact of noise due to differences in behavior.

The \acf{SCT} have been especially developed to simulate typical conversational tasks that caller and callee need to solve together \cite[\cf,][p. 76]{moller_assessment_2000}.
Here caller and callee need to exchange a defined set of information while a conversation structure is suggested.
The \ac{SCT} mimic a common situation in which the caller has a demand with specific requirements, which he tries to fulfill by initiating the telephone conversation and informing the callee about his needs.
Based upon this information the callee selects the appropriate option or information and presents this to the caller.
If the presented option or information fulfills the requirements of the caller and is confirmed by him, a second information transfer is initiated in which the callee provides the caller information, so that he can finally fulfill the requirement using the selected option.
The standardized \acs{SCT} \citep{itu-t_p.805:_2007} should result in a conversation duration of \unit[3]{min} to \unit[7]{min}.

While conversations and the resulting perceived quality is affected by varying user behavior, it allows to study the whole range of degradations for speech telephony and also determine how those degradations \emph{affect} the usage behavior.
Furthermore, an active conversation allows to investigate the impact of degradations in a setting, in which a speech telephony service is actually used.

Beside the advantages the evaluation of perceived quality in an active conversation requires a large effort, because \emph{a)} a service with desired performance levels must be available, \emph{b)} requires multiple participants at the same time and \emph{c)} the users behavior is an additional factor of noise in quality judgments.
%Social impact?

\subsubsection{Third-party Listening}
Perceived quality of speech telephony can be assessed in a passive situation to a certain degree.
A subject listens to a conversation and thus not an actual part of the conversation.
This procedure is denoted as third-party listening \citep[\cf, ][p. 13]{itu-t_p.832:_2000}.
As the subject is not part of the conversation, his behavior cannot affect the conversation.
This is, in fact, an uncommon situation as it cannot occur in a two-party situation, but is not an uncommon situation in multi-party conferencing \todo{REF JANTO, Kathrin oder so?}.

Elimination the influence of a subjects behavior on a conversation allows to use recordings of conversations and thus present exactly the same material to multiple subjects.
If the degradations under considerations do not affect the behavior of the people participating in a conversation, the degradations can be inserted after recording the conversation.
If recordings are used alone no system is necessary that can provide required service performance live.

As a subject only experiences a listening-only situation alone, the impact of degradations on the speaking phase cannot be assessed.
Another limitation is that the usage situation is different.
A passive observer is not forced to follow a conversation, if it is known to him beforehand that content of the conversation is not relevant for him.
This can be avoided by applying a task that requires that subjects follow the content of a conversation.

If conversations are based upon \acs{SCT}, a note taking task can be applied.
Here the task is to write down information exchanged between caller and callee, and vice versa.
As standardized \acs{SCT} \citep{itu-t_p.805:_2007} contain similar content, this task can be structured by asking content-specific questions like \emph{"What is the name of the caller/callee?"}, or \emph{"What does the callee want?"}.
In a \ac{SCT} induced conversation those question are implicitly answered by caller and callee.

For the assessment of multi-episodic perceived quality, third-party listening has some advantages over two-party conversation although the usage situation is artificial.
Most important the task can be solved by one subject alone and thus eliminating the need of a conversation partner.
Furthermore, the same conversation with preprocessed degradations can be presented to multiple subjects in exactly the same manner.
In this case also the length of a usage episode is known beforehand allowing to investigate the impact episode duration on multi-episodic judgment.

\subsection{Media Consumption: Audio and Audiovisual}


%passive

%passive
%\subsection{Web-browsing}
%interactive


\chapter{Conducted Experiments}
\section{Performance Levels}

%Describe each used service and usage situation (mobile vs. PC vs. [optional] living room)
%Describe tasks per service and requirements

\section{Overview on Studies}
%Put table with studies here!?

\begin{table}[h]
	\begin{tabular}{|c|c|c|c|c|}
	Identifier	& Service type 			& Task 									& Usage Period	& Hypotheses \\
	\hline
	S1L			& Telephony				& Two-party Conversation with \ac{SCT}	& One session	& H1, H2, H3, H5 \\
	S1F			& Telephony				& Two-party Conversation with \ac{SCT}	& One week		& H1 \\
	S2			& Telephony				& Third-party Listening					& One session	& H1, H2 \\ 
	S3L			& Audio Entertainment	& -										& One session	& H1, H2, H\\
	S3F			& Audio Entertainment	& -										& One week		& \\
	S4			& Audio Entertainment and Video Entertainment & - 				& One session	& H6 \\
	
	
	\end{tabular}
\end{table}
