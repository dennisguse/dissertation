\chapter{Towards Multi-episodic Perceived Quality}\label{chap:towards}
First work on multi-episodic perceived quality with a defined-use methodology was conducted by \citet{moller_single-call_2011}.
Although only limited results were obtained with regard to multi-episodic perceived quality, the conducted experiment showed that this assessment methodology can be successfully applied.
Based upon this and prior work on temporal effects on retrospective judgments, I designed and conducted a series of experiments to examine the formation process of multi-episodic perceived quality in detail. 
In addition, to investigate potential effects the results of those experiments are then used to develop potential prediction models based upon episodic perceived quality judgments.

In the following I first present details that form the basis for the application of the defined-use method.
This includes judgments, performance levels, usage periods as well as used service types and applied tasks.
Then I describe the hypotheses I am going to investigate for multi-episodic perceived quality.

\section{Aspects}

\subsection{Initial Experiences}
\citet{moller_single-call_2011} investigated multi-episodic perceived quality for the case, in which a service delivers highest performance for the very first usage episodes and latter episodes are potentially presented with reduced performance.
This allows a person to experience the service under the best performance, and familiarize with this performance level and the service characteristics.
When reduced performance occurs this person can compare his current experience with his prior experiences with said service.
This avoids that the person compares the current experience resulting from reduced performance to prior experiences that do not result of experiences with said service.
In the experiment by \citet{moller_single-call_2011} at least four episodes were presented with highest performance.
For my experiments, I follow this concept.

In difference to \citet{moller_single-call_2011} before conducting the multi-episodic assessment, participants need to assess typical degradations for each presented service type.
This presentation of short stimuli allows participants to familiarize with potentially occurring degradations rather than be surprised about variations in performance.
Although this might set expectations for to be experienced performance parameters, this can be used by participants as a basis for their judgments and thus might reduce variance due to prior knowledge.

\subsection{Judgments}
Episodic perceived quality judgments are collected, which are in the following denoted as \emph{episodic judgments}.
The episodic judgment allows to determine, how the performance of a usage episode was perceived.
Those judgments are collected directly after finishing a usage episode.
%This ensures that the episodic judgment reflects the perceived quality as precise as possible.

The investigation of multi-episodic perceived quality will be based on the retrospective judgment, which is in the following denoted \emph{multi-episodic judgment}.
For this judgment a person is requested to assess his \emph{perceived quality of all prior usage episodes with the service}.
It is assumed that different conditions, if those are different enough, manifest in consistent variations in the multi-episodic judgments.
Here \emph{condition} refers to a presented order of episodes with defined performance level and their defined occurrence over the usage period.

Similar to episodic judgments, multi-episodic judgments are acquired after finishing a usage episode.
If a multi-episodic judgment is required, it is always assessed after the episodic judgment to prevent an influence on the episodic judgment due to the presentation of an additional question occurring only in some cases.
In fact, assessing multi-episodic perceived quality directly after the episodic judgment might increase the impact of the this very episode on this multi-episodic judgment.
However, it is not yet known, if such an effect occurs and, if so, how much time should be between the episodic judgment and multi-episodic judgment.
Following \citet{moller_single-call_2011} the 7-point \ac{CCR} scale is applied for all judgments about perceived quality (\cf, \autoref{img:chap05:quality-scale}, \autopageref{img:chap05:quality-scale}).
The 7-point \ac{CCR} allows for more fine-grained judgments than the 5-point \ac{ACR} scale.
This enables to observe small differences between different conditions, but can be a source of noise (between subject and also within-subject) as more rating possibilities are provided.
Using the same scale for episodic judgments and multi-episodic judgments enables a direct comparison without requiring a conversion between scales.

\subsection{Performance Levels}
For the investigation of multi-episodic perceived quality I will use three performance levels.
Those are denoted as \acf{HP}, \acf{MP}, and \acf{LP}.
\ac{HP} denotes the highest performance that should lead to higher episodic judgment than \ac{MP} and both should be judged better than \ac{LP}.
In line with \citet{moller_single-call_2011} performance levels are selected to provide a near \emph{constant} quality per episode, \ie, no macroscopic performance fluctuations.
This avoids effects due to within-episodic quality fluctuations as effects on episodic judgments as well multi-episodic judgments are not yet fully understood (\cf, \autoref{chap:04}).
The presentation of all three performance levels in one condition allows to investigate the existence of a peak effect, which cannot be quantified using two performance levels alone (\cf, \autoref{chap:state-of-the-art}).
%In difference to \citet{moller_single-call_2011} in all conditions \ac{HP} as well as \ac{LP} are presented.

The results of \cite{moller_single-call_2011} show that episodic judgments are in line with the defined performance levels as those show a clear difference of episodic judgments between \ac{HP} and \ac{LP}.
However, the impact on the multi-episodic judgments due to \ac{LP} usage episodes is very limited (\cf, \autoref{prior:moeller}).
This indicates that the selected performance level for \ac{LP} did produce degradations that were perceived, but were not severe enough to produce a clear, observable effect on multi-episodic judgments.
Thus, \ac{LP} must be selected in such a way that degradations are \emph{severe} enough to produce an \emph{observable} effect on multi-episodic perceived quality, but still allow successful task fulfillment.

\subsection{Usage Periods}
\citet{moller_single-call_2011} applied the \emph{defined-use} methodology in a usage period of 12 days. 
This period has been expected to allow new users of a service to determine its multi-episodic perceived quality while allowing for a realistic usage period to determine service adaption, \ie, future use \citep[\cf,][]{moller_single-call_2011}.
However, it is so far not known, if the length of the usage period affects the formation process of multi-episodic perceived quality.

Whereas \citet{moller_single-call_2011} focused on a usage period of multiple days, multi-episodic perceived quality also occurs in one session, if this session consists of multiple, distinct usage episodes.
Studying multi-episodic perceived quality in one session alone allows to conduct the experiment under controlled laboratory conditions similar to standardized experiments on sub-episodic as well as episodic perceived quality.
In fact, this limits the duration of usage period to avoid an influence of fatigue.
Typical experiments on perceived quality do not exceed \unit[90]{min} to avoid such influence.
In fact, reducing the usage period to such a short time-frame compared to multiple days reduces the required effort, allows to study a higher number on conditions in detail.
This avoids potential sources of hard to explainable variance for example due to differences in the presentation environment.

Beside the reduced effort investigation multi-episodic perceived quality in session, the findings form a meaningful starting point for investigate multi-episodic perceived quality over several days.

\subsection{Service Types}
With regard to multi-episodic perceived quality for telecommunication services, such types of services are well suited that are well-known with regard to perceived quality as well as are frequently used in daily life while enabling rather short duration per usage episode.
In the following I present the service types including tasks that I will use to investigate multi-episodic perceived quality.
In fact different service types must be considered as it is not known, if the formation process of multi-episodic perceived quality is affected by the service type.
For each service type a task must be selected in such way that tasks result in an episodic experience.

\subsubsection*{Telephony}\label{method:sct}
Speech telephony services provide the live communication of between two or more remote parties for spoken interaction.
This is a well established and, in fact, classical service for telecommunication providers.
The quality perception and underlying influence factors for speech telephony are well understood and standardized evaluation methods have been developed for evaluation of perceived quality.
This also allows to follow prior work of \citet{moller_single-call_2011}, which used a video telephony service.

\paragraph*{Two-party Conversation}
Speech telephony is most often used for the communication between two remote parties, who engage in an conversation.
A conversation is in itself interactive as in general roles of speaker and listener are changing while caller and callee speak with each to exchange information \citep[\eg,][]{hopper_telephone_2002}.
The interaction behavior of caller and callee, in fact, can affect the perceived quality for both parties \citep[\cf,][]{schoenenberg_why_2014}.

For the evaluation of perceived quality standardized methods have been developed to achieve a comparable interaction behavior in a conversation and thus limited the impact of variance due to differences in usage behavior.
The \acp{SCT} have been especially developed to simulate typical conversational tasks that caller and callee need to solve together \citep[\cf,][p. 76]{moller_assessment_2000}.
Here caller and callee need to exchange a defined set of information while a conversation structure is suggested.
The \acp{SCT} mimic a common situation in which the caller has a demand with specific requirements, which he tries to fulfill by initiating the telephone conversation and informing the callee about his needs.
Based upon this information the callee selects the appropriate option or information and presents this to the caller.
If the presented option or information fulfills the requirements of the caller and is confirmed by him, a second information transfer is initiated in which the callee provides the caller information, so that he can finally fulfill the requirement using the selected option.
The standardized \acp{SCT} \citep{itu-t_p.805:_2007} usually result in a conversation duration of 3~to 7 min.

While conversations and the resulting perceived quality is affected by varying user behavior, it allows to study the whole range of degradations for speech telephony and also determine how those degradations \emph{affect} the usage behavior.
Furthermore, an active conversation allows to investigate the impact of degradations in a setting, in which a speech telephony service is actually used.

Beside the advantages the evaluation of perceived quality in an active conversation requires a large effort, because \emph{a)} a service that can provide desired performance levels reliable must be available and \emph{b)} the varying user behavior is an additional factor influencing in quality judgments.

\paragraph*{Third-party Listening}
Perceived quality of speech telephony to a certain degree can be assessed in a passive situation.
Here a participant listens to a two-party conversation and thus is not an actual part of this conversation, \ie, his behavior cannot affect the conversation.
This is denoted as third-party listening \citep[\cf,][p. 13]{itu-t_p.832:_2000}.
This is, in fact, an uncommon situation as it rarely occurs in a two-party conversation, but is not a uncommon situation in multi-party conferencing.
The elimination of user behavior allows to use recordings of conversations and thus present the exact same material to multiple participants.
If the degradations under considerations do not affect the user behavior, \eg, Lombard speech, the degradations can be inserted in a post-processing stage to the recordings of the conversation.
In fact, using recordings alone avoids requiring a technical system that can provide required performance in real-time.

A passive listening-only has one major limitation beside the inability to assess the impact of degradations on the speaking phase.
A passive observer is not forced to follow a conversation, if it is known to him beforehand that the content is not relevant for him.
This can be avoided by applying a task, which requires to follow the conversation.
If conversations are based upon \acp{SCT}, a note taking task can be applied.
Here the task is to write down all information exchanged between caller and callee, and vice versa.
As standardized \acp{SCT} contain similar content, this task can be structured by asking content-specific questions like \emph{"What is the name of the caller/callee?"}, or \emph{"What does the callee want?"}.
%In a \ac{SCT} induced conversation those question are implicitly answered by caller and callee.

For the assessment of multi-episodic perceived quality, third-party listening has some advantages over two-party conversation although the usage situation is artificial.
Most important the task can be solved by one subject alone and thus eliminating the need of a conversation partner.
Furthermore, the same conversation with preprocessed performance can be presented to multiple participants in exactly the same manner.
%In this case also the length of a usage episode is known beforehand, allowing to investigate the impact episode duration on retrospective judgments.

\subsubsection{Entertainment Media Consumption: Audio and Audiovisual Services}
Telecommunication services for media consumption provide a unidirectional transmission of (multi-)media content to a user on his request.
This can be unimodal content like audio, or speech, as well as multi-modal content like audiovisual content.

A typical usage scenario is the provision of media content for entertainment purpose.
Services that provide media content on-demand are denoted \ac{AoD} for audio-only and \ac{VoD} for audiovisual content.
Here a user can select from available content, the currently desired item, which is then transmitted to him.
%A common use case for media-on-demand services is the consumption of a series content-related parts like a TV series.

While the media selection procedure is often interactive as a user can select the desired content, the actual media consumption only provides limited interactivity.
Here a user might be allowed to pause, seek or abort the consumption.
However, this interactivity can limited completely by presenting predefined content and not allow in-presentation interaction.

%In fact, a third-party listening could be considered as a special case of media consumption, \ie, consumption of recorded two-party speech conversation.
%However, third-party listening focuses on the simulation of multi-party conferencing in which the observer takes a passive part of the conversation whereas media consumption focus on the media consumption alone.
In difference to telephony, media consumption focuses on the consumption of pre-produced content.
This allows to use high-end recording equipment, adequate post-processing, and efficient but time consuming compression.
This, in general, limits the sources of severe degradations in general to transmission as well as reproduction.
%Recently, media-on-demand services have been widely deployed that adapt to the current network conditions to avoid stalling at a reduced service performance, \eg, limiting transmission bandwidth.
%For a usage episode this can either happen while content is consumed resulting in performance fluctuations, or beforehand trying to achieve a constant service performance.

Media-on-demand services are well-suited for investigation multi-episodic perceived quality, because those can be setup in a non-interactive way and thus avoid impact of user behavior especially on duration of usage episodes.
Furthermore, media-on-demand services are nowadays widely used in mobile as well as in stationary usage situations, \ie, home entertainment. 

\section{Hypotheses}
Based upon prior work on retrospective experiences (\cf, \autoref{chap:03}) and quality assessment (\cf, \autoref{chap:04}), I derived 7 hypotheses to provide information about the formation process.
Those hypotheses are presented in the following.

\subsection{\emph{Hypothesis}: Number of Consecutive \acs{LP} Episodes}
\begin{hypothesis}[\autoref{hypo:number}]\label{hypo:number}
Increasing the number of \ac{LP} episodes before a multi-episodic judgment decreases this judgment.
\end{hypothesis}

The presentation of \ac{LP} episode(s) is expected to result in a reduction in multi-episodic judgment compared to the presentation of those usage episode(s) in \ac{HP}.
When presenting all episodes in \ac{HP}, then the multi-episodic judgment should be sufficiently reflected by averaging all prior episodic judgments \citep[\cf,][]{moller_single-call_2011}.
The more \ac{LP} episode(s) are presented, the higher is the expected reduction until multi-episodic judgments reach the same level as the episodic judgments of \ac{LP} usage episodes.
This is expected to be a lower boundary.

\subsection{\emph{Hypothesis}: Position of \acs{LP} Episode(s)}
\begin{hypothesis}[\autoref{hypo:position}]\label{hypo:position}
Increasing the number of \ac{HP} episode(s), which follow one or more \ac{LP} episode(s), results in a lower reduction of the multi-episodic judgment.
\end{hypothesis}

A \emph{recency effect} has been observed in sequential learning, retrospective judgments of experiences in general, and for perceived quality with one stimulus containing macroscopic performance fluctuations.
If an effect of recency affect multi-episodic perceived quality, then episodes with close temporal proximity to this judgment must have a higher impact on it.
By varying the position of \ac{LP} episode(s) before this judgment, the existence of a recency effect can be investigated.
If a recency effect occurs, conditions that present \ac{LP} episode(s) closer to the multi-episodic judgment will result in a lower multi-episodic judgment than those conditions that present more \ac{HP} episodes following the \ac{LP} episode(s).

\subsection{\emph{Hypothesis}: Recovery after \acs{LP} Episodes}
\begin{hypothesis}[\autoref{hypo:recovery}]\label{hypo:recovery}
Presenting additional \ac{HP} episodes after a negatively affected multi-episodic judgment, increases the following multi-episodic judgment.
\end{hypothesis}

This hypothesis is similar to \autoref{hypo:position}, but focuses on the recovery of a negatively affected multi-episodic judgments.
Presenting after this judgment and before the next multi-episodic judgment only \ac{HP} episodes, should result in an increase of the final multi-episodic judgment.
If enough \ac{HP} episodes are presented, the final judgment should reach the same level as if no \ac{LP} episodes were presented at all.

\subsection{\emph{Hypothesis}: Non- vs. Consecutive \acs{LP} Episodes}
\begin{hypothesis}[\autoref{hypo:consecutive}]\label{hypo:consecutive}
The presentation of non-consecutive \ac{LP} episodes lead a higher reduction in multi-episodic judgments than the number as consecutive \ac{LP} episodes.
\end{hypothesis}

Presenting consecutive \ac{LP} episodes is expected to produce higher multi-episodic judgments than the same number of non-consecutive \ac{LP} episodes, because in the latter case the service performance varies more often.
It is thus not reliable and less predictable for the user than in the former case.

\subsection{\emph{Hypothesis}: Duration of \acs{LP} Episode}
\begin{hypothesis}[\autoref{hypo:duration}]\label{hypo:duration}
\ac{LP} episodes with a longer duration have a higher impact on multi-episodic perceived quality than shorter \ac{LP} episode, because the experienced duration the performance level \ac{LP} is longer.
\end{hypothesis}

For retrospective judgments of episodic experiences with variations a \emph{duration neglect} could be observed, \ie, the actual duration of a variations in an episodic experience is not reflected in the retrospective judgment (\cf, \autoref{chap:03}).
Such an effect must not necessarily occur for multi-episodic perceived quality.
In \autoref{hypo:number} the number of \ac{LP} episodes and the impact on multi-episodic judgment is investigated.
For an increasing number of \ac{LP} episodes, a longer overall duration of \ac{LP} is experienced.
This, however, leaves open, if the formation process of multi-episodic perceived quality relies \emph{a)} on the overall duration of \ac{LP} episode(s), \emph{b)} the number of \ac{LP} episode(s), or \emph{c)} both.
In fact, for near constant performance the actual duration has been found to have a constant positive but rather small effect on a retrospective judgments, if the duration is longer than \unit[30]{sec} \citep[\cf,][]{frohlich_qoe_2012}.
However, the shift is constant suggesting an impact of usage situation rather than an encoding of duration into the episodic judgment.
In case of a) and c) episodic judgments alone would not contain all information for prediction of the multi-episodic judgment.\footnote{\citet[p. 2]{moller_single-call_2011} did not define the duration of \ac{LP} usage episodes and thus implicitly assumed a duration neglect in the formation process of multi-episodic perceived quality.}

The impact of the duration of \ac{LP} episodes on multi-episodic perceived quality can be investigated by comparing the multi-episodic judgment of at least two conditions with the same number and positions of \ac{LP} episode(s) with varying duration of the \ac{LP} episode(s).

\subsection{\emph{Hypothesis}: Strength of one \acs{LP} Episode}
\begin{hypothesis}[\autoref{hypo:strength}]\label{hypo:strength}
The \emph{lowest} experienced episodic performance has a higher impact on multi-episodic judgment.
\end{hypothesis}

The so-called peak effect, which has been observed in retrospective assessment of episodic experiences, denotes the higher impact of the worst part of an experience on this judgment (\cf, \autoref{chap:03}).
Such an effect has been observed for perceived quality of one stimulus as well as for one usage episode affecting the quality formation process.
Such an effect might also occur for multi-episodic perceived quality, but has so far not been investigated.
It can be investigating by presenting more than two performance levels, which result in different episodic judgments.
If a no difference in multi-episodic judgments can be observed, this indicates a peak effect.

\subsection{\emph{Hypothesis}: Services are Judged Independent}
\begin{hypothesis}[\autoref{hypo:independent}]\label{hypo:independent}
The multi-episodic judgment for one service is not affected not affected by the presentation of a second service in the same usage period.
\end{hypothesis}

The quality formation process is affected by expectations, knowledge, and prior experiences, which also change the encoding of the perceived quality into a judgment.
With regard to multi-episodic perceived quality, however, the use of another service might affect the multi-episodic judgment of the assessed service.
Here participants might either use one service as reference of even might not be able to attribute perceived quality to a service for the multi-episodic judgment.
This might especially occur, if services are very similar.
While for studying multi-episodic perceived quality in one session alone, such an effect can be prevent by using one service alone.
In a usage period covering several days, this can hardly be prevented.
It is therefore important to understand, if the use of other services affect this multi-episodic judgment of a service under multi-episodic quality assessment.

\section{Conclusion}
In this chapter I first presented the aspects that I will base my investigation of multi-episodic perceived quality on.
This included the different judgments, performance levels and also service types.
In the second part I presented the hypotheses I that I used to design experiments.
In the following chapter the experiments on multi-episodic perceived quality in one session are presented while in \autoref{chap:field} presents the experiments covering a usage period of multiple days.