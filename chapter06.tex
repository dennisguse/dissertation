\chapter{Multi-episodic QoE in 1 hour}\label{chap:06}
\section*{Abstract}
Here I describe my approach towards understanding multi-episodic QoE.
Major question: can multi-episodic QoE be studied in "short" laboratory study?
I with my laboratory studies on with one speech telephony service focusing on effects of position(s) of degraded usage episodes.%(the finished Journal paper).
Then the parallel-use studies are presented (impact of distraction).
This chapter closes with the description of cross-service QoE (using results of studies using two services).
%Important point: argue that must degradations must be severe!

\section{Sequential use} %Consider one service alone! Cross-service below
Studied aspects using speech telephony

Describe study design in the beginning as it is similar for all sequential studies

\begin{itemize}
\item Position (Conversation and Listening): yes/no, found for conversation
\item Length of degradation (Conversation and Listening): yes, similar
\item Task (Conversation vs. Listening): their might be an effect due to mental capacity
\item Strength of degradation (Conversation): yes, supports peak
\item Length of degraded episode (Extension of Study 2): results not yet known %TODO
\item Does episodic quality judgment impact multi-episodic QoE? %TODO
\item Second _unimpaired_ service: reduced effect of LP episodes; reason: good quality?; 
\item Second _impaired_ service: same effect
\end{itemize}

\section{Parallel use}
Present here that using two services in parallel only has a slight impact on Web-QoE.
First present degraded web-only and then web+TV.

\section{Multi-episodic QoE of two services} %TODO Cross-service as extra chapter?