%*******************************************************
% Abstract
%*******************************************************
%\renewcommand{\abstractname}{Abstract}
\pdfbookmark[1]{Abstract}{Abstract}
\begingroup
\let\clearpage\relax
\let\cleardoublepage\relax
\let\cleardoublepage\relax

\chapter*{Abstract}
Telecommunication services have to cope with degradations resulting from the transmission of data.
A telecommunication service might thus not be able to to provide the same performance for a user, if it is used repeatedly by this user.
If perceived, this variation in perceived quality might affect the user's satisfaction, attitude, behavior, and also future-use intention towards this telecommunication service.

In this thesis, the integration process towards the perceived quality over multiple, distinct interactions with one telecommunication service is investigated.
The integration process of the so-called multi-episodic perceived quality is investigated here for two time spans.
As a baseline repeated use in one session is investigated. %of up to \unit[45]{min}
This is complemented by investigating the integration process considering usage over several days.
This investigating was conducted by performing several empirical experiments under controlled laboratory settings and also under field situations.
These experiments are based upon the \acs{MOS}, \ie, let the perceived quality of (almost) identical stimulus/condition be assessed by multiple observers, to derive the judgment of an \emph{average observer}.
The impact of individual behavior is limited here by applying the defined-use method, \ie, defining the task, content, and also time for each episode as well as the performance level.
Besides verifying feasibility, the results of the experiments show that the integration the integration process resembles an averaging process with a higher impact of more recent episodes.
It could be observed that the found effects do not seem to depend on the duration of the experiments, \ie, one session versus multiple days.

%\vfill
%
%\begin{otherlanguage}{ngerman}
%\pdfbookmark[1]{Zusammenfassung}{Zusammenfassung}
%\chapter*{Zusammenfassung}
%
%
%\end{otherlanguage}
\endgroup			
\vfill