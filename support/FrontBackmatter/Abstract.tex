%*******************************************************
% Abstract
%*******************************************************
%\renewcommand{\abstractname}{Abstract}
\pdfbookmark[1]{Abstract}{Abstract}
%\begingroup
%\let\clearpage\relax
%\let\cleardoublepage\relax
%\let\cleardoublepage\relax

\chapter*{Abstract}
Telecommunication services have to cope with degradations resulting from the transmission of data.
% Impairments of data transmission can lead to degradations of perceived quality in telecommunication services. %
A telecommunication service might thus not always be able to to provide the same performance for a user.
% A user might thus not be provided with constant service performance. %
This variation in perceived quality might affect the user's satisfaction, attitude, behavior, and also future-use intention towards a telecommunication service.

%In this thesis, the integration process of perceived quality over multiple, distinct interactions, \ie, usage episodes, with one telecommunication service is investigated.
This thesis investigates, the formation process of perceived quality across multiple, distinct interactions with one telecommunication service.
%This formation process of the so-called multi-episodic perceived quality is examined in this thesis for two time spans.
The formation process of the so-called multi-episodic perceived quality is examined for two different time spans.
Here, repeated use in one session consisting of multiple usage episodes is investigated with an overall duration of up to~\unit[45]{min}.
% First, we investigated repeated use in one single session consisting of multiple usage episodes with an overall duration of up to~\unit[45]{min}. % 
This is complemented by studying the integration process covering a usage period of several days.
% Second, we studied the quality formation process covering a usage period of several days. %

This investigation was conducted by performing empirical experiments under controlled laboratory settings as well as under field conditions.
These experiments are based upon the Mean Opinion Score \acs{MOS}, \ie,  the perceived quality of an (almost) identical stimulus/condition is assessed by multiple observers to derive the judgment of an~\emph{average observer}.
The impact of individual user behavior was limited here by defining the task, content, and also time for each usage episode as well as the provided performance (defined-use method).
The empirical data shows that applying the defined-use method is feasible and yields consistent results.

%The results of the experiments show that more recent episodes have a higher impact on the multi-episodic perceived quality (recency~effect) while a saturation is observed for consecutive degraded usage episodes, \ie, the multi-episodic judgment remains on the same level \emph{above} the judgment of degraded usage episodes.
The results of the experiments show that more recent episodes have a higher impact on the multi-episodic perceived quality (recency~effect). 
A saturation is observed for consecutive degraded episodes, \ie, the multi-episodic judgment remains on the same level \emph{above} the judgment of degraded episodes.
In addition, a duration neglect is observed, \ie, a longer degraded episode does not have a higher negative impact on the judgment of multi\=/episodic perceived quality. 

%With the empirical data, prediction models based upon the weighted average are evaluated.
%For prediction of multi-episodic judgments based upon episodic judgments, a linear weighting function outperforms a window function with regard to prediction accuracy and robustness.

With the empirical data, models for the prediction of multi-episodic judgments are evaluated.
These models are based on the weighted average of single episodic judgments.
The evaluation shows that a linear weighting function outperforms a window function in regard to prediction accuracy and robustness.

%\vfill

\begin{otherlanguage}{ngerman}
\pdfbookmark[1]{Zusammenfassung}{Zusammenfassung}
\chapter*{Zusammenfassung}
Telekommunikationsdienste ermöglichen Sender und Empfänger den zeitnahen Austausch von Daten.
Aus Sicht eines Nutzers unterliegt die wahrgenommene Qualität eines Telekommunikationsdienstes häufig Schwankungen, da der Dienst nicht immer in der Lage ist die gleich Performanz zu liefern.
Diese Schwankungen können die Zufriedenheit, Meinung, Nutzungsverhalten als auch zukünftige Nutzungsverhalten beeinflussen.

In dieser Dissertation wird der Integrationsprozess der wahrgenommenen Qualität bei wiederholter Nutzung, einzelner Nutzungsepisodes, eines Telekommunikationsdienstes untersucht.
Dieser Integrationsprozess der sogenannten multi-episodischen wahrgenommenen Qualität wurde für zwei Zeitspannen untersucht.
Zuerst wurde die Integration mehrere Nutzungsepisoden in einer Session mit einer maximalen Dauer von~\unit[45]{min} untersucht.
Auf Basis dieser wurde die Untersuchung auf die Integration über mehrere Tage komplimentiert.

Für die Untersuchung wurden mehrere empirische Experimente unter kontrollierten Laborbedingungen als auch unter realen Bedingungen im Feld durchgeführt.
Diese Experimente basieren auf dem \acs{MOS}, dass heißt die wahrgenommene Qualität des (annähernd) gleichen Stimulus/Bedingung wird durch mehrere Beobachter bewerten, um die Bewertung des \emph{durchschnittlichen Beoachters} zu erhalten.
Der Einfluss von unterschiedlichem Nutzungsverhalten wurde durch die Methode der definierten Nutzung stark eingeschränkt, so dass der Einfluss des Nutzungsverhalten die Bewertung möglichst wenig beeinflusst.
Herzu wird für jede Nutzungsepisode festgelegt wann, wie und mit welchem Inhalt diese durchgeführt werden soll und welche Performanz bereitgestellt wird.

Die empirischen Daten zeigen, dass die Methode der definierten Nutzung durchführbar ist und konsistente Ergebnisse liefert.
Die Ergebnisse der durchgeführten Experimente zeigen, dass die wahrgenommene Qualität späterer Nutzungsepisoden einen größeren Einfluss auf die multi-episodisch wahrgenommene Qualität haben.
Weiterhin wurde eine Sättigung für aufeinanderfolgenden gestörte Nutzungsepisoden beobachtet.
Hierbei sinkt die Bewertung der multi-episodischen wahrgenommenen Qualität nicht weiter, obwohl mehr gestörte Nutzungsepisoden präsentiert worden und die Bewertung noch über dem Niveau von gestörten Nutzungsepisoden ist.
Darüberhinaus wurde beobachtet, dass die Dauer einer gestörten Nutzungsepisode die multi-episodischen wahrgenommenen Qualität nicht beeinflusst.

Auf Basis der empirischen Daten wurden Vorhersagemodell auf Basis des gewichteten Mittelwerts untersucht.
Hierbei zeigte sich, dass eine lineare Gewichtung eine bessere Genauigkeit als auch Stabilität liefert als eine Fensterfunktion.

\end{otherlanguage}
%\endgroup			
%\vfill