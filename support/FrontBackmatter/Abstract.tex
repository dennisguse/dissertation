%*******************************************************
% Abstract
%*******************************************************
%\renewcommand{\abstractname}{Abstract}
\pdfbookmark[1]{Abstract}{Abstract}
%\begingroup
%\let\clearpage\relax
%\let\cleardoublepage\relax
%\let\cleardoublepage\relax

\chapter*{Abstract}
Telecommunication services have to cope with degradations resulting from the necessary transmission of data.
% Impairments of data transmission can lead to degradations of perceived quality in telecommunication services. %
A telecommunication service might thus not always be able to provide the same performance to a user.
% A user might thus not be provided with constant service performance. %
The resulting variation in perceived quality might affect the user's satisfaction, attitude, behavior, and also future-use intention towards a telecommunication service.

%In this thesis, the integration process of perceived quality over multiple, distinct interactions, \ie, usage episodes, with one telecommunication service is investigated.
This thesis investigates the formation process of perceived quality across multiple, distinct interactions with one telecommunication service.
%This formation process of the so-called multi-episodic perceived quality is examined in this thesis for two time spans.
The formation process of the so-called multi-episodic perceived quality is examined for two different time spans.
Here, repeated-use in one session consisting of multiple usage episodes is investigated with an overall duration of up to~\unit[45]{min}.
% First, we investigated repeated use in one single session consisting of multiple usage episodes with an overall duration of up to~\unit[45]{min}. % 
This is complemented by studying the formation process spanning several days.
% Second, we studied the quality formation process covering a usage period of several days. %

This investigation was conducted by performing empirical experiments under controlled laboratory settings as well as field experiments.
These experiments are based upon the \acf{MOS}, \ie, the assessment of the perceived quality of an (almost) identical stimulus/condition by multiple observers to derive the judgment of an~\emph{average observer}.
The impact of individual user behavior was limited here by defining the task, content, and also time for each usage episode as well as the provided performance (defined-use method).
The empirical data shows that applying the defined-use method is feasible and yields consistent results.

%The results of the experiments show that more recent episodes have a higher impact on the multi-episodic perceived quality (recency~effect) while a saturation is observed for consecutive degraded usage episodes, \ie, the multi-episodic judgment remains on the same level \emph{above} the judgment of degraded usage episodes.
The results of the experiments show that more recent episodes have a higher impact on the multi-episodic perceived quality (recency~effect). 
A saturation is observed for consecutive degraded episodes, \ie, the multi-episodic judgments remain on the same level \emph{above} the episodic judgments of degraded episodes.
In addition, a duration neglect is observed, \ie, a longer degraded episode does not have a higher negative impact on judgments of multi\-/episodic perceived quality. 

%With the empirical data, prediction models based upon the weighted average are evaluated.
%For prediction of multi-episodic judgments based upon episodic judgments, a linear weighting function outperforms a window function with regard to prediction accuracy and robustness.

With the empirical data, models for the prediction of multi\-/episodic judgments are evaluated.
These models are based on the weighted average of the episodic judgments.
The evaluation showed that a linear function outperforms a window function in regard to prediction accuracy and robustness.

%\vfill

\begin{otherlanguage}{ngerman}
%{\let\cleardoublepage\relax \chapter*{Zusammenfassung}}
\chapter*{Zusammenfassung}
\pdfbookmark[1]{Zusammenfassung}{Zusammenfassung}
Da Datenübertragungen anfällig für Störungen sind, kann die wahrgenommene Qualität eines Telekommunikationsdienstleisters permanent Schwankungen unterliegen.
Diese können die Zufriedenheit, die Meinung sowie das gegenwärtige und zukünftige Nutzungsverhalten ihres Anwenders beeinflussen. 

In dieser Dissertation wird untersucht, wie sich die wahrgenommene Qualität bei wiederholter Nutzung eines Telekommunikationsdienstleisters zusammensetzt.
Die sogenannte multi\-/episodisch wahrgenommene Qualität wurde für jeweils zwei Zeitspannen evaluiert.
In einem kontrollierten Laborversuch wurden Probanden in einer Session von bis zu 45 Minuten in mehrere Nutzungsepisoden involviert.
Auf Basis der gewonnenen Ergebnisse wurde eine Untersuchung der Nutzung über mehrere Tage konzipiert und als Feldversuch umgesetzt.
Alle Experimente basieren auf dem \acf{MOS}, d.\,h. die wahrgenommene Qualität jedes einzelnen Nutzers auf einen annähernd gleichen Stimulus wird zu einem Durchschnittswert aller Nutzer zusammengefasst. 
Um den individuellen Einfluss des Nutzungsverhaltens auf die Bewertung möglichst gering zu halten, wurden Art, Dauer, Inhalt und Performanz jeder Nutzungsepisode streng definiert.
Die empirischen Daten zeigen, dass die Methode der definierten Nutzung durchführbar ist und konsistente Ergebnisse liefert.

Die Ergebnisse der durchgeführten Experimente zeigen, dass die wahrgenommene Qualität zeitlich später erlebter Nutzungsepisoden einen größeren Einfluss auf die multi\-/episodisch wahrgenommene Qualität des gesamten Erlebnisses hat.
Weiterhin wurde eine Sättigung für aufeinander folgende gestörte Nutzungsepisoden beobachtet: Obwohl mehrere gestörte Nutzungsepisoden hintereinander präsentiert wurden, sinkt die Bewertung der multi\-/episodisch wahrgenommenen Qualität nicht weiter.
Auffällig ist, dass die multi\-/episodische Bewertung deutlich positiver ausfällt als die Bewertung der einzelnen gestörten Episoden.
Unterschiede in der Dauer einer gestörten Nutzungsepisode zeigen hingegen keinen Einfluss auf die multi\-/episodisch wahrgenommene Qualität. 


Auf Basis der empirischen Untersuchungen wurde ein Modell zur Vorhersage der multi\-/episodischen Bewertung konzipiert.
Dieses beruht auf dem gewichteten Mittelwert der einzelnen Bewertungen aller bisher erlebten Episoden.
Hierbei erweist sich eine lineare Gewichtung als genauer und stabiler als eine Fensterfunktion. 

\end{otherlanguage}
%\endgroup			
%\vfill