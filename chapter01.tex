\chapter{Introduction}\label{chap:01}

\section{Motivation}
In contrast to products, which can be manufactured and stored, a service is created on demand.
A service is simultaneously produced by the \emph{service provider} and consumed by the \emph{customer}.
A typical example is a \emph{telecommunication service} like speech telephony, or Internet access.
%US Communications Act 1934
Telecommunication services provide communication between one or more parties, \ie, data transmission.
Here, a party can be a person, but also a computer.
A service provider maintains the network infrastructure to provide telecommunication services and makes it available to potential users\footnote{Throughout this thesis a person engaging a service is called \emph{user} of this service without distinction if he is the actual customer.}.
An actual service is produced when a user interacts with a telecommunication service.

As a service is produced when it is used, the provided \emph{performance} might be varying in a usage instance as well as between distinct usage instances.
Here, performance are absolutely measurable parameters of this service like transmission bandwidth and one-way transmission delay~\citep[][p. 12]{moller_assessment_2000}.
In case of a telecommunication service, varying performance can be the result of current network load conditions, network equipment failure, but also related to end-user devices.
Usage of telecommunication services can be divided into two aspects.
First, a user can actively interact with a service like starting and engaging a telephone conversation.
Second, a service can also be used passively, \eg, providing reachability of a telephone.

The active use of a telecommunication service results in a \emph{perceived quality} of this interaction.
It is assumed that perceived quality results from a comparison between the \emph{desired experience} and \emph{actual experience} \citep[][p.~13]{raake_quality_2014}. %Expected experience?
Experience includes all perceptions resulting from this service instance \citep[][p.~13]{raake_quality_2014}.
The perceived quality of an interaction with a service is affected by the experienced service performance, but also by individual factors like expectations, usage situation, etc \citep[\eg,][p.~55ff.]{reiter_quality_2014}.

%QoE Methods; human perception 
Understanding the relationship between performance of a service and perceived quality has become an important field of research investigating the so-called \ac{QoE}.
Starting from speech transmission for telephony \citep[][]{licgee_ieee_1969} and its inherent impairments, \ac{QoE} now mainly considers digital transmission of multimedia data. % \citep{moller_quality_2014}.
This includes the production, coding, transmission, decoding and reproduction of text, voice, speech, image and video information.
Specific characteristics of human perceptual system have been used to develop enhanced coding mechanisms like \ac{MP3} and \ac{H.264}.
These use apply compression to provide a reduction in data rate while maintain only little reduction in perceived quality.

Another application of knowledge about \ac{QoE} is the monitoring of network transmission infrastructure.
This allows to estimate the actual impact of provided performance on a user.
Such knowledge enables a service provider to react accordingly to avoid reductions in perceived quality.
For example, in case of a video streaming service stalling, provoked by a temporary reduction in network bandwidth, could be avoided by reducing the transmission bandwidth.
Although perceived quality is likely to be affected, the service is still provided and remains useful for the user.

%Service choice
In general, a user can choose from a variety of service providers, which provide a similar service, and select the one provider that suits his needs best.
For a reasoned selection a user must know his needs, requirements, economical limits and estimate the perceived quality (\ie, \emph{assumed quality}) \citep[][p.~13]{raake_quality_2014}.
Based upon this knowledge he can estimate, if the usage of a service is likely to be satisfying for him.
After experiencing an interaction with a service, the user can then evaluate, if this service fulfills his needs including his perceived quality.
He can then decide to use the service again or rather select a different service provider, when he needs this type of service again~\citep[][]{geerts_linking_2010}.

\section{Research Question}
Of special interest is the perception of varying performance over distinct interactions with one service and the resulting perceived quality.
%This is, in fact, a common case for telecommunication service(s).
%For example a user of a mobile telephony provider will use the provided service(s) usually repeatedly as his telephone number is bound to the service provider.
Especially for telecommunication services the impact of varying performance from usage instance to usage instance is of interest.
Although, the impact of varying performance during one usage episode has been investigated, it is so far not known how perceived quality evolves over several distinct episodes (\ie, individual experiences).
This is one factor that contributes to the \emph{service quality} of a service \citep[][]{berry_quality_1985, zeithaml_behavioral_1996}.
The construct of service quality maintains a holistic view from a business perspective of service usage focusing on aspects likely customer loyalty\cite{parasuraman_conceptual_1985}.
%In fact, perceived quality is an implicit part of service quality.

In this thesis I investigate perceived quality over several distinct and meaningful interactions with a service.
Such an interaction is denoted as \emph{usage episode} (\cf, \autoref{chap:03}).
The quality of a usage episode is denoted as \emph{episodic quality}.
The perceived quality over several distinct usage episodes is denoted as \emph{multi-episodic perceived quality}.

In this thesis I address one research question: 
\subparagraph*{Research Question:}
How does multi-episodic quality for one user evolves over several usage episodes with one service?}

\subparagraph*{}
%Memory
%Each usage episode results in a perceived quality, which can be assessed momentary during the usage episode and also in retrospective after the usage episode is finished.
So far, the integration process of perceived quality of distinct usage episode into a multi-episodic perceived quality is not yet known.
It could be a continuous process that integrates current experiences immediately into a current view, or a retrospective assessment evaluating all memorized and recallable information. 
In fact, it could also be a mixture of both.%\footnote{\citet{hogarth_order_1992} proposes th}

%%%I focus on telecommunication services(s) as those are well-studied with regard to perceived quality, are widely used, and variations in performance, irrespective of the actual reason, occur relative frequently.

\section{Goals}
In this thesis, I pursue two goals towards understanding the formation process of multi-episodic perceived quality.
I will focus solely on telecommunication services, because those are prone to variations in performance and usage episodes are relatively short.
As the formation process of multi-episodic perceived quality cannot be observed directly (\cf, \autoref{chap:02}), this investigation will rely on the judgments of multi-episodic perceived quality.
%This investigation will rely on judgments of multi-episodic perceived quality alone as perceived quality cannot be observed directly (\cf, \autoref{chap:02}).

This leads to two goals:

\subparagraph*{Goal 1:}
%I will investigate how episodic performance determines the judgment of multi-episodic perceived quality and investigate potential anomalies of the formation process.
I will investigate how the performance of a sequence of usage episodes determines judgments of multi-episodic perceived quality.

\subparagraph*{Goal 2:}
I will investigate how multi-episodic judgments can be predicted based upon episodic judgments.

\subparagraph*{}
I will conduct this investigation for different \emph{usage periods}.
Usage periods denotes the time frame in which multiple a user interacts repeatedly with a service.
Here, I consider one session consisting of several usage episodes and individual usage episodes spread distributed over several days.
This is necessary, because it is not known if the time between usage episodes affects the formation process of multi-episodic perceived quality.

\section{Structure}
This thesis is structured as follows.
In \autoref{chap:02}, I introduce concepts and fundamentals that form the basis for my investigation of multi-episodic perceived quality.
It starts with an introduction to psychophysics and \ac{QoE}, followed by an introduction into human memory and known biases for retrospective judgments.
Then, the state of the art on perceived quality and performance fluctuations is presented.
Finally, then prior work on multi-episodic perceived quality is presented and discussed.
%Here, the assessment method developed by \citet{moller_single-call_2011}, which is the following denoted as \emph{defined-use}, is presented in detail.

Then, I present my work towards multi-episodic perceived quality. % based upon the defined use method.
In \autoref{chap:towards}, I describe the selected aspects of the defined-use method and the to be investigated hypotheses.
Those hypotheses are the basis for my investigations on multi-episodic perceived quality in one session (\autoref{chap:lab}) as well as over several days (\autoref{chap:field}).
%Those different time frames are necessary to investigate as it is not known, if the time between usage episodes affect the formation process.
Based upon the conducted experiments, I investigate potential models for prediction of multi-episodic judgments based upon prior episodic judgments (\autoref{chap:modeling}).
Finally, I summarize my thesis, discuss the results and present directions for future work in \autoref{chap:discussion}.