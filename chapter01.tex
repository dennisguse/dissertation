\chapter{Introduction}
\section*{Abstract}
Here I give an overview on my thesis.
This includes:
\begin{itemize}
\item general topic
\item field of research: perceptual quality + judgment (relying on memorization and recall?)
\item relevance: why is this question relevant at all? Practical implications?
\item overview on service provider things (money stuff)
\item my personal motivation to tackle this topic (why important to me?)
\item how am I tackling the question?
\item what can the reader expect? (just hints)
\end{itemize}

\section{Goals}
\begin{itemize}
\item How does perceptual quality evolve over multiple interactions (multiple episodes) with the one system/service for one user?
\item How does perceptual overall quality evolve over multiple interactions using a bundle, e.g. more than one system/service for one user? (basically an extension of previous stuff)
\item Goal 1: Find underlying effects temporal effects? 
\item Goal 2: Implement either service-independent or service-dependent models.
\end{itemize}


\paragraph*{Service provider's goal}
Make customer happy for as little money as possible for infrastructure: cost-efficiency, risk reduction
Services usage, re-usage, churn etc.

\section{Basic definitions}
\begin{itemize}
\item What is \emph{to service}? [Whitepaper] [http://www.merriam-webster.com/dictionary/service]
\item What is a service, application, system?
\item Definition of service quality
\item Telecommunication services + Service Provider
\item Where degradations come from? (production/recording, coding, transmission, reproduction)
\end{itemize}

\section{Aspects}
\begin{itemize}
\item Money: NPS, Churn, Acceptability, Willingness-to-paying
\item Retainability (sMOS)
\item Adaptive services: service that adjust performance to current conditions (best + economic feasible)
\end{itemize}

\section{Structure}
%TODO Here I describe the structure of my thesis    