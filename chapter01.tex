\chapter{Introduction}
\section*{Abstract}
Here I give an overview on my thesis.
This includes:
\begin{itemize}
\item general topic
\item field of research: perceptual quality + judgment (with human memory)
\item relevance: why is this question relevant at all? Practical implications?
\item overview on service provider things (money stuff)
\item my personal motivation to tackle this topic (why important to me?)
\item how am I tackling the question?
\item section with explicit goals
\item what can the reader expect? (just hints)
\item Structure of the thesis
\end{itemize}

\section{Goals}
\begin{itemize}
\item How does perceptual quality evolve over multiple interactions (multiple episodes) with the one system/service for one user?
\item How does perceptual overall quality evolve over multiple interactions using a bundle, e.g. more than one system/service for one user?
\item Goal: Find underlying effects and create service-dependent or service-independent models.
\end{itemize}

Temporal perceptual integration for Telecommunication services.
		Goal: Make customer happy for as little money as possible for infrastructure
		Services usage, re-usage, churn etc.

\section{Why is this important? + Side aspects}
\begin{itemize}
\item What is \emph{to service}? [Whitepaper] [http://www.merriam-webster.com/dictionary/service]
\item What is a service, application, system?
\item What is service quality?
\item Money: NPS, Churn, Acceptability, Willingness-to-paying
\item Retainability (sMOS)
\item Telecommunication services + Service Provider
\item Adaptive services: service that adjust performance to current conditions (best + economic feasible)
\item Where degradations come from? (production, coding, transmission, reproduction)
\end{itemize}
