\chapter{Introduction}\label{chap:01}

\section{Motivation}
In contrast to products, which can be manufactured and stored, a service is created on demand.
A service is simultaneously produced by the \emph{service provider} and consumed by the \emph{customer}.
A typical example are \emph{telecommunication services}, such as speech telephony or Internet access.
%US Communications Act 1934
Telecommunication services provide communication between one or more parties in the form of data transmission.
Here, a party can be a person or a computer.
A service provider maintains the network infrastructure to provide telecommunication services and makes it available to potential users\footnote{Throughout this thesis, persons engaged in a service are called \emph{users} of this service without distinction if they are the actual customers.}.
An actual service is created when a user interacts with a telecommunication service.

As a service is created when it is used, the provided \emph{performance} might vary within a usage instance as well as between distinct usage instances.
Performance is the "ability of a unit to provide the function it has been designed for"~\citep[][p.\,360]{moller_quality_2005}.
Performance can be determined by measurable parameters of a service, such as transmission bandwidth and one-way transmission delay~\citep[][p.\,12]{moller_assessment_2000}.
In the case of a telecommunication service, varying performance might be the result of current network load conditions, network equipment failure, or related to end-user devices.
The usage of telecommunication services can be divided into two aspects.
First, a user can actively interact with a service, such as starting and engaging a telephone conversation.
Second, a service can also be used passively, \eg, providing reachability of a telephone.

The active use of a telecommunication service results in a \emph{perceived quality} of this interaction.
It is assumed that the perceived quality results from a comparison between the \emph{desired experience} and the \emph{actual experience} \citep[][p.\,13]{raake_quality_2014}. %Expected experience?
Experience includes all perceptions resulting from this interaction \citep[][p.\,13]{raake_quality_2014}.
The perceived quality of an interaction with a service is determined by the experienced performance, but also by individual factors, such as expectations, usage situation, etc. \citep[\eg,][p.\,55ff.]{reiter_factors_2014}.

%QoE Methods; human perception 
Understanding the relationship between performance of a service and perceived quality has become an important field of research investigating the so-called \ac{QoE}.
Starting from speech transmission for telephony \citep[][]{ieee_audio_and_electroacoustics_group_ieee_1969} and its inherent impairments, \ac{QoE} now mainly considers digital transmission of multimedia data. % \citep{moller_quality_2014}.
This includes the production, coding, transmission, decoding, and reproduction of text, voice, speech, audio, image, and video information.
Specific characteristics of the human perceptual system have been used to develop enhanced coding mechanisms, such as \ac{MP3} and the video codec~\textsc{\lowercase{H.264}}.
These apply compression to provide a reduction in data rate while achieving only little to no reduction in perceived quality.
Knowledge about \ac{QoE} is also applied to planning and monitoring of the network transmission infrastructure for telecommunication services \citep[][]{schatz_qoe-based_2014}.
This enables telecommunication service providers to tailor their network infrastructure to the estimated need and on-the-fly control their infrastructure to avoid reductions in perceived quality for their users.
%Such knowledge enables a service provider to react accordingly to avoid reductions in perceived quality.
For example, in the case of a video streaming service, stalling provoked by a temporary limitation in network bandwidth might be avoidable by reducing the video encoding bandwidth.
Although perceived quality is likely to be affected, the service is still provided and remains useful for the user.

%Service choice
In general, a user can choose from a variety of service providers which provide similar services and select the one provider that suits his needs best.
For a reasonable selection, a user must know his needs, requirements, financial constraints, and estimate the perceived quality, \ie, \emph{assumed quality} \citep[][p.\,13]{raake_quality_2014}.
Based on this knowledge, he can estimate if the usage of a service is likely to be satisfactory to  him.
After experiencing an interaction with a service, the user can then evaluate if this service fulfilled his needs including his perceived quality.
He can then decide to use the service again or select a different service provider when he once again needs this type of service~\citep[][]{geerts_linking_2010}.

\section{Research Question}
Of special interest is the perception of varying performance over distinct interactions with one telecommunication service and the resulting perceived quality.
%This is, in fact, a common case for telecommunication service(s).
%For example a user of a mobile telephony provider will use the provided service(s) usually repeatedly as his telephone number is bound to the service provider.
Although the impact of varying performance during one usage instance has been investigated, it is so far not known how perceived quality evolves over several distinct instances, \ie, individual experiences.
In fact, this is one factor that contributes to the \emph{service quality} of a service \citep[][]{berry_quality_1985, zeithaml_behavioral_1996}.
The construct service quality maintains a holistic view from a business perspective of service usage, focusing on aspects such as customer loyalty \citep{parasuraman_conceptual_1985}.
%In fact, perceived quality is an implicit part of service quality.

This thesis investigates perceived quality over several distinct and meaningful interactions with a service.
Such an interaction is denoted as a \emph{usage episode} (for the definition see \autoref{chap:03}).
The perceived quality of a usage episode is denoted as \emph{episodic quality}.
The perceived quality over several usage episodes is denoted as \emph{multi\-/episodic perceived quality}.

This particular thesis addresses the following research question: 
\subparagraph*{Research Question:}
How does the multi\-/episodic perceived quality for one user evolve over several usage episodes with a single service?

\subparagraph*{}
%Memory
%Each usage episode results in a perceived quality, which can be assessed momentary during the usage episode and also in retrospective after the usage episode is finished.
So far, the formation process of perceived quality of distinct usage episodes into a multi\-/episodic perceived quality is not yet known.
It could be a continuous process that integrates current experiences immediately into a current view or a retrospective assessment evaluating all memorized and recallable information \citep[][]{hogarth_order_1992}. 
In fact, it might also be a mixture of both.

%%%I focus on telecommunication services(s) as those are well-studied with regard to perceived quality, are widely used, and variations in performance, irrespective of the actual reason, occur relative frequently.

\section{Goals}
In this thesis, I pursue two goals towards understanding the formation process of multi\-/episodic perceived quality.
I focus solely on telecommunication services, because these are prone to variations in performance and usage episodes can be relatively short.
As the formation process of multi\-/episodic perceived quality cannot be observed directly (\cf{} \autoref{chap:02}), this investigation relies on the judgments of multi\-/episodic perceived quality.
%This investigation will rely on judgments of multi\-/episodic perceived quality alone as perceived quality cannot be observed directly (\cf{} \autoref{chap:02}).

This leads to two goals:

\subparagraph*{Goal 1:}
%I will investigate how episodic performance determines the judgment of multi\-/episodic perceived quality and investigate potential anomalies of the formation process.
To investigate how the performance of a sequence of usage episodes determines judgments of multi\-/episodic perceived quality.

\subparagraph*{Goal 2:}
To investigate how multi\-/episodic judgments can be predicted based on episodic judgments.

\subparagraph*{}
I conduct this investigation for two \emph{usage periods}.
Usage periods denote the time frame in which a user interacts repeatedly with a service.
First, I investigate multi\-/episodic perceived quality in one session consisting of several usage episodes.
The duration of one session is chosen here to be up to \unit[45]{min} long, as this is a common duration for subjective experiments on perceived quality.
Second, I investigate the formation process of multi\-/episodic perceived quality over a usage period of several days.
These two usage periods provide an initial starting point for the investigation of the formation process of multi\-/episodic perceived quality.
Considering two different usage periods is necessary because it is not yet known if the time between usage episodes affects the formation process of multi\-/episodic perceived quality.

\section{Structure}
This thesis is structured as follows:
In \autoref{chap:02}, I introduce concepts and fundamentals that form the basis for my investigation of multi\-/episodic perceived quality.
It starts with an introduction to psychophysics and \ac{QoE}, followed by an introduction into human memory and known biases for retrospective judgments.
Following this, the state of the art on perceived quality under performance fluctuations is presented.
Finally, the prior work on multi\-/episodic perceived quality is presented and discussed.
%Here, the assessment method developed by \citet{moller_single-call_2011}, which is the following denoted as \emph{defined-use}, is presented in detail.

Subsequently, I present my work towards multi\-/episodic perceived quality. % based on the defined use method.
In \autoref{chap:towards}, I describe the considered aspects of the defined-use method as well as the hypotheses that were investigated.
These hypotheses form the basis for my investigations on multi\-/episodic perceived quality in one session (\autoref{chap:lab}) as well as over several days (\autoref{chap:field}).
%Those different time frames are necessary to investigate as it is not known if the time between usage episodes affect the formation process.
Based on the conducted experiments, I evaluate potential models for the prediction of multi\-/episodic judgments based on prior episodic judgments (\autoref{chap:modeling}).
Finally, I summarize my thesis, discuss the results, and present directions for future work in \autoref{chap:discussion}.